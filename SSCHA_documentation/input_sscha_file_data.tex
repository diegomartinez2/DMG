\documentclass{article}
\usepackage[landscape]{geometry}
\usepackage{url}
\usepackage{multicol}
\usepackage{amsmath}
\usepackage{esint}
\usepackage{bigints}
\usepackage{amsfonts}
\usepackage{xcolor}
\usepackage{tikz}
\usetikzlibrary{calc}
\usetikzlibrary{decorations.pathmorphing}
\usepackage{amsmath,amssymb}

\usepackage{colortbl}
\usepackage{xcolor}
\usepackage{mathtools}
\usepackage{amsmath,amssymb}
\usepackage{enumitem}
\usepackage{xhfill}
\makeatletter

\newcommand*\bigcdot{\mathpalette\bigcdot@{.5}}
\newcommand*\bigcdot@[2]{\mathbin{\vcenter{\hbox{\scalebox{#2}{$\m@th#1\bullet$}}}}}
\makeatother

%opening
\title{SSCHA input file}
%\author{}
\date{}

\advance\topmargin-.8in
\advance\textheight3in
\advance\textwidth3in
\advance\oddsidemargin-1.5in
\advance\evensidemargin-1.5in
\parindent0pt
\parskip2pt
\newcommand{\hr}{\centerline{\rule{3.5in}{1pt}}}
%\colorbox[HTML]{e4e4e4}{\makebox[\textwidth-2\fboxsep][l]{texto}
\newcommand{\nc}[2][]{%
	\tikz \draw [draw=black, ultra thick, #1]
	($(current page.center)-(0.5\linewidth,0)$) --
	($(current page.center)+(0.5\linewidth,0)$)
	node [midway, fill=white] {#2};
}% tomado de https://tex.stackexchange.com/questions/179425/a-new-command-of-the-form-tex

\begin{document}

%\maketitle
\begin{center}{\huge{\textbf{SSCHA input file cheatsheet}}}\\
\end{center}
\begin{multicols*}{2}
	
	\tikzstyle{mybox} = [draw=black, fill=white, very thick,
	rectangle, rounded corners, inner sep=75pt, inner ysep=10pt]
	\tikzstyle{fancytitle} =[fill=black, text=white, font=\bfseries]
	
	%--------------------------
	\begin{tikzpicture}
	\node [mybox] (box){%
		\begin{minipage}{0.3\textwidth}
		\begin{itemize}
		\addtolength{\itemsep}{-2pt}
		%\item[$g = $]
		%  9.80 m s$^{-2}$
		\item [fildyn\_prefix :] [REQUIRED] 
        Location of the files with the dynamical matrices.
        \item [nqirr :] [REQUIRED]
        The number of irriducible q points (just look how many dynamical-matrixes files are there).
		\item[t :] [REQUIRED]
		The temperature used to generate the ensemble (in Kelvin).  
		\item[tg :]
		The temperature that will be used in the minimization (in Kelvin).
		\item[minim\_struc :]
		Do SSCHA minimize the structure? [.true./.false.]
		\item[n\_random :]
		The dimension of the ensemble
		\item [meaningful\_factor :]
		The stopping criteria. The code will end the minimization after the gradient is lower than meaningful\_factor times its stochastic error.
		\item[n\_random\_eff :]
		The Kong-Liu effective sample size.When this size
		becomes lower than the given threshold the minimization is stopped,
		and you should re-generate a new ensemble. Usually in the begining you can chose 0.1 the original ensemble, and raise it to 0.4 or 0.5 when you are close to convergence. 
		\item[root\_representation :]
        Set root representation, a trick to increase the speed of the minimization and to avoid the imaginary frequency error. The python new code is not able to do simultaneously precondition and root\_representation. Options are:
\begin{itemize}
\item "normal" : normal minimization, can lead to imaginary frequencies
\item "sqrt"   : square root representation.
\item "root4"  : fourth-root represnetation, the best one (and the slowest).
\end{itemize}
		\item[precond\_wyck :]
		Preconditioning variable [.true./.false.]
		\item[preconditioning :]
		Preconditioning dynamical martix variable for a very fast minimization [.true./.false.] (preconditioning can be used only if root\_representation = "normal")
		\item [load\_bin :]
		*************************
		
		%  Constante de gases ideales$ = 8.31 \frac{\mbox{J}}{\mbox{mol K}} = 0.0821 \frac{\mbox{l atm}}{\mbox{mol K}}$
		\end{itemize}
		\end{minipage}
	};
	%---------------------------------
	\node[fancytitle, right=10pt] at (box.north west) {\&inputsscha};
	\end{tikzpicture}
	
	%---------------------------
	\begin{tikzpicture}
	\node [mybox] (box){%
		\begin{minipage}{0.3\textwidth}
			\begin{itemize}
			\item [population :]
			The population id. This is an integer that distinguish different ensembles and allows for use the same data\_dir for several minimizations.
			\item [supercell\_size :]"
			The supercell size.
			\item [data\_dir :]
			The position of the ensemble (where the data are stored). Unit of measurements must be in bohr for displacements and $Ry/bohr$ for forces and $Ry/bohr^3$ for stress tensors. Energy is in Ry.
			\item [eq\_energy :]
			Set the equilibrium energy.
			It should be the energy of the structure without fluctuations,
			it is used to separate the electronic and the vibrational energy,
			since they are usually of different order of magnitude.
			It is measured in Ry.
	    	\item[lambda\_a :]
            Force constant minimization step. Step for the force constant matrix
            \item[lambda\_w :]
            Force constant minimization step. Step for the structure			
			\item [max\_ka :]
			Maximum number of steps after which the code is automatically stopped
			\item [stress\_offset :]
			A number or a file with the stress offset.
			\item [gradi\_op :]
			Which gradient is used to trigger the stopping condition.
			By default, both of them should satisfy the meaningufl criteria.
			Options are:
			\begin{itemize}
			\item "all" - both the gradient should satisfy the meaningulf (default)
			\item "gw"  - only the wyckoff (i.e. structure) gradient.
			\item "gc"  - only the force-constant matrix gradient.
			\end{itemize} 
			\item [print\_stress :]
			Legacy flag, not used anymore, now it automatically print the stress if it finds the stress tensor inside the ensemble. 
			\item [use\_spglib :]
			Use spglib for symmetrization. [.true./.false.]
			\item[neglect\_symmetries :]
			Disable the symmetrization. Usefull if we want to
			relax a structure. [.true./.false.]
			\end{itemize}
		\end{minipage}
	};
	%---------------------------------
	\node[fancytitle, right=10pt] at (box.north west) {\&inputsscha};
	\end{tikzpicture}
	
	%---------------------------
	\begin{tikzpicture}
	\node [mybox] (box){%
		\begin{minipage}{0.3\textwidth}
\begin{itemize}
\item [type :] [REQUIRED]
Relaxation options are:
\begin{itemize}
%\item "sscha" Only one step.
\item "relax" SSCHA Relaxation	
\item "vc-relax" Variable cell SSCHA relaxation
\end{itemize}
\item [n\_configs :] [REQUIRED]
Number of configurations. This namespace is able to generate new ensembles to perform several minimizations
\item [max\_pop\_id :]
Maximun population index before stop [INTEGER]
\item [start\_pop :]
Initial population index [INTEGER]
\item [ensemble\_datadir :]
Location to save the ensemble.
\item [generate\_ensemble :]
If .false. will get the ensemble from 'inputscha'.
\item [target\_pressure :]
In GPa
\item [fix\_volume :]
[.true./.false.]
\item [bulk\_modulus :]
In GPa
\item [sobol\_sampling :]
Set the Sobol method for the extraction of the samples [.true./.false.]
\item [sobol\_scatter :]
Set the scatter value for the Sobol sampling.
\end{itemize}
		\end{minipage}
	};
	%---------------------------------
	\node[fancytitle, right=10pt] at (box.north west) {\&relax};
	\end{tikzpicture}
	
	%---------------------------
\begin{tikzpicture}
\node [mybox] (box){%
	\begin{minipage}{0.3\textwidth}
\begin{itemize}
\item [program :] [REQUIRED]
The calculator type (e.g. "quantum-espresso")
\item [k\_points :] [REQUIRED]
k grid for the electronic calculation
\item [k\_offset :]
Offset of the k grid
\item [disable\_check :]
*************************
\item [binary :]
**************************
\item [pseudo\_ :]
Here the pseudopotentials. Note they are pseudo\_ followed by the atom name (not case sensitive)
\end{itemize}
------------------------------------------

"quantum-espresso" accepted parameters:
%\begin{itemize}

%\item 
"ecutrho", "ecutwfc", "smearing", "degauss", 
"occupations", "conv\_thr", "tstress", "tprnfor",
"verbosity", "disk\_io", "input\_dft", "use\_all\_frac"

%\end{itemize}
	\end{minipage}
};
%---------------------------------
\node[fancytitle, right=10pt] at (box.north west) {\&calculator};
\end{tikzpicture}

	%---------------------------
\begin{tikzpicture}
\node [mybox] (box){%
	\begin{minipage}{0.3\textwidth}
	\begin{itemize}
	\item [save\_freq\_filename :] Set the name of the frequencies file.
	\item [save\_rho\_filename :] Set the name of the rho file.
	\item [mu\_lock\_start :] This flag set the value for locking the modes.
	\item [mu\_lock\_end :] This flag set the value for locking the modes.
	\item [mu\_free\_start :] This flag set the value for unlockin the modes.
	\item [mu\_free\_end :] This flag set the value for unlockin the modes.
	\item [project\_dyn :] Project on the mode subspace the dynamical matrix. [.true./.false.]
	\item [project\_structure :] Project on the structure [.true./.false.]
	\end{itemize}
	\end{minipage}
};
%---------------------------------
\node[fancytitle, right=10pt] at (box.north west) {\&utils};
\end{tikzpicture}	

	%---------------------------
\begin{tikzpicture}
\node [mybox] (box){%
	\begin{minipage}{0.3\textwidth}
	\begin{itemize}
\item [template :]
Look for a cluster template [.true./.false.]
\item [SSCHA\_CLUSTERS\_DIR :]
Load cluster info from this file template
\item [hostname :]
Set cluster hostname
\item [pwd :]
Set work directory
\item [account :]
\item [binary\_path :]
\item [mpicmd :]
\item [reconnect\_attempts :]
Set the maximun connection attempts
\item [port :]
\item [shell :]
\item [submit\_cmd :]
\item [queue\_directive :]
	\end{itemize}
	\end{minipage}
};
%---------------------------------
\node[fancytitle, right=10pt] at (box.north west) {\&cluster};
\end{tikzpicture}	

	%---------------------------
\begin{tikzpicture}
\node [mybox] (box){%
	\begin{minipage}{0.3\textwidth}
	\begin{itemize}

	\item [v\_nodes :]
	\item [n\_nodes :]
	\item [use\_nodes :]
	\item [v\_cpu :]
	\item [n\_cpu :]
	\item [use\_cpu :]
	\item [v\_time :]
	\item [n\_time :]
	\item [n\_pools :]
	\item [use\_time :]
	\item [v\_memory :]
	\item [max\_ram :]
	\item [ :]
	"use\_memory"
	"v\_partition"
	"partition\_name"
	"use\_partition"
	"init\_script"
	"max\_recalc"
	"batch\_size"
	"local\_workdir"
	"v\_account"
	"use\_account"
	"sshcmd"
	"scpcmd"
	"timeout"
	"job\_numbers"
	"n\_together"
	
	
	"workdir"
	
	
	\end{itemize}
	\end{minipage}
};
%---------------------------------
\node[fancytitle, right=10pt] at (box.north west) {\&cluster};
\end{tikzpicture}	

	
	
	
\end{multicols*}
\begin{abstract}
The input file perform the minimization. 

To run the SSCHA code with the input file use:
\begin{verbatim}
>>> sscha -i simple_input.in --save-data simple_input.out
\end{verbatim}
The file can have any name (often *.in).

An example of an input file:
\begin{verbatim}
!
! * * * * * * * * * * * * * * *
! *                           *
! *    VC - RELAX EXAMPLE     *
! *                           *
! * * * * * * * * * * * * * * *
!
!
! This is the input to perform the sscha minimization, followed
! by the change of the unit cell given by the stress step.
!
! This is not the recommended way to do it (you can do everything automatically)
! But usefull if you want to control manually each submission
!

&relax
type = "vc-relax"
start_pop = 2
max_pop_id = 2
generate_ensemble = .false.
fix_volume = .false.
target_pressure = 0 ! [GPa]
bulk_modulus = 15 ! [GPa]
n_configs = 1000
&end

&inputscha
n_random = 1000
data_dir = "../ensemble_data_test"
population = 2
fildyn_prefix = "../ensemble_data_test/dyn"
nqirr = 1
supercell_size =  1 1 1
Tg = 0 
T = 0
meaningful_factor = 1e-4
gradi_op = "all" 
n_random_eff = 500
print_stress = .true.
eq_energy = -144.40680397
lambda_a = 1
lambda_w = 1
root_representation = "normal"
preconditioning = .true.
max_ka= 20
/
\end{verbatim}

\end{abstract}



\end{document}
