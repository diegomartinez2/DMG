\documentclass[]{scrartcl}

%opening
\title{SSCHA input file}
\author{}

\begin{document}

\maketitle

\begin{abstract}
The input file perform the minimization. 

To run the SSCHA code with the input file use:
\begin{verbatim}
>>> sscha -i simple_input.in --save-data simple_input.out
\end{verbatim}
The file can have any name (often *.in).

An example of an input file:
\begin{verbatim}
!
! * * * * * * * * * * * * * * *
! *                           *
! *    VC - RELAX EXAMPLE     *
! *                           *
! * * * * * * * * * * * * * * *
!
!
! This is the input to perform the sscha minimization, followed
! by the change of the unit cell given by the stress step.
!
! This is not the recommended way to do it (you can do everything automatically)
! But usefull if you want to control manually each submission
!

&relax
type = "vc-relax"
start_pop = 2
max_pop_id = 2
generate_ensemble = .false.
fix_volume = .false.
target_pressure = 0 ! [GPa]
bulk_modulus = 15 ! [GPa]
n_configs = 1000
&end

&inputscha
n_random = 1000
data_dir = "../ensemble_data_test"
population = 2
fildyn_prefix = "../ensemble_data_test/dyn"
nqirr = 1
supercell_size =  1 1 1
Tg = 0 
T = 0
meaningful_factor = 1e-4
gradi_op = "all" 
n_random_eff = 500
print_stress = .true.
eq_energy = -144.40680397
lambda_a = 1
lambda_w = 1
root_representation = "normal"
preconditioning = .true.
max_ka= 20
/
\end{verbatim}

\end{abstract}

\section{\&inputsscha}

\begin{itemize}
	\item "lambda\_a"
	\item "lambda\_w"
	\item "minim\_struc"
	\item "precond\_wyck"
	\item "preconditioning"
	\item "root\_representation"
	\item "neglect\_symmetries"
	\item "n\_random\_eff"
	\item "n\_random"
	The dimension of the ensemble
	\item "meaningful\_factor"\\
	The stopping criteria. The code will end the minimization after the gradient is lower than meaningful\_factor times its stochastic error.
	\item "eq\_energy"\\
	Set the equilibrium energy.
	It should be the energy of the structure without fluctuations,
	it is used to separate the electronic and the vibrational energy,
	since they are usually of different order of magnitude.
	It is measured in Ry.
	\item "fildyn\_prefix" [REQUIRED]
	\item "nqirr" [REQUIRED]\\
	The number of irriducible q points (just look how many dynamical-matrixes files are there).
	\item "data\_dir"\\
	The position of the ensemble (where the data are stored). Unit of measurements must be in bohr for displacements and $Ry/bohr$ for forces and $Ry/bohr^3$ for stress tensors. Energy is in Ry.
	\item "load\_bin"
	\item "t" [REQUIRED]\\
	The temperature used to generate the ensemble (in Kelvin).
	\item "tg"\\
	The temperature that will be used in the minimization (in Kelvin).
	\item "supercell\_size"\\
	The supercell size.
	\item "max\_ka"
	\item "stress\_offset"
	\item "gradi\_op"\\
	Which gradient is used to trigger the stopping condition.
 By default, both of them should satisfy the meaningufl criteria.
 Options are "all", "gc", and "gw".
\begin{enumerate}
	\item "all" - both the gradient should satisfy the meaningulf (default)
	\item "gw"  - only the wyckoff (i.e. structure) gradient.
	\item "gc"  - only the force-constant matrix gradient.
\end{enumerate} 
	\item "population"\\
	The population id. This is an integer that distinguish different ensembles and allows for use the same data\_dir for several minimizations.
	\item "print\_stress"
	\item "use\_spglib"
\end{itemize}
\section{\&relax}
\begin{itemize}
	\item "type" [REQUIRED]
	\begin{itemize}
		\item "sscha"
		\item "relax"	
		\item "vc-relax"
	\end{itemize}
	\item "n\_configs" [REQUIRED]
	\item "max\_pop\_id"
	\item "start\_pop"
	\item "ensemble\_datadir"
	\item "generate\_ensemble"
	\item "target\_pressure"
	\item "fix\_volume"
	\item "bulk\_modulus"
	\item "sobol\_sampling"
	\item "sobol\_scatter"
\end{itemize}
\section{calculator}
\begin{itemize}
	\item "k\_points" [REQUIRED]
	\item "k\_offset"
	\item "disable\_check"
	\item "program" [REQUIRED]
	\item "binary"
	\item "pseudo\_"
\end{itemize}
\begin{enumerate}
	\item "quantum-espresso"
	\begin{itemize}
		\item "ecutrho", "ecutwfc", "smearing", "degauss", 
		"occupations", "conv\_thr", "tstress", "tprnfor",
		"verbosity", "disk\_io", "input\_dft", "use\_all\_frac"
	\end{itemize}
\end{enumerate}
\section{cluster}
"template"
"SSCHA\_CLUSTERS\_DIR"
"hostname"
"pwd"
"account"
"binary\_path"
"mpicmd"
"reconnect\_attempts"
"port"


"shell"
"submit\_cmd"
"queue\_directive"
"v\_nodes"
"n\_nodes"
"use\_nodes"
"v\_cpu"
"n\_cpu"
"use\_cpu"
"v\_time"
"n\_time"
"n\_pools"
"use\_time"
"v\_memory"
"max\_ram"
"use\_memory"
"v\_partition"
"partition\_name"
"use\_partition"
"init\_script"
"max\_recalc"
"batch\_size"
"local\_workdir"
"v\_account"
"use\_account"
"sshcmd"
"scpcmd"
"timeout"
"job\_numbers"
"n\_together"


"workdir"
\section{utils}
"utils"
"save\_freq\_filename"
"save\_rho\_filename"
"mu\_lock\_start"
"mu\_lock\_end"
"mu\_free\_start"
"mu\_free\_end"
"project\_dyn"
"project\_structure"
\end{document}
