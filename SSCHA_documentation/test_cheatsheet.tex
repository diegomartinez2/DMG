\documentclass{article}
\usepackage[landscape]{geometry}
\usepackage{url}
\usepackage{multicol}
\usepackage{amsmath}
\usepackage{esint}
\usepackage{bigints}
\usepackage{amsfonts}
\usepackage{xcolor}
\usepackage{tikz}
\usetikzlibrary{calc}
\usetikzlibrary{decorations.pathmorphing}
\usepackage{amsmath,amssymb}

\usepackage{colortbl}
\usepackage{xcolor}
\usepackage{mathtools}
\usepackage{amsmath,amssymb}
\usepackage{enumitem}
\usepackage{xhfill}
\makeatletter

\newcommand*\bigcdot{\mathpalette\bigcdot@{.5}}
\newcommand*\bigcdot@[2]{\mathbin{\vcenter{\hbox{\scalebox{#2}{$\m@th#1\bullet$}}}}}
\makeatother

\title{física eléctrica}
\usepackage[brazilian]{babel}
\usepackage[utf8]{inputenc}
%cambios de unidades (pt, mm, cm,ex,em,bp,dd,pc,sp) https://tex.stackexchange.com/questions/8260/what-are-the-various-units-ex-em-in-pt-bp-dd-pc-expressed-in-mm
\advance\topmargin-.8in
\advance\textheight3in
\advance\textwidth3in
\advance\oddsidemargin-1.5in
\advance\evensidemargin-1.5in
\parindent0pt
\parskip2pt
\newcommand{\hr}{\centerline{\rule{3.5in}{1pt}}}
%\colorbox[HTML]{e4e4e4}{\makebox[\textwidth-2\fboxsep][l]{texto}
\newcommand{\nc}[2][]{%
\tikz \draw [draw=black, ultra thick, #1]
    ($(current page.center)-(0.5\linewidth,0)$) --
    ($(current page.center)+(0.5\linewidth,0)$)
    node [midway, fill=white] {#2};
}% tomado de https://tex.stackexchange.com/questions/179425/a-new-command-of-the-form-tex
\begin{document}

\begin{center}{\huge{\textbf{Electricidad y Magnetismo}}}\\
\end{center}
\begin{multicols*}{3}

\tikzstyle{mybox} = [draw=black, fill=white, very thick,
    rectangle, rounded corners, inner sep=10pt, inner ysep=10pt]
\tikzstyle{fancytitle} =[fill=black, text=white, font=\bfseries]

%--------------------------
\begin{tikzpicture}
\node [mybox] (box){%
    \begin{minipage}{0.3\textwidth}
\begin{itemize}
\addtolength{\itemsep}{-2pt}
%\item[$g = $]
%  9.80 m s$^{-2}$
\item[$c = $]
  Velocidad de la luz $ = 3.00\times 10^8$ ms$^{-1}$
\item[$k = $]
  Constante de Coulomb $ = 8.9876\times 10^9$ Nm$^2C^{-2}$ %8.9875517873681764×109 N·m2/C2 (m/F)​.
\item[$\epsilon_o = $]
  Constante dieléctrica vació $ = 8.85\times 10^{-12}$ N$^{-1}$C$^2m^{-2}$ (m/F)
\item[$\mu_o = $]
  Permeabilidad del vació $ = 4\pi\times 10^{-7}$ H/m $= 1.256\times 10^{-6}$ Kgs$^{-2}$A$^{-2}$ %1.25663706 × 10-6 m kg s-2 A-2
\item[$e^{\pm} = $]
  Carga del electrón-protón  $ = 1.60\times10^{-19}$ C
\item[$m_e = $]
  Masa del electrón $ = 9.11\times10^{-31}$ kg % \\ $ = 0.511$ MeV/$c^2$
\item[$m_n = $]
  Masa de neutrón-protón $ = 1.67\times10^{-27}$ kg % \\ $ = 938$ MeV/$c^2$
\item[$N_A = $]
  Número de Avogadro $ = 6.022 \times 10^{23}$ moléculas/mol
\item[$k_B = $]
  Constante de Boltzmann $ = 1.38 \times 10^{-23} {\mbox{J}}/{\mbox{K}}$
%\item[$R = $]
%  Constante de gases ideales$ = 8.31 \frac{\mbox{J}}{\mbox{mol K}} = 0.0821 \frac{\mbox{l atm}}{\mbox{mol K}}$
%\item[ $c_{\mbox{w}} = $]
%  Calor específico agua $ = 1 {\mbox{ cal}}/({\mbox{g K}})$]
%\item[ $1 {\mbox{ cal}} $] $= 4.186 {\mbox{ J}}$]
%\item[ $\sigma = $]
%  Constante Stefan-Boltzmann $ = 5.67\times 10^{-8}\>\frac{\mbox{W}}{\mbox{m}^2 \mbox{K}^4}$
\end{itemize}
    \end{minipage}
};
%---------------------------------
\node[fancytitle, right=10pt] at (box.north west) {Constantes Fundamentales};
\end{tikzpicture}

%---------------------------
\begin{tikzpicture}
\node [mybox] (box){%
    \begin{minipage}{0.3\textwidth}
    $\Vec{F}_{ij}=k\frac{Q_iQ_j}{r^2_{ij}}\hat{r}_{ij} = \frac{1}{4\pi\epsilon_o}\frac{Q_iQ_j}{r^2_{ij}}\hat{r}_{ij}=Q_i\Vec{E}_j$\\
    $\Vec{E}_i=k\frac{Q_i}{r^2_{io}}\hat{r}_{io}$ \\
    $V=k\frac{Q_i}{r_{oi}}$
    \end{minipage}
};
%---------------------------------
\node[fancytitle, right=10pt] at (box.north west) {Distribuciones Discretas};
\end{tikzpicture}

%---------------------------
\begin{tikzpicture}
\node [mybox] (box){%
    \begin{minipage}{0.3\textwidth}
    $\lambda= \frac{q}{L}=\frac{dq}{dL}$, $dq=\lambda dL$\\
    $\sigma= \frac{q}{A}=\frac{dq}{dA}$, $dq=\sigma dA$\\
    $\rho= \frac{q}{V}=\frac{dq}{dV}$, $dq=\rho dV$
    \end{minipage}
};
%---------------------------------
\node[fancytitle, right=10pt] at (box.north west) {Densidad de Carga};
\end{tikzpicture}

%---------------------------
\begin{tikzpicture}
\node [mybox] (box){%
    \begin{minipage}{0.3\textwidth}
     \nc{Campo}\\
    $d\vec{E} = k\frac{dq}{r^2_{do}}\hat{r}_{do}$\\
    $\vec{E} = k\bigintsss \frac{\lambda dL}{r^2_{do}}\hat{r}_{do}$, $\vec{E} = k\bigintsss \frac{\sigma dA}{r^2_{do}}\hat{r}_{do}$, $\vec{E} = k\bigintssss \frac{\rho dV}{r^2_{do}}\hat{r}_{do}$\\
    $\vec{E}_{linea}=\frac{1}{2\pi\epsilon_o}\frac{\lambda}{r}\hat{r}$\\ $\vec{E}_{placa}=\frac{\sigma}{2\epsilon_o}\hat{r}_{\scriptstyle{\perp}}$\\ Entre placas opuestas $\vec{E}_{placas}=\frac{\sigma}{\epsilon_o}\hat{r}_{\scriptstyle{\perp}}$\\
    $E_i=-\frac{dV}{dx_i}$; $x_i=x,y,z$\\
    $\vec{E}=-\vec{\nabla} V$
    \end{minipage}
};
%------------ campo titulo---------------------
\node[fancytitle, right=10pt] at (box.north west) {Distribuciones Continuas};
\end{tikzpicture}
%---------------------------------
%\bigskip
%---------------------------
\begin{tikzpicture}
\node [mybox] (box){%
    \begin{minipage}{0.3\textwidth}
    \nc{Gauss}\\
    $\Phi=\bigointsss \vec{E}\cdot d\vec{A}=\frac{q_{enc}}{\epsilon_o}$\\
    \nc{Potencial}\\
    $V=\frac{U}{q}$\\
    $V=\vec{E}\cdot\vec{r}$\\
    $V-V_o=-\bigintsss \vec{E}\cdot d\vec{l}$\\
    $V=\frac{1}{4\pi\epsilon_o}\bigintsss \frac{dq}{r}$\\
    $V= k\bigintsss \frac{\lambda dL}{r_{do}}$, $V= k\bigintsss \frac{\sigma dA}{r_{do}}$, $V = k\bigintssss \frac{\rho dV}{r_{do}}$\\
    \nc{Energía}\\
    $\Delta U=-\bigintsss_a^b\vec{F}d\vec{l}=-q\bigintsss_a^b\vec{E}d\vec{l}$\\
    $\Delta U=q\Delta V=-W_{ab}=qEl$
    \end{minipage}
};
%------------ potencial titulo  ---------------------
\node[fancytitle, right=10pt] at (box.north west) {Distribuciones Continuas};
\end{tikzpicture}

%---------------------------
\begin{tikzpicture}
\node [mybox] (box){%
    \begin{minipage}{0.3\textwidth}
    $C_o\frac{q}{V}=\frac{\epsilon_oA}{d}$, con dieléctrico $C=kC_o=\frac{k\epsilon_oA}{d}$\\
    $C_p=C_1+C_2+...+C_n=\sum_i^nC_i$\\
    $C_s=1/C_1+1/C_2+...+1/C_n=(\sum_i^n\frac{1}{C_i})^{-1}$\\
    $E_{pot-elec}=U=\frac{1}{2}qV=\frac{q^2}{2C}=\frac{1}{2}CV^2$\\
    $\mu=\frac{\epsilon_oE^2}{2}=\frac{\epsilon_oV^2}{2d^2}=\frac{U}{V_{olumen}}$
    \end{minipage}
};
%---------------------------------
\node[fancytitle, right=10pt] at (box.north west) {Capacitancia};
\end{tikzpicture}








\end{multicols*}
\end{document}_oI}{4\pi R}(cos\theta_1-cos\theta_2)$\\
	\nc{Cargas en Movimiento}\\
	$r=\frac{mv}{qB}$\\
	$a=\frac{qvB}{m}=\frac{v^2}{r}$
	\end{minipage}
};
%---------------------------------
\node[fancytitle, right=10pt] at (box.north west) {Cargas en Movimiento con Presencia de Campos};
\end{tikzpicture}
\bigskip\\




\end{multicols*}
\end{document}
