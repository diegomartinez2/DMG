\newif\iftablet    %\tablettrue % format for tablets/screen
\newif\ifcalli     \callitrue  % calligraphy for letters
\newif\ifaudio     \audiotrue  % link for audio files


\iftablet
\documentclass[12pt]{book}
\else
\documentclass[twoside,12pt]{book}

\fi

\title{Vorlage Schwangerschaftstagebuch}

\usepackage[utf8]{inputenc}
\usepackage[ngerman]{babel}
%\usepackage[T1]{fontenc}
\usepackage[object=vectorian]{pgfornament}
\usepackage{graphicx}
\usepackage{color}
\usepackage{wallpaper}
\usepackage[hang,splitrule]{footmisc}
%\usepackage{wrapfig}
\usepackage{bookmark}
\usepackage{hyperref}
%\usepackage{booktabs}
\usepackage{fancyhdr}
\usepackage{lettrine}
\usepackage{geometry}
\usepackage[final,stretch=10,protrusion=true,tracking=true,spacing=on,kerning=on,expansion=true]{microtype}
\usepackage{fourier-orns}
%\usepackage{epigraph}
%\usepackage{makeidx}
\usepackage{calligra}
% \usepackage{pgfornament}
\usepackage{chngcntr}
%\usepackage{lipsum}
\usepackage{titlesec,titletoc}
% \usepackage{tikz}
\usepackage[light]{antpolt}

\hyphenation{Mi-khay-lo-vna Do-lo-khov Be-zu-khov Kar-lo-vich Lange-ron
Co-pen-ha-gen Dmi-trie-vna Pav-lo-vich Na-ta-sha Kuz-mi-ni-chna Bol-kho-vi-ti-nov
Bo-ro-di-no  Ku-tu-zov Pav-lo-vna
Ger-mers-heim Ca-me-na Lou-ana Ka-rin Ka-rolin Ma-nu-el }


% babel
\selectlanguage{german}

% graphixs
\graphicspath{{img/}}

% chngcntr
\counterwithout{figure}{chapter}
\counterwithout{table}{chapter}

% color
\definecolor{ornament}{RGB}{0,80,0}

% \definecolor{marron}{RGB}{60,30,10}
% \definecolor{darkblue}{RGB}{0,0,80}
% \definecolor{lightblue}{RGB}{80,80,80}
% \definecolor{darkgreen}{RGB}{0,80,0}
% \definecolor{darkgray}{RGB}{0,80,0}
% \definecolor{darkred}{RGB}{80,0,0}
% \definecolor{shadecolor}{rgb}{0.97,0.97,0.97}

\input Acorn.fd
\newcommand*\initfamily{\usefont{U}{Acorn}{xl}{n}}

\newcommand{\nm}{}

\iftablet
\geometry{
  paperwidth=220mm,
  paperheight=150mm,
  left=2cm,
  right=2cm,
  top=1.5cm,
  bottom=2.5cm,
%  nohead
}
\else % periodicalaureo
\geometry{
 paperwidth=170mm,
 paperheight=240mm,
 right=18.9mm, %37.8mm,
 left= 37.8mm,%18.9mm,
 top=26.7mm,
 bottom=41mm, % should be 53.3mm
 }
\fi

\renewcommand{\headrule}{{\color{ornament}
\raisebox{-2.1pt}[10pt][10pt]{\leafright} \hrulefill
\raisebox{-2.1pt}[10pt][10pt]{~~~\decofourleft \decotwo\decofourright~~~} \hrulefill
\raisebox{-2.1pt}[10pt][10pt]{ \leafleft}}
}

\renewcommand{\footnoterule}{\vspace{-0.8em}
\noindent
 {\color{ornament}\pgfornament[width=5cm]{88} }
{\color{ornament}\includegraphics[width=5cm]{tikz/88} }
\vspace{.5em} }

\fancyfoot[RO,CO,LO,RE,CE,LE]{ }
\fancyhead[RO,CO,LO,RE,CE,LE]{ }

\fancyhead[RO,LE]{\scriptsize\thepage}
\fancyfoot[RE,LO]{{\scriptsize Dein Fußzeilentext -- identisch auf jeder linken Seite}} % XXX Hier änderst du den Fußzeilentext
\fancyfoot[RO,LE]{\scriptsize\thepage}
 \fancyfoot[C]{{\color{ornament}\pgfornament[width=1cm]{69}}}
\fancyfoot[C]{{\color{ornament}\includegraphics[width=1cm]{tikz/69}}}


\fancypagestyle{plain}{%
\fancyfoot[RO,LE]{\scriptsize\thepage}
\fancyhead[RO,LE]{\scriptsize\thepage}
\fancyhead[CO]{}
\fancyfoot[LO,CO]{}
}



\titleformat{\part}
{\hbox{\begin{minipage}{\textwidth}\color{ornament}
% \pgfornament[height=1cm]{63}\hfill%
% \pgfornament[height=1cm]{71}\iftablet\pgfornament[height=1cm]{71}\fi\hfill%
% \pgfornament[height=1cm]{64}\\[1em]
\includegraphics[height=1cm]{tikz/63}\hfill%
\includegraphics[height=1cm]{tikz/71}\iftablet\includegraphics[height=1cm]{tikz/71}\fi\hfill%
\includegraphics[height=1cm]{tikz/64}\\
\end{minipage}}
\centering \LARGE\bfseries}
{\newline \thepart.}{.5em}{}[\vspace{1ex}\hbox{%
\begin{minipage}{\textwidth}\color{ornament}
% \pgfornament[height=1cm,symmetry=h]{63}\hfill%
% \pgfornament[height=1cm,symmetry=v]{71}\iftablet\pgfornament[height=1cm]{71}\fi\hfill%
% \pgfornament[height=1cm,symmetry=h]{64}\\[1em]
\scalebox{1}[-1]{\includegraphics[height=1cm]{tikz/63}}\hfill%
\scalebox{1}[-1]{\includegraphics[height=1cm]{tikz/71}}\iftablet\scalebox{1}[-1]{\includegraphics[height=1cm]{tikz/71}}\fi\hfill%
\scalebox{1}[-1]{\includegraphics[height=1cm]{tikz/64}}\\
\end{minipage}
}]


\titleformat{\chapter}
{\iftablet\vspace{-20mm}\else\vspace{-15mm}\fi\hbox{\begin{minipage}[0pt]{\textwidth}\color{ornament}
% \pgfornament[height=1cm]{63}\hfill%
% \pgfornament[height=1cm]{87}\hfill%
% \pgfornament[height=1cm]{64}\\[1em]
\includegraphics[height=1cm]{tikz/63}\hfill%
\includegraphics[height=1cm]{tikz/87}\hfill%
\includegraphics[height=1cm]{tikz/64}\\
\end{minipage}}\vspace{-.5em}
\LARGE\bfseries}
{\newline \thechapter.}{0.5em}{}%
[\iftablet\vspace{-13mm}\else\vspace{-10mm}\fi]


\titleformat{\section}
{\hbox{\begin{minipage}{\textwidth}\color{ornament}
\hfill\parbox{4cm}{\pgfornament[width=2cm]{84}\pgfornament[width=2cm]{84}}%
\raisebox{2.6pt}{\pgfornament[width=1cm,symmetry=v]{14}}
\end{minipage}}
\Large\bfseries}
{\newline \thesection.}{.5em}{}



\ifcalli
\newcommand{\calli}{\calligra}
\else
\newcommand{\calli}{\itshape}
\fi


\let\oldquote\quote
\let\oldendquote\endquote
\def\quote{\begingroup \oldquote \flushleft }
\def\endquote{\endflushleft \oldendquote \endgroup \bigskip}






\makeatletter

\def\initial{\@ifstar\@initial\@@initial}
\def\@initial#1#2{
\lettrine[lines=2, loversize=0.3, lraise=0]{``\initfamily #1}{#2}
}
\def\@@initial#1#2{
\lettrine[lines=2, loversize=0.3, lraise=0]{\initfamily #1}{#2}
}

\makeatother

\hypersetup{
  bookmarks=true,         % ¿mostrar la barra de marcadores?
  bookmarksnumbered=true,
  linktocpage=true,
  pdfborder={0 0 0},
  unicode=false,          % caracteres no latinos en los marcadores de Acrobat
  pdftoolbar=true,        % ¿mostrar la barra de herramientas de Acrobat?
  pdfmenubar=true,        % ¿mostrar el menú de Acrobat?
  pdffitwindow=false,     % ¿ajustar la ventana a la página al abrir?
  pdfnewwindow=true,      % enlaces en una nueva ventana
  colorlinks=true,        % false: enlaces con bordes; true: enlaces coloreados
  linkcolor=blue,         % color de los enlaces internos
  citecolor=blue,         % color de los enlaces a la bibliografía
  filecolor=blue,         % color de los enlaces a archivos
  urlcolor=blue,          % color de los enlaces externos
  hyperfootnotes=false,
}

\ifpdf
  \pdfinfo{
    /Author (Diego MArtínez Gutiérrez) % XXX Ingresa tu nombre
    /Title  (El Fanal) % XXX esto puedes cambiarlo
    /Subject (Ciencia Ficción) % XXX esto puedes cambiarlo
      }
\fi


\begin{document}

\selectlanguage{ngerman}
\pagestyle{empty}
%test\newpage
{\begingroup
%%TITELSEITE%%
\iftablet
\ThisCenterWallPaper{1.2}{Strand.jpg} % XXX aquí puedes insertar una imagen diferente para la vista de tablet, después de haberla subido a la carpeta img.
\else
\ThisLLCornerWallPaper{2.0}{Strand.jpg} % XXX aquí puedes insertar una imagen diferente para la vista en dispositivos que no sean tablets, después de haberla subido a la carpeta img.
\fi
\noindent
\hspace{-8mm}\parbox{12cm}{\rule{10cm}{1.6pt}\vspace*{-\baselineskip}\vspace*{2pt}\\ % Línea horizontal gruesa
\rule{10cm}{0.4pt}\\[0.2\baselineskip] % Línea horizontal fina
\parbox[c][2cm]{10cm}{\centering\textsc{\LARGE \textbf{El Fanal}}\par}\\ % XXX ¡aquí cambias tu título!]

 \noindent\rule{10cm}{0.4pt}\vspace*{-\baselineskip}\vspace{3.2pt}\\ % Línea horizontal fina
\rule{10cm}{1.6pt}}\\[\baselineskip] % Línea horizontal gruesa
\endgroup}
\cleardoublepage
{\begingroup
% INNENTITEL
\centering
\vfill
\parbox{\textwidth}{%
\rule{\textwidth}{1.6pt}\vspace*{-\baselineskip}\vspace*{2pt} % Línea horizontal gruesa
\rule{\textwidth}{0.4pt}\\[0.2\baselineskip] % Línea horizontal fina
\parbox{\textwidth}{%
\parbox[c][1.5cm][c]{1.5cm}{\includegraphics[height=1.5cm]{tikz/19}}
~ \hrulefill ~
\parbox[c][2cm][c]{7cm}{\centering\textsc{\Large \textbf{Mi diario de embarazo}}}
 ~\hrulefill ~
\parbox[c][1.5cm][c]{1.5cm}{\scalebox{-1}[1]{\includegraphics[height=1.5cm]{tikz/19}}}}\\[0.2\baselineskip] % Título
\rule{\textwidth}{0.4pt}\vspace*{-\baselineskip}\vspace{3.2pt} % Línea horizontal fina
\rule{\textwidth}{1.6pt}}\\[\baselineskip] % Línea horizontal gruesa
\scshape % Letras versalitas
Como parte de la crónica familiar \\[0.2\baselineskip] % XXX Puedes eliminar esto

\vspace*{2\baselineskip} % W
escrito por \\[0.5\baselineskip]
{\Large TU NOMBRE}  \\
\Large
TU NOMBRE % XXX Aquí puedes añadir un segundo nombre. Si no lo necesitas, puedes eliminar esta sección desde "\\" hasta después de este texto azul.
\vfill
{Con la colaboración de  \\ {Abuela / Abuelo, etc.}} % XXX Aquí puedes añadir más nombres. Si no quieres añadir ninguno, puedes eliminar esta sección desde "escrito por" hasta después de este texto azul.
\vfill \vfill
{Fotos de  \\ {Tu fotógrafo}\par} % XXX Aquí puedes añadir el nombre de tu fotógrafo o eliminar esta sección desde "{Fotos de" hasta después de este texto azul.

\vfill %

{\scshape 2020} \\[0.3\baselineskip] % XXX Aquí puedes insertar el año de publicación / finalización, o eliminar la línea desde "{\scshape" hasta después de este texto azul.

\endgroup}
\iftablet \newpage \else \cleardoublepage \fi

\pagestyle{fancy}
\pagenumbering{roman}
\tableofcontents \iftablet \newpage \else \cleardoublepage \fi
% XXX Wenn du kein Inhaltsverzeichnis haben möchtest, kannst du diese Zeile ab \tableofcontents löschen.

\pagenumbering{arabic}
\makeatletter
\let\Oldpart\part
\def\part{\@ifstar\@DASTpart\@@DASTpart}
\def\@DASTpart#1{
\cleardoublepage
\phantomsection
\bookmarksetup{startatroot}% this is it
\fancyhead[LE,LO]{} %Kapitel
\Oldpart*{#1}
\fancyhead[RE,RO]{}
\fancyhead[LE,LO]{\scriptsize #1}
}
\def\@@DASTpart#1{
\Oldpart{#1}
\fancyhead[LE,LO]{}
\fancyhead[RE,RO]{}
\fancyhead[LE,LO]{ \scriptsize \partname ~ \thepart : ~ #1}
}

\let\Oldchapter\chapter
\def\chapter{\@ifstar\@DASTchapter\@@DASTchapter}
\def\@DASTchapter#1{
\Oldchapter*{#1}
\fancyhead[RE,RO]{ \scriptsize #1}
\addcontentsline{toc}{chapter}{#1}
}
\def\@@DASTchapter#1{
\Oldchapter{#1}
\fancyhead[RE,CO]{ \scriptsize \chaptername ~ \thechapter : ~ #1}
}
\makeatother


% XXX Setze vor die Teile, die nicht im PDF auftauchen sollen, ein %. Dadurch wird die ganze Zeile blau. Der Code wird nicht mehr gelesen.
\part*{Einleitung}

% % % % % % % % % % % % % % % % % % % % % % % % % % % % % % % % %
% % % % % % % % % % % % % % % % % % % % % % % % % % % % % % % % %
% % % % % % % % % % % % % % % % % % % % % % % % % % % % % % % % %
% % % % % % % % % % % % % % % % % % % % % % % % % % % % % % % % %
% % % % % % % % % % % % % % % % % % % % % % % % % % % % % % % % %
% % % % % % % % % % % % % % % % % % % % % % % % % % % % % % % % %
% % % % % % % % % % % % % % % % % % % % % % % % % % % % % % % % %
% % % % % % % % % % % % % % % % % % % % % % % % % % % % % % % % %
% % % % % % % % % % % % % % % % % % % % % % % % % % % % % % % % %
% % % % % % % % % % % % % % % % % % % % % % % % % % % % % % % % %
% % % % % % % % % % % % % % % % % % % % % % % % % % % % % % % % %
% % % % % % % % % % % % % % % % % % % % % % % % % % % % % %

\chapter{Gebrauchsanleitung für diese Vorlage}

Willkommen in deiner Vorlage für dein Schwangerschaftstagebuch! Ich gehe davon aus, dass du mit der Software \LaTeX{} bisher nichts zu tun hattest und auch nicht in die wunderbare Welt des Programmierens einsteigen willst. Deshalb habe ich mir zum Ziel gesetzt, diese Vorlage nicht nur als Endergebnis optisch schön zu gestalten, sondern auch den Code möglichst so einfach zu halten, dass du immer weißt, was gerade zu tun ist.

Ich führe dich deshalb an dieser Stelle nur sehr oberflächlich durch das Dokument. Wann immer du Hilfe benötigst, sende mir gerne eine Mail an \href{mailto:meine@geburtsgeschichte.de}{meine@geburtsgeschichte.de}.

\section{Sich in Overleaf zurechtfinden}

Normalerweise ist dein Overleaf-Bildschirm dreigeteilt. Links siehst du die Dateien im Projekt, in der Mitte den Code und rechts das PDF-Ergebnis.

\subsection{PDF-Ergebnis}
Um zu sehen, wie das Ergebnis deiner Codierung als PDF aussieht, kannst du einfach rechts oben auf den grünen Button \emph{Recompile} drücken. Overleaf lädt dann eine aktuelle Version deines PDFs. Du kannst nach oben und unten scrollen, um zu sehen, wie die einzelnen Seiten aussehen.

\subsection{Code-Eingabe}
In der Mitte ist die Code-Eingabe. Wenn du dieses Dokument das erste Mal öffnest, müsste dort die Code-Eingabe für das Dokument \emph{main} liegen, also die Hauptdatei. Hast du eine Datei angeklickt und startest Overleaf später neu, siehst du jeweils den Code der Datei, die du zuletzt links im Bereich der Projektdateien ausgewählt hast.


\subsection{Projektdateien}
Die Projektdateien ganz links kannst du dir wie einzelne Dateien in einem Ordner vorstellen. Erst aus dem Zusammenspiel dieser einzelnen Dateien kann Overleaf dein PDF generieren. Lösche also keine der Dateien vorschnell.

Die meisten der Dateien kannst du ignorieren. Sie beinhalten Code, den du nicht verändern willst. \newline
\textbf{Dateien, die du nicht anrühren brauchst}
\begin{itemize}
       \item pgfornament
    \item tikz
    \    \item DASTpackages
    \item DASTsec
     \item latexmkrc
\end{itemize}
\textbf{Dateien, die du verändern kannst
}\begin{itemize}
    \item Ordner \textbf{img}: Hier liegen Bilder. Ich habe ein paar Beispiele hochgeladen. Das Bild "Strand" ist das aktuelle Titelbild.
    \item Ordner \textbf{Inhalt}: In diesem Ordner liegen die Dateien, in die du die einzelnen Tagebucheinträge schreibst.
    \item \textbf{DASTmeta}: Hier kannst du die PDF-Daten ändern. Wenn du also dein Dokument als PDF speicherst, werden diese Dinge in den Meta-Daten angegeben.
    \item \textbf{main}: Hier kannst du Teile aus dem PDF nehmen, z.B., wenn du keine Kapitel über das Wochenbett schreiben willst.
    \item \textbf{DASTmacros}: Hier kannst du genau eine Sache verändern, nämlich deinen Fußzeilentext.
    \item \textbf{titlepage}: Hier gibt es eine Menge zu bearbeiten. Wir schauen uns das später noch genauer an.

\end{itemize}



Wenn du eine der entsprechenden Dateien links anklickst, öffnet sich ein neues mittleres Fenster. Sprich: Du siehst den Code, der in diesem Ordner liegt. Damit du dich möglichst wenig mit Code zurechtfinden musst, habe ich in die Ordner, die dich hauptsächlich interessieren, kaum Code gepackt. Du kannst dort vielmehr direkt anfangen, zu schreiben. Wie das geht, zeige ich dir jetzt!

\section{Deine persönlichen Daten und die Titelseite ändern}
Du willst natürlich kein Schwangerschaftstagebuch, auf dem nur plump "Tagebuch" steht. Es soll DEIN Name da stehen! Deshalb solltest du deine persönlichen Daten entsprechend eingeben.\newline

Damit du in den entsprechenden Dokumenten schnell die Stellen findest, die du ändern willst, habe ich sie immer mit \emph{XXX} markiert. Du kannst also einfach die Suchfunktion nutzen. Wichtig: Klicke erst den Code-Teil deines Bildschirmes an, sonst erkennt \LaTeX{} nicht, wo es suchen soll. Dann drückst du gleichzeitig STRG+F, also die normale Suchfunktion. Du kannst dann in das Feld \emph{XXX} eingeben und auf Suchen klicken. Das funktioniert wie im Internetbrowser oder in Word oder Writer.\newline

Es gibt schwarzen und blauen Code. Der schwarze Code wird tatsächlich als Code ausgelesen. Der blaue Code steht immer hinter einem \% und wird nicht ausgelesen. Das sind also Notizen -- so ähnlich, wie Kommentare, die du am Ende nicht mit druckst. Diese Kommentare zeigen dir, was ein bestimmter Codeabschnitt bewirkt.

Das kannst du in den folgenden Dokumenten tun:

\begin{itemize}
\item Titlepage (engl., für Deckblatt oder Titelseite)
\begin{description}
\item[Deckblatttitel] Du kannst dir einen eigenen Titel aussuchen.
\item[Innentitel] Auch hier kannst du den Text ändern
\item[Als Teil der Familienchronik] Kannst du löschen oder ändern
\item[Geschrieben von] Trage den oder die Hauptautor*innen ein
\item[Mitarbeit] Wenn du nichts einträgst, kannst du das löschen.
\item[Fotos von] Auch das kannst du löschen, wenn du niemanden einträgst
\item[Jahr] Das kannst du aktualisieren oder löschen
\end{description}
    \item IMG: Lade ein neues Bild in den Ordner hoch, um es als Titelbild zu verwenden
\end{itemize}


\section{Fußzeilentext ändern}
Wenn du den Fußzeilentext ändern möchtest, klick das Dokument \emph{DASTmacros} an. In Zeile 70 findest du den Fußzeilentext.

\section{Bilder hochladen}
Wenn du ein Bild hochladen möchtest, kannst du das ganz einfach über den Button \emph{Upload} in der linken oberen Ecke tun. Die Bilder landen dann im Ordner \textit{img}. Um ein Bild ins Dokument einzufügen, gibst du folgenden Code an der entsprechenden Stelle ein: \textbackslash includegraphics $\{Bildname\}$. Hierbei ersetzt du \emph{Bildname} durch den Namen deines Bildes.

\section{Meta-Daten des PDFs anpassen}
In dieser Datei kannst du anpassen, welche Daten das PDF nachher als Metadaten übernimmt.

\section{Auswählen, welche Kapitel im PDF angezeigt werden sollen}
Im Dokument \emph{main} kannst du zum einen auswählen, ob du ein Inhaltsverzeichnis haben möchtest, oder nicht. \newline

Außerdem kannst du auswählen, welche Teile des Dokuments überhaupt im PDF auftauchen sollen. Dafür setzt du vor alle Teile, die nicht auftauchen sollen, ein \%. Zum Beispiel vor diese Einleitung, damit mein Text nicht mit in deinem PDF landet!


\section{Inhalt schreiben}
Nach so viel Einführung kommen wir nun zum Inhalt! Darum bist du schließlich hier... Den Inhalt zu schreiben, wird die vermutlich jetzt total einfach vorkommen. Das ist es nämlich auch. Du klickst einfach das gewünscht Kapitel links in der Auswahl an, z.B. das Kapitel \emph{Vorgeschichte}. Sobald sich der Codeeingabebereich geöffnet hat, kannst du lostippen.

\subsection{Gliederung}
Du kannst verschiedene Typen von Überschriften nutzen. Folgende Befehle sind dafür nötig. (Setzte immer erst den \textbackslash und dann das Wort direkt dahinter -- ohne Leerzeichen! In die geschweiften Klammern trägst du dann den Text ein, der als entsprechende Überschrift angezeigt werden soll.
\begin{itemize}
    \item \textbackslash \textit{part}: Das sind die obersten Kapitel. Ich habe sie für die Abgrenzung zwischen Einleitung, Schwangerschaft, Geburt und Wochenbett genutzt.
    \item \textbackslash \textit{chapter}: Das ist die nächst höhere Gliederungsebende. Ich habe zum Beispiel jeden Schwangerschaftsmonat als ein Chapter bezeichnet.
    \item \textbackslash se\textit{}ction: Das ist die nächste Gliederungsebene. Beispiel: Die Überschrift \emph{Stillen} im Kapitel \emph{Erste Tage}
    \item \textbackslash \textit{subsection}: Beispiel: Die Überschrift \emph{Oma und Opa} im Kapitel \emph{Erste Tage}
    \item \textbackslash \textit{subsubsection}: Beispiel: Die Überschrift \emph{der ältere Bruder} im Kapitel im Kapitel \emph{Erste Tage}
\end{itemize}

\subsubsection{Paragraphenüberschriften}
Mit dem Befehl \textbackslash paragraph kannst du auch innerhalb eines Abschnittes noch fett gedruckte Abschnittsüberschriften einfügen.


\subsubsection{Mit oder ohne Numerierung}
Wenn du zwischen den Befehl und die geschweifte Klammer ein Sternchen setzt, so wie dieses \*, werden die Kapitel nicht nummeriert. Ohne Sternchen werden sie nummeriert.

\subsubsection{Absätze}
Wenn du im Code einfach nur einen Zeilenumsprung nutzt \textit{(ENTER-Taste)}, wird \LaTeX{} eine neue Zeile anfangen, aber ohne Leerzeile. Wenn du noch eine Leerzeile dazwischen haben willst, gib am Ende des vorigen Absatzes den Befehl \textbackslash \textit{newline} ein.

\subsection{Kalligrafie zum Kapitelanfang}

Wenn du die kalligrafischen Buchstaben als Kapiteleinstieg nutzen willst, brauchst du folgenden Code (direkt hintereinander, ohne Zeilenumbruch!):

\begin{verbatim}
    \lettrine[lines=2, loversize=0.3, lraise=0]
    {\initfamily I}{ch}
\end{verbatim}

Diesen Code kopierst du an die entsprechende Stelle. Den ersten Buchstaben des Wortes gibst du dann anstatt des \emph{I} ein, die weiteren Buchstaben statt \emph{ch}.

\subsection{Text}
Nicht umsonst sind ganze Bücher über \LaTeX{} geschrieben worden. Dies hier ist also lediglich eine ganz ganz kurze Liste der häufigsten Befehle, die du beim Schreiben brauchst:




\paragraph{Text Fett drucken} \textbf{Ergebnis}
\begin{verbatim}
    \textbf{Text} \end{verbatim}
\paragraph{Text kursiv drucken} \textit{Ergebnis} \begin{verbatim}
    \textit{Text} \end{verbatim}
\paragraph{Aufzählungen mit Punkten}  \begin{itemize}
        \item Ergebnis Zeile 1
        \item Ergebnis Zeile 2
    \end{itemize}
    \begin{verbatim}\begin{itemize}
\item Hier trägst du Text ein.
\item Für jeden neuen Punkt nutzt du wieder "\item"
\end{itemize}    \end{verbatim}
\paragraph{Aufzählungen mit Nummern}
\begin{enumerate}
        \item Ergebnis Zeile 1
        \item Ergebnis Zeile 2
    \end{enumerate}
\begin{verbatim}\begin{enumerate}
\item Hier trägst du Text ein.
\item Für jeden neuen Punkt nutzt du wieder "\item"
\end{enumerate}
\end{verbatim}
\paragraph{Zeilenumbrüche} \begin{verbatim}
    \newline
\end{verbatim}
\paragraph{Direkte Rede} Damit die Anführungszeichen korrekt angezeigt werden, kannst du nicht einfach das Gänsefüßchen oberhalb der 2 auf der Tastatur nutzen. "Dabei kommt das hier raus: Korrekte englische Zeichensetzung." Du willst aber deutsche Zeichensetzung, \glqq so wie in diesem Beispiel.\grqq~ Um das zu erreichen, setzt du den folgenden Code vor die wörtliche Rede: \textit{\textbackslash glqq} und diesen dahinter: \textit{\textbackslash grqq}. Hierzu gibt es eine Eselsbrücke: gl steht für \textit{Gänsefüßchen links} und gr steht für \textit{Gänsefüßchen rechts}. Die zwei kleinen q hintereinander sehen fast aus wie Gänsefüßchen.
\paragraph{Kalligrafie}
\begin{quote} \calli
Dies ist ein Beispiel für einen Paragraph geschrieben mit dem Befehl CALLI. Es ist etwas schwierig zu lesen, aber vielleicht sinnvoll für kurze Abschnitte.\newline

Der Befehl ist zusammen mit dem QUOTE-Befehl genutzt, so dass auch die Formatierung anders ist.
\end{quote}
\begin{verbatim}
\begin{quote}
\calli
TEXT
\end{quote}
\end{verbatim}

\paragraph{Verlinkungen ins Internet}
Wenn du einen Link einfach als Text einfügst, wird dieser unter Umständen falsch ausgegeben. \href{www.online-geburtsvorbereitungskurse.de}{Ansprechende Links sehen so aus}. Und so bekommst du sie hin:

\begin{verbatim}
\href{www.beispiel.de}{Der angezeigte Text}
\end{verbatim}


\section{Fehlerbehebung}

manchmal schreibt man einfach einen Text, ist fertig, klickt auf \emph{recompile} und \LaTeX{} spuckt irgendeinen Fehler aus. Das ist frustrierend -- erst recht, weil es für Einsteiger*innen nicht immer einfach ist, den Fehler schnell zu beheben. Manchmal hast du einfach nur eine geschweifte Klammer vergessen; manchmal ist der Fehler tiefer versteckt.

Folgende Ideen habe ich, wie du Fehler beheben kannst:
\begin{enumerate}
    \item Noch mal neu kompilieren: Falls der Fehler aufgrund eines Abbruchs des Kompilierungsprozesses zu Stande kam, hilft es, einfach neu zu kompilieren.
    \item Schau dir die Fehlermeldung an. Neben dem \emph{Recompile}-Button siehst du in roter Schrift die Anzahl der Fehler. Wenn du sie anklickst, kannst du nachlesen, wo \LaTeX{} den Fehler gefunden hat. Da steht dann zum einen, in welchem Teildokument der Fehler auftrat (z.B. \textit{Inhalt/01-}) und auch, in welcher Zeile. Du weißt also, wo du suchen musst.
    \item Falls du tatsächlich vorhest, dich eigehender mit \LaTeX{} auseinanderzusetzen, bietet es sich an, einen der vielen kostenlosen Einstiegskurse zu nutzen. \begin{itemize}
        \item Auf deutsch finde ich diese Anleitung hier als Einstieg sinnvoll: \href{https://www.heise.de/download/blog/Einfuehrung-in-LaTeX-3599742}{\LaTeX-Einleitung auf heise.de}
        \item Es gibt aber auch ein ganzes Wiki: \href{https://en.wikibooks.org/wiki/LaTeX}{Enlisches \LaTeX-Wiki}, \href{https://de.wikibooks.org/wiki/LaTeX-Kompendium}{Deutsches \LaTeX-Wiki}
         \item Auf Englisch gibt es auf dem Overleaf-Blog eine gute Zusammenfassung: \href{https://www.overleaf.com/learn/latex/Learn_LaTeX_in_30_minutes}{Learn \LaTeX{} in 30 minutes}
         \end{itemize}
         \item Bei ganz speziellen Fragen und Problemen empfehle ich das spezialisierte Forum auf Stackoverflow. Dort helfen dir erfahrene Nutzer*innen. Die Forumsregeln sind streng. Lies dich ein, nutz Minimalbeispiele und sei bereit, selber dein Hirn anzuschalten. Dann wirst du dort auf alles eine Antwort bekommen. Nachteil: Das Forum ist auf Englisch. \href{https://tex.stackexchange.com/}{Hier geht es zum \LaTeX-Forum auf Stackeverflow}.
         \item Frag mich. Ich bin im Vergleich zu den Expter*innen, die sich auf Stackoverflow tummeln, immer noch grün hinter den Ohren. Aber wer weiß, vielleicht kann ich dir helfen. Schreib mir eine Email an \href{mailto:meine@geburtsgeschichte.de}{meine@geburtsgeschichte.de}. Ich schaue, was ich tun kann.
\end{enumerate}

\section{Speichern und Drucken}
Oberhalb der PDF-Ansicht findest du den Button zum Herunterladen deines PDF-Dokuments. Du kannst es dann abspeichern, drucken und per Email an alle Menschen verschicken, die du magst.

\section{Du magst nicht mehr}
Vielleicht hast du es zuerst versucht, oder du hast von vornherein entschieden, dass dir \LaTeX{} nicht liegt. Du möchtest lieber dein Tagebuch anders schreiben. Ärger dich nicht. Genauso ging es mir auch.\newline

Überleg dir, was der richtige Weg für dich ist:
\begin{itemize}
    \item Vielleicht wirst du lieber ein handschriftliches Tagebuch führen. Ich stelle dir auf hierzu eine Vorlage zur Verfügung. Schau einfach auf \href{www.ichgebaere.com/schwangerschaftstagebuch}{Ich Gebäre} nach der Vorlage zum Ausdrucken.
    \item Vielleicht möchtest du aber auch, dass ich das Buch für dich schreibe oder formatiere. Melde dich in diesem Fall gern unter \href{mailto:meine@geburtsgeschichte.de}{meine@geburtsgeschichte.de}
\end{itemize}

Wie auch immer du dich entscheidest. Die Zeit der Schwangerschaft und Geburt ist einzigartig. Ich wünsche dir viel Spaß. Lass dich von niemandem verrückt machen. Tu, was euch gut tut. Genieße. \newline

Hör auf dein Herz.\newline

{\LARGE{\calli Deine Katharina}}

\chapter*{Widmung}


\lettrine[lines=2, loversize=0.3, lraise=0]{\initfamily I}{ch} widme diese Vorlage als erstes allen Menschen in der wunderbaren LaTeX-Community, die mich in den letzten zehn Jahren vor Nervenzusammenbrüchen wegen nahender Abgabetermine bewahrt haben. Ihr seid großartig, und das wisst ihr hoffentlich auch. Namentlich zu nennen sind an dieser Stelle \href{http://www.noatamir.com/}{Noa Tamir} und \href{https://www.zib.de/de/members/drachmann}{Marcel Drachmann}, die zwar mit diesem Projekt nichts zu tun hatten, aber die Grundlagen dafür legten, dass ich mich überhaupt an so etwas traue.

\initial*{D}{iese} Vorlage entstand im Jahr 2020 auf Grundlage der grandiosen Arbeit von Daniel Studzinski. Er nutzt seine Vorlage für \emph{Krieg und Frieden} von Leo Tolstoi. Krieg und Frieden ist \href{https://www.overleaf.com/articles/war-and-peace/kdgcwgqzrrfd}{hier}\ zum Download verfügbar. \

\part*{Die Vorgeschichte}

% % % % % % % % % % % % % % % % % % % % % % % % % % % % % % % % %
% % % % % % % % % % % % % % % % % % % % % % % % % % % % % % % % %
% % % % % % % % % % % % % % % % % % % % % % % % % % % % % % % % %
% % % % % % % % % % % % % % % % % % % % % % % % % % % % % % % % %
% % % % % % % % % % % % % % % % % % % % % % % % % % % % % % % % %
% % % % % % % % % % % % % % % % % % % % % % % % % % % % % % % % %
% % % % % % % % % % % % % % % % % % % % % % % % % % % % % % % % %
% % % % % % % % % % % % % % % % % % % % % % % % % % % % % % % % %
% % % % % % % % % % % % % % % % % % % % % % % % % % % % % % % % %
% % % % % % % % % % % % % % % % % % % % % % % % % % % % % % % % %
% % % % % % % % % % % % % % % % % % % % % % % % % % % % % % % % %
% % % % % % % % % % % % % % % % % % % % % % % % % % % % % %

\chapter*{Wie wir beschlossen, ein Baby zu bekommen}

\lettrine[lines=2, loversize=0.3, lraise=0]{\initfamily U}{nsere}

\section*{Teil 1}
\lettrine[lines=2, loversize=0.3, lraise=0]{\initfamily U}{nsere}
\section*{Teil 2}

Und noch ein zweiter Teil. Mit Buchstabe. \lettrine[lines=2, loversize=0.3, lraise=0]{\initfamily U}{nsere}

\chapter{El Fanal Brilla en Urano}

\lettrine[lines=2, loversize=0.3, lraise=0]{E}{n} la sala de crisis de la Casa Blanca, el aire estaba cargado de una furia contenida, como si las paredes mismas exhalasen el peso de un agravio cósmico. El presidente de los Estados Unidos, con el rostro endurecido por la incredulidad, golpeó un informe contra la mesa de roble, un sonido seco que resonó como el crujir de un trueno lejano.

\begin{quote}
\calli
\glqq ¿Me dicen que este\ldots \emph{Fanal}, un artilugio de una empresa europea insignificante, llegó a Urano en un suspiro de cero coma dieciséis segundos?\grqq~ Su voz no era un grito, sino un murmullo afilado, más peligroso que cualquier alarido. \glqq ¿Y que la NASA, la ESA, e incluso los chinos han confirmado los datos? ¡Hemos derramado océanos de dinero en propulsión durante décadas, y ni un maldito telegrama nos avisa de esto!\grqq
\end{quote}

El asesor de seguridad nacional, un hombre de gafas gruesas y nervios crispados, ajustó sus lentes como si intentara alinear las estrellas mismas. \glqq Señor, los datos son incontrovertibles. Las mediciones de desplazamiento espacio-temporal coinciden en las bases de todas las agencias. El satélite, registrado en Luxemburgo, se desvaneció de nuestro radar subluminal al encenderse y reapareció en la órbita de Urano, en el tiempo y lugar exactos predichos. El salto es real.\grqq

El presidente se dejó caer en su silla, como un titán abatido por el peso de su propia corona. \glqq Es una humillación. Una afrenta de proporciones estelares. Y para colmo, leo que parte del dinero vino de la Unión Europea, de un tal `Eurociencia 2030'. ¿Dónde estaban nuestros espías? ¿Qué hacíamos nosotros?\grqq~ El almirante de la Marina, mudo hasta entonces, se removió inquieto, su rostro teñido de una vergüenza que parecía manar de las profundidades del océano.

Mientras tanto, al otro lado del mundo, en un búnker subterráneo en las afueras de Pekín, la escena era un espejo de furia, aunque envuelta en una calma gélida. Un oficial del Ministerio de Seguridad del Estado, con la precisión de un cirujano, presentó una pantalla de datos a su superior.

\begin{quote}
\calli
\glqq La tecnología es real. El motor FTL funciona. Nuestro satélite de reconocimiento captó el evento, aunque no pudimos seguir la trayectoria. La firma energética es distinta a todo lo que conocemos.\grqq
\end{quote}

El general, con una calma que parecía tallada en jade, asintió. \glqq No me inquieta la tecnología; ya la descifraremos. Me inquieta su origen. ¿Quién la creó? ¿Quién es el hombre detrás de esto?\grqq~ El oficial tecleó con dedos rápidos, y el rostro de Joris de Vries, el CEO holandés de Stellarius Lux, apareció en la pantalla, pálido como un amanecer invernal. \glqq Es el rostro público. Pero el cerebro, el artífice del motor, solo lo conocemos por un nombre: Louis. Ningún agente en Europa ha podido identificarlo.\grqq

El general frunció el ceño, su mirada cortante como una espada samurái. \glqq ¿Quién es este Louis? Quiero su rostro, su historia, su sombra. ¡Ahora!\grqq

\section{La Caza del Fantasma}

Lo que la CIA y el MSS ignoraban, y lo que Louis Martin había logrado sin proponérselo, era convertirse en un espectro en un mundo de ojos vigilantes. No era un maestro del subterfugio ni un espía adiestrado; era simplemente un hombre tocado por la casualidad, un alma que el destino había envuelto en un manto de invisibilidad.

En Langley, un analista de la CIA, con los ojos enrojecidos por noches sin dormir, se mesaba el cabello frente a un informe. \glqq No hay fotos. Ni una. Ni un selfie en redes, ni un rastro en blogs de viajeros. Es como si no existiera.\grqq~ Había intentado una búsqueda facial inversa con la imagen borrosa de Louis en la videoconferencia con el banco, pero la pantalla de Joris, pixelada como un mosaico antiguo, frustraba cualquier esfuerzo.

En Berlín, un equipo del MSS enfrentaba el mismo enigma. Habían rastreado a Joris de Vries, fotografiándolo al salir del edificio de Stellarius Lux. En la imagen, Louis caminaba detrás, pero en el instante preciso del disparo, una paloma descomunal, como enviada por un dios bromista, cruzó volando, su ala cubriendo el rostro del científico. No había otra toma. El agente juró que era la paloma más grande de su vida, pero sus colegas solo rieron, incrédulos.

Los intentos de los espías se volvieron una comedia de desatinos:

\begin{itemize}
    \item Una cámara de vigilancia en un café vienés, donde Louis y Joris discutían negocios, se empañó con el vaho de un café recién servido, como si el aliento de la bebida conspirara con el destino.
    \item Un agente estadounidense en un aeropuerto intentó escanear a Louis desde lejos, pero un turista, absorto en un selfie, bloqueó la toma, y la cámara solo captó el destello del sol en su teléfono.
    \item La CIA hurgó en los registros de licencias de conducir de Luxemburgo, pero Louis no conducía; pedaleaba en bicicleta o se perdía en el transporte público, sin dejar huella.
    \item El MSS, en un último esfuerzo, accedió a los datos de inmigración, pero el nombre \glqq Louis Martin\grqq~ no tenía foto. Sus viajes, confinados a la zona Schengen, eran un susurro sin rostro en un mundo sin fronteras.
\end{itemize}

Mientras tanto, en las oficinas de Stellarius Lux, Louis, ajeno al torbellino que su invisibilidad desataba, servía un café a Joris con la calma de un monje. \glqq Joris,\grqq~ dijo, su voz suave como el rumor de un río, \glqq he estado pensando en el astillero. Tal vez necesitemos una aleación más resistente para los estabilizadores inerciales. Las fuerzas podrían ser\ldots\grqq

Joris, con un suspiro que parecía cargar el peso del firmamento, se frotó el puente de la nariz. \glqq Louis, por favor, solo tómate el café.\grqq

Y así, Louis Martin, sin saberlo, había logrado lo que los espías más astutos no podían: una invisibilidad perfecta en un mundo donde cada sombra era escrutada. Lo había hecho sin intentarlo, como si el universo, en un guiño travieso, lo hubiera elegido para ser su secreto mejor guardado.



% % % % % % % % % % % % % % % % % % % % % % % % % % % % % % % % %
% % % % % % % % % % % % % % % % % % % % % % % % % % % % % % % % %
% % % % % % % % % % % % % % % % % % % % % % % % % % % % % % % % %
% % % % % % % % % % % % % % % % % % % % % % % % % % % % % % % % %
% % % % % % % % % % % % % % % % % % % % % % % % % % % % % % % % %
% % % % % % % % % % % % % % % % % % % % % % % % % % % % % % % % %
% % % % % % % % % % % % % % % % % % % % % % % % % % % % % % % % %
% % % % % % % % % % % % % % % % % % % % % % % % % % % % % % % % %
% % % % % % % % % % % % % % % % % % % % % % % % % % % % % % % % %
% % % % % % % % % % % % % % % % % % % % % % % % % % % % % % % % %
% % % % % % % % % % % % % % % % % % % % % % % % % % % % % % % % %
% % % % % % % % % % % % % % % % % % % % % % % % % % % % % %

\chapter*{Mehr zur Vorgeschichte}


\lettrine[lines=2, loversize=0.3, lraise=0]{\initfamily N}{ach }
 

\part*{Die Schwangerschaft}

% % % % % % % % % % % % % % % % % % % % % % % % % % % % % % % % %
% % % % % % % % % % % % % % % % % % % % % % % % % % % % % % % % %
% % % % % % % % % % % % % % % % % % % % % % % % % % % % % % % % %
% % % % % % % % % % % % % % % % % % % % % % % % % % % % % % % % %
% % % % % % % % % % % % % % % % % % % % % % % % % % % % % % % % %
% % % % % % % % % % % % % % % % % % % % % % % % % % % % % % % % %
% % % % % % % % % % % % % % % % % % % % % % % % % % % % % % % % %
% % % % % % % % % % % % % % % % % % % % % % % % % % % % % % % % %
% % % % % % % % % % % % % % % % % % % % % % % % % % % % % % % % %
% % % % % % % % % % % % % % % % % % % % % % % % % % % % % % % % %
% % % % % % % % % % % % % % % % % % % % % % % % % % % % % % % % %
% % % % % % % % % % % % % % % % % % % % % % % % % % % % % %

\chapter*{Der erste Schwangerschaftsmonat}

\lettrine[lines=2, loversize=0.3, lraise=0]{\initfamily A}{uch}

\chapter{La Diplomacia Secreta y la Sombra de Langley}

\lettrine[lines=2, loversize=0.3, lraise=0]{E}{n} las entrañas del Departamento de Justicia de Estados Unidos, donde los relojes marcaban el tiempo como verdugos inexorables, el bloqueo de los fondos de Stellarius Lux se confirmó en menos de veinticuatro horas, un susurro burocrático que se extendió como niebla sobre un lago. La justificación oficial era un velo tenue: \glqq cumplimiento de la Ley de Secreto Bancario y revisión de actividades de exportación.\grqq~ Pero en las alturas del poder, donde las decisiones se tejían con hilos de sombra, todos conocían la verdad. La cuenta de Joris de Vries era una carnada reluciente, lanzada al mar profundo para atrapar al pez chino, que ya había mordido con avidez.

En un salón privado de un club elitista en Washington D.C., donde el humo de cigarros antiguos flotaba como fantasmas de acuerdos olvidados, un hombre ataviado con un traje que parecía cosido con precisión quirúrgica ---el subdirector de la CIA--- se sentó frente a un empresario europeo. Su nombre en clave era \glqq Señor Smith\grqq, un alias tan anodino como el aire que respiraban.

\begin{quote}
\calligra
\glqq Señor Smith,\grqq~ inició el subdirector, con una sonrisa que se curvaba en sus labios pero moría antes de alcanzar sus ojos, fríos como el acero de una daga. \glqq Su empresa enfrenta un dilema. Y a mi gobierno no le agrada que compañías extranjeras utilicen nuestro sistema bancario para financiar proyectos que podrían amenazar nuestra seguridad nacional.\grqq
\end{quote}

El \glqq Señor Smith\grqq, emisario secreto de Joris de Vries, respondió con la firmeza de un roble centenario. \glqq Esos fondos nos pertenecen. Están en un banco estadounidense porque la compañía se forjó allí. El gobierno no puede arrebatar el dinero de una empresa sin una causa justificada.\grqq

El subdirector soltó una risa hueca, como el eco de un trueno lejano en una noche sin estrellas. \glqq Causa, la tenemos. Una de nuestras invenciones. La tecnología FTL es un quiebre en el tapiz del mundo: la próxima revolución en el transporte, la guerra, la economía. No podemos permitir que caiga en manos equivocadas. Por ejemplo, en las de una potencia extranjera que ya ha extendido sus tentáculos hacia ustedes, Señor Smith, ofreciendo `ayuda'.\grqq

El rostro del emisario palideció, como si la sangre se hubiera retirado a las profundidades de su ser. El gobierno estadounidense ya sabía del contacto con China. No había escapatoria; se movían en órbitas distintas, en un baile cósmico donde uno era el sol y el otro, un mero satélite.

\begin{quote}
\calligra
\glqq ¿Qué desean?\grqq~ preguntó Smith, su voz un hilo fino, tenso como una cuerda de violín a punto de romperse.
\end{quote}

\glqq Una asociación,\grqq~ replicó el subdirector, su tono suave como seda pero afilado como una hoja. \glqq Una verdadera. El gobierno de Estados Unidos descongelará los fondos y los multiplicará, proporcionará ingenieros, laboratorios, seguridad. Todo lo que necesiten, siempre y cuando el proyecto se traslade a nuestro suelo. Y, por supuesto, que Louis Martin se convierta en consultor exclusivo del gobierno.\grqq

El emisario sabía que Joris jamás aceptaría tal condición. Louis era un genio anárquico, libre como el viento que recorre los campos sin ataduras, un alma que aborrecía los protocolos y las cadenas de las reglas.

\begin{quote}
\calligra
\glqq Señor, dudo que el Señor Martin acceda,\grqq~ dijo Smith, su voz un murmullo de resignación.
\end{quote}

El subdirector se incorporó, su voz enfriándose como el aliento del invierno. \glqq No tiene elección. Dígale a Joris que, si rechazan, su gobierno lo considerará un activo enemigo. Invocaremos la Ley de Seguridad Nacional, que permite perseguir a cualquier ciudadano o empresa que trate con un país adversario. Dígale que podemos hacer su vida un infierno, tanto a él como a Louis, cerrando puertas en todos los bancos del mundo, vigilando cada paso como sombras inseparables.\grqq

Con esa amenaza, flotando en el aire como un presagio oscuro, el gobierno estadounidense lanzó su jugada: una oferta que Joris no podía rechazar, pero que tampoco podía abrazar sin perder su alma.

\section{La Sombra de Louis}

Mientras tanto, a miles de kilómetros de distancia, ajeno al torbellino de intrigas que giraba en torno a su nombre, Louis Martin se hallaba sentado en un parque de Luxemburgo, donde los árboles susurraban secretos al viento y las hojas caídas formaban tapices de oro y carmín. Estaba frustrado, su teléfono ---un relicto antiguo, tosco como una piedra tallada--- se negaba a conectar con el Wi-Fi público, un velo invisible que lo separaba de enviar sus cálculos a Joris.

Justo entonces, una joven cercana, absorta en la pantalla de su propio dispositivo como si contuviera los misterios del universo, tropezó y cayó al suelo con un sonido sordo, como el de una fruta madura desprendiéndose de la rama. Louis, impulsado por una bondad instintiva, corrió a ayudarla. Al agacharse, el teléfono de la chica, dañado por el impacto, se conectó milagrosamente a la red, como si el destino hubiera tejido un hilo invisible entre sus aparatos.

El brillo en los ojos de Louis, al ver la conexión establecida, podía confundirse con el de la joven que le sonreía en agradecimiento, un destello de luz en medio de la penumbra cotidiana. Al levantarse, una ráfaga de viento surgió de la nada, provocada por un camión de basura que pasaba con el rugido de un dragón adormecido, y esa ráfaga impidió que una cámara de vigilancia capturara el momento, como si el aire mismo conspirara para preservar su invisibilidad. Segundos después, el teléfono de la chica se desconectó nuevamente, dejando solo el eco de un milagro efímero.

Louis, con una sonrisa que parecía nacer de las profundidades de su ser, regresó a su banco, ignorante de que, en ese instante fugaz, había esquivado una vez más las redes del mundo. El parque, con su quietud eterna, parecía guardar su secreto, un refugio donde los genios como él podían soñar sin ser vistos.

\section{El Susurro del Astillero}

En el astillero del norte de Alemania, donde el viento del Báltico cantaba una elegía salada, las grúas se alzaban como gigantes de hierro que custodiaban un secreto más antiguo que el tiempo mismo. Bajo la luz gris del amanecer, los trabajadores, con rostros curtidos por el frío y el esfuerzo, soldaban placas de una aleación desconocida, un metal que parecía beber la luz en lugar de reflejarla. Nadie sabía para qué servía, pero en sus corazones, endurecidos por años de labor, sentían que estaban forjando algo que cambiaría el destino de los hombres.

El astillero, un laberinto de acero y sal, había sido sellado al mundo exterior. Inspectores europeos, con pretextos de contaminación marina, patrullaban los muelles, mientras agentes encubiertos de potencias lejanas merodeaban en las sombras, disfrazados de pescadores o turistas despistados. Pero el secreto de \emph{El Fanal} permanecía intacto, custodiado no por hombres armados, sino por el silencio de quienes trabajaban bajo su hechizo.

En una oficina improvisada, entre planos y tazas de café que olían a noches sin fin, Joris de Vries se paseaba como un lobo enjaulado. Frente a él, Louis Martin dibujaba en un cuaderno, su lápiz trazando líneas que parecían más conjuros que ecuaciones.

\begin{quote}
\calligra
\glqq Joris,\grqq~ dijo Louis, sin alzar la vista, su voz suave como el murmullo de un arroyo, \glqq he recalculado las tolerancias del campo de deformación. Si añadimos un estabilizador cuádruple, la nave podría soportar una transición más allá de Urano. Tal vez hasta Alfa Centauri.\grqq
\end{quote}

Joris se detuvo, su rostro pálido como la niebla que cubría el astillero. \glqq Louis, no estamos listos para Alfa Centauri. Apenas podemos mantener a los americanos y a los chinos fuera de aquí. Ayer interceptaron un dron sobre el Báltico. ¡Un dron, Louis! Sabemos que era chino, pero no podemos probarlo.\grqq

Louis alzó la mirada, sus ojos brillando con una calma que parecía desafiar las leyes del tiempo. \glqq Que busquen. No encontrarán nada. \emph{El Fanal} no es una máquina que se pueda robar. Es un sueño que aún no han aprendido a soñar.\grqq

En ese momento, una paloma blanca, idéntica a la que había cruzado los cielos de Berlín y Luxemburgo, se posó en el alféizar de la ventana. Sus alas, como lienzos de un pintor celestial, reflejaban el amanecer con un resplandor que parecía prometer milagros. Joris la miró, y por un instante, sintió que el universo entero conspiraba para protegerlos.

\section{La Danza de las Sombras}

Lejos del astillero, en un café de Copenhague, una agente de la CIA conocida solo como \glqq Clara\grqq~ sorbía un té con la precisión de un reloj. Frente a ella, un informante danés, un hombre de rostro arrugado como un mapa antiguo, deslizó un sobre bajo la mesa.

\begin{quote}
\calligra
\glqq Es todo lo que tengo,\grqq~ susurró, su voz temblando como las olas del Báltico. \glqq Planos parciales del astillero. Pero no hay nada sobre el motor. Solo\ldots rumores. Dicen que el hombre detrás de todo, ese Louis, no aparece en ninguna cámara. Es como un fantasma.\grqq
\end{quote}

Clara abrió el sobre, sus dedos fríos rozando el papel. Los planos mostraban una estructura colosal, una nave que no se parecía a nada que la NASA hubiera soñado. Pero los detalles del motor, del corazón de \emph{El Fanal}, eran un vacío, un espacio en blanco que parecía burlarse de ella. \glqq ¿Y este Louis? ¿Alguien lo ha visto?\grqq~ preguntó, su voz baja pero afilada como una navaja.

El informante negó con la cabeza. \glqq Nadie. Hay quien dice que no existe, que es un nombre inventado para despistar. Pero yo vi algo\ldots en el astillero. Un hombre, delgado, con el cabello desordenado. Iba a tomar una foto, pero una paloma, maldita sea, una paloma blanca voló justo frente a mi lente. Cuando miré de nuevo, él ya no estaba.\grqq

Clara frunció el ceño, su mente girando como las hélices de un barco perdido. La paloma. Siempre la paloma. En los informes de Berlín, de Luxemburgo, de todas partes, esa criatura alada aparecía como un guardián improbable, un emisario del destino que protegía al hombre sin rostro.

En Pekín, la frustración era un río que corría bajo la superficie. Un general del Ministerio de Seguridad del Estado, con el rostro endurecido por años de secretos, revisaba un video granulado. Mostraba el astillero desde un satélite, pero en el momento crucial, cuando Louis Martin salía de un hangar, una nube pasajera oscureció la imagen, como si el cielo mismo hubiera cerrado los ojos. \glqq Otra vez,\grqq~ gruñó el general. \glqq Siempre algo. Una paloma, una nube, un reflejo. ¡Encuentren a ese hombre!\grqq

\section{El Susurro del Metal}

De vuelta en el astillero, el trabajo continuaba bajo la vigilancia de las grúas, que se movían como titanes en un vals lento. Los trabajadores, sin saberlo, eran parte de un milagro. Cada soldadura, cada placa, era un paso hacia un futuro que ninguno podía imaginar. Pero en el corazón del astillero, en un hangar sellado donde la luz apenas se atrevía a entrar, \emph{El Fanal} comenzaba a tomar forma.

Louis, solo en el hangar, tocó el metal frío de la estructura, y por un instante, sintió que vibraba con una energía que no era de este mundo. \glqq Pronto,\grqq~ susurró, como si hablara con la nave misma. \glqq Pronto cruzarás las estrellas.\grqq

La paloma, posada ahora en una viga del hangar, ladeó la cabeza, sus ojos brillando como dos perlas negras. Y en el silencio del astillero, bajo el canto del viento y el rugido lejano del mar, el universo parecía contener el aliento, esperando el momento en que \emph{El Fanal} iluminaría los confines del cosmos.

 \chapter*{Der zweite Schwangerschaftsmonat}

\lettrine[lines=2, loversize=0.3, lraise=0]{\initfamily U}{nsere}

 \chapter*{Der dritte Schwangerschaftsmonat}
\lettrine[lines=2, loversize=0.3, lraise=0]{\initfamily E}{in} Beispiel für Kalligrafie:


\begin{quote} \calli

Dies ist ein Beispiel für einen Paragraph geschrieben mit dem Befehl CALLI. Es ist etwas schwierig zu lesen, aber vielleicht sinnvoll für kurze Abschnitte.

Der Befehl ist zusammen mit dem QUOTE-Befehl genutzt, so dass auch die Formatierung anders ist.
\end{quote}

% % % % % % % % % % % % % % % % % % % % % % % % % % % % % % % % %
% % % % % % % % % % % % % % % % % % % % % % % % % % % % % % % % %
% % % % % % % % % % % % % % % % % % % % % % % % % % % % % % % % %
% % % % % % % % % % % % % % % % % % % % % % % % % % % % % % % % %
% % % % % % % % % % % % % % % % % % % % % % % % % % % % % % % % %
% % % % % % % % % % % % % % % % % % % % % % % % % % % % % % % % %
% % % % % % % % % % % % % % % % % % % % % % % % % % % % % % % % %
% % % % % % % % % % % % % % % % % % % % % % % % % % % % % % % % %
% % % % % % % % % % % % % % % % % % % % % % % % % % % % % % % % %
% % % % % % % % % % % % % % % % % % % % % % % % % % % % % % % % %
% % % % % % % % % % % % % % % % % % % % % % % % % % % % % % % % %
% % % % % % % % % % % % % % % % % % % % % % % % % % % % % %

 \chapter*{Der vierte Schwangerschaftsmonat}

\lettrine[lines=2, loversize=0.3, lraise=0]{\initfamily T}{he}

 \input{Inhalt/08- 5. Schwangerschaftsmonat}
 \input{Inhalt/09- 6. Schwangerschaftsmonat}
 \input{Inhalt/10- 7. Schwangerschaftsmonat}
 \input{Inhalt/11- 8.Schwangerschaftsmonat}
 

\chapter*{Der neunte Schwangerschaftsmonat}


\initial{I}{n} Berlin

 \input{Inhalt/13- 10.Schwangerschaftsmonat}
 \part*{Die Geburt}

% % % % % % % % % % % % % % % % % % % % % % % % % % % % % % % % %
% % % % % % % % % % % % % % % % % % % % % % % % % % % % % % % % %
% % % % % % % % % % % % % % % % % % % % % % % % % % % % % % % % %
% % % % % % % % % % % % % % % % % % % % % % % % % % % % % % % % %
% % % % % % % % % % % % % % % % % % % % % % % % % % % % % % % % %
% % % % % % % % % % % % % % % % % % % % % % % % % % % % % % % % %
% % % % % % % % % % % % % % % % % % % % % % % % % % % % % % % % %
% % % % % % % % % % % % % % % % % % % % % % % % % % % % % % % % %
% % % % % % % % % % % % % % % % % % % % % % % % % % % % % % % % %
% % % % % % % % % % % % % % % % % % % % % % % % % % % % % % % % %
% % % % % % % % % % % % % % % % % % % % % % % % % % % % % % % % %
% % % % % % % % % % % % % % % % % % % % % % % % % % % % % %

\chapter*{Ankündigungen}
\initial{D}{ie} ersten Anzeichen

\chapter*{Falscher Alarm}
\initial{D}{och}

 \part*{Das Wochenbett}

% % % % % % % % % % % % % % % % % % % % % % % % % % % % % % % % %
% % % % % % % % % % % % % % % % % % % % % % % % % % % % % % % % %
% % % % % % % % % % % % % % % % % % % % % % % % % % % % % % % % %
% % % % % % % % % % % % % % % % % % % % % % % % % % % % % % % % %
% % % % % % % % % % % % % % % % % % % % % % % % % % % % % % % % %
% % % % % % % % % % % % % % % % % % % % % % % % % % % % % % % % %
% % % % % % % % % % % % % % % % % % % % % % % % % % % % % % % % %
% % % % % % % % % % % % % % % % % % % % % % % % % % % % % % % % %
% % % % % % % % % % % % % % % % % % % % % % % % % % % % % % % % %
% % % % % % % % % % % % % % % % % % % % % % % % % % % % % % % % %
% % % % % % % % % % % % % % % % % % % % % % % % % % % % % % % % %
% % % % % % % % % % % % % % % % % % % % % % % % % % % % % %

\chapter*{Erste Tage}
\lettrine[lines=2, loversize=0.3, lraise=0]{\initfamily U}{nsere}

\section*{Stillen}

\section*{Rückbildung}

\section*{Aufnahme in die Familie}

\subsection*{Oma und Opa}

\subsection*{Ältere Geschwister}

\subsubsection*{der ältere bruder}

\subsubsection*{die ältere Schwester}

\subsection*{Nachbarschaft}

\chapter*{Die ersten Wochen}
\lettrine[lines=2, loversize=0.3, lraise=0]{\initfamily B}{ereits } nach





\end{document}
