\hyphenation{Mi-khay-lo-vna Do-lo-khov Be-zu-khov Kar-lo-vich Lange-ron
Co-pen-ha-gen Dmi-trie-vna Pav-lo-vich Na-ta-sha Kuz-mi-ni-chna Bol-kho-vi-ti-nov
Bo-ro-di-no  Ku-tu-zov Pav-lo-vna
Ger-mers-heim Ca-me-na Lou-ana Ka-rin Ka-rolin Ma-nu-el }


% babel
\selectlanguage{german}

% graphixs
\graphicspath{{img/}}

% chngcntr
\counterwithout{figure}{chapter}
\counterwithout{table}{chapter}

% color
\definecolor{ornament}{RGB}{0,80,0}

% \definecolor{marron}{RGB}{60,30,10}
% \definecolor{darkblue}{RGB}{0,0,80}
% \definecolor{lightblue}{RGB}{80,80,80}
% \definecolor{darkgreen}{RGB}{0,80,0}
% \definecolor{darkgray}{RGB}{0,80,0}
% \definecolor{darkred}{RGB}{80,0,0}
% \definecolor{shadecolor}{rgb}{0.97,0.97,0.97}

\input Acorn.fd
\newcommand*\initfamily{\usefont{U}{Acorn}{xl}{n}}

\newcommand{\nm}{}

\iftablet
\geometry{
  paperwidth=220mm,
  paperheight=150mm,
  left=2cm,
  right=2cm,
  top=1.5cm,
  bottom=2.5cm,
%  nohead
}
\else % periodicalaureo
\geometry{
 paperwidth=170mm,
 paperheight=240mm,
 right=18.9mm, %37.8mm,
 left= 37.8mm,%18.9mm,
 top=26.7mm,
 bottom=41mm, % should be 53.3mm
 }
\fi

\renewcommand{\headrule}{{\color{ornament}
\raisebox{-2.1pt}[10pt][10pt]{\leafright} \hrulefill
\raisebox{-2.1pt}[10pt][10pt]{~~~\decofourleft \decotwo\decofourright~~~} \hrulefill
\raisebox{-2.1pt}[10pt][10pt]{ \leafleft}}
}

\renewcommand{\footnoterule}{\vspace{-0.8em}
\noindent
 {\color{ornament}\pgfornament[width=5cm]{88} }
{\color{ornament}\includegraphics[width=5cm]{tikz/88} }
\vspace{.5em} }

\fancyfoot[RO,CO,LO,RE,CE,LE]{ }
\fancyhead[RO,CO,LO,RE,CE,LE]{ }

\fancyhead[RO,LE]{\scriptsize\thepage}
\fancyfoot[RE,LO]{{\scriptsize Dein Fußzeilentext -- identisch auf jeder linken Seite}} % XXX Hier änderst du den Fußzeilentext
\fancyfoot[RO,LE]{\scriptsize\thepage}
 \fancyfoot[C]{{\color{ornament}\pgfornament[width=1cm]{69}}}
\fancyfoot[C]{{\color{ornament}\includegraphics[width=1cm]{tikz/69}}}


\fancypagestyle{plain}{%
\fancyfoot[RO,LE]{\scriptsize\thepage}
\fancyhead[RO,LE]{\scriptsize\thepage}
\fancyhead[CO]{}
\fancyfoot[LO,CO]{}
}



\titleformat{\part}
{\hbox{\begin{minipage}{\textwidth}\color{ornament}
% \pgfornament[height=1cm]{63}\hfill%
% \pgfornament[height=1cm]{71}\iftablet\pgfornament[height=1cm]{71}\fi\hfill%
% \pgfornament[height=1cm]{64}\\[1em]
\includegraphics[height=1cm]{tikz/63}\hfill%
\includegraphics[height=1cm]{tikz/71}\iftablet\includegraphics[height=1cm]{tikz/71}\fi\hfill%
\includegraphics[height=1cm]{tikz/64}\\
\end{minipage}}
\centering \LARGE\bfseries}
{\newline \thepart.}{.5em}{}[\vspace{1ex}\hbox{%
\begin{minipage}{\textwidth}\color{ornament}
% \pgfornament[height=1cm,symmetry=h]{63}\hfill%
% \pgfornament[height=1cm,symmetry=v]{71}\iftablet\pgfornament[height=1cm]{71}\fi\hfill%
% \pgfornament[height=1cm,symmetry=h]{64}\\[1em]
\scalebox{1}[-1]{\includegraphics[height=1cm]{tikz/63}}\hfill%
\scalebox{1}[-1]{\includegraphics[height=1cm]{tikz/71}}\iftablet\scalebox{1}[-1]{\includegraphics[height=1cm]{tikz/71}}\fi\hfill%
\scalebox{1}[-1]{\includegraphics[height=1cm]{tikz/64}}\\
\end{minipage}
}]


\titleformat{\chapter}
{\iftablet\vspace{-20mm}\else\vspace{-15mm}\fi\hbox{\begin{minipage}[0pt]{\textwidth}\color{ornament}
% \pgfornament[height=1cm]{63}\hfill%
% \pgfornament[height=1cm]{87}\hfill%
% \pgfornament[height=1cm]{64}\\[1em]
\includegraphics[height=1cm]{tikz/63}\hfill%
\includegraphics[height=1cm]{tikz/87}\hfill%
\includegraphics[height=1cm]{tikz/64}\\
\end{minipage}}\vspace{-.5em}
\LARGE\bfseries}
{\newline \thechapter.}{0.5em}{}%
[\iftablet\vspace{-13mm}\else\vspace{-10mm}\fi]


\titleformat{\section}
{\hbox{\begin{minipage}{\textwidth}\color{ornament}
\hfill\parbox{4cm}{\pgfornament[width=2cm]{84}\pgfornament[width=2cm]{84}}%
\raisebox{2.6pt}{\pgfornament[width=1cm,symmetry=v]{14}}
\end{minipage}}
\Large\bfseries}
{\newline \thesection.}{.5em}{}



\ifcalli
\newcommand{\calli}{\calligra}
\else
\newcommand{\calli}{\itshape}
\fi


\let\oldquote\quote
\let\oldendquote\endquote
\def\quote{\begingroup \oldquote \flushleft }
\def\endquote{\endflushleft \oldendquote \endgroup \bigskip}






\makeatletter

\def\initial{\@ifstar\@initial\@@initial}
\def\@initial#1#2{
\lettrine[lines=2, loversize=0.3, lraise=0]{``\initfamily #1}{#2}
}
\def\@@initial#1#2{
\lettrine[lines=2, loversize=0.3, lraise=0]{\initfamily #1}{#2}
}

\makeatother
