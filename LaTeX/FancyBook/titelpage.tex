{\begingroup
%%TITELSEITE%%
\iftablet
\ThisCenterWallPaper{1.2}{Strand.jpg} % XXX hier kannst du für die Tablet-Ansicht ein anderes Bild eintragen, nachdem du es in den Ordner img hochgeladen hast.
\else
\ThisLLCornerWallPaper{2.0}{Strand.jpg} % XXX hier kannst du für die Ansicht für andere Geräte als ein Tablet ein anderes Bild eintragen, nachdem du es in den Ordner img hochgeladen hast.
\fi
\noindent
\hspace{-8mm}\parbox{12cm}{\rule{10cm}{1.6pt}\vspace*{-\baselineskip}\vspace*{2pt}\\ % Dicke horizontale Linie
\rule{10cm}{0.4pt}\\[0.2\baselineskip] % Dünne horizontale Linie
\parbox[c][2cm]{10cm}{\centering\textsc{\LARGE \textbf{Mein Schwangerschaftstagebuch}}\par}\\ % XXX hier änderst du deinen Titel!]

 \noindent\rule{10cm}{0.4pt}\vspace*{-\baselineskip}\vspace{3.2pt}\\ % Dünne horizontale Linie
\rule{10cm}{1.6pt}}\\[\baselineskip] % Dicke horizontale Linie
\endgroup}
\cleardoublepage
{\begingroup
% INNENTITEL
\centering
\vfill
\parbox{\textwidth}{%
\rule{\textwidth}{1.6pt}\vspace*{-\baselineskip}\vspace*{2pt} % Dicke horizontale Linie
\rule{\textwidth}{0.4pt}\\[0.2\baselineskip] % Dünne horizontale Linie
\parbox{\textwidth}{%
\parbox[c][1.5cm][c]{1.5cm}{\includegraphics[height=1.5cm]{tikz/19}}
~ \hrulefill ~
\parbox[c][2cm][c]{7cm}{\centering\textsc{\Large \textbf{Mein Schwangerschaftstagebuch}}}
 ~\hrulefill ~
\parbox[c][1.5cm][c]{1.5cm}{\scalebox{-1}[1]{\includegraphics[height=1.5cm]{tikz/19}}}}\\[0.2\baselineskip] % Title
\rule{\textwidth}{0.4pt}\vspace*{-\baselineskip}\vspace{3.2pt} % Dünne horizontale Linie
\rule{\textwidth}{1.6pt}}\\[\baselineskip] % Dicke horizontale Linie
\scshape % Small caps
Als Teil der Familienchronik \\[0.2\baselineskip] % XXX Kannst du löschen

\vspace*{2\baselineskip} % W
geschrieben von \\[0.5\baselineskip]
{\Large DEIN NAME}  \\
\Large
DEIN NAME % XXX Hier kann ein zweiter Name eingetragen werden. Falls du keinen brauchst, kannst du diesen Abschnitt ab "\\" und bis hinter diesen blauen Text löschen.
\vfill
{Unter Mitarbeit von  \\ {Oma / Opa usw.}} % XXX Hier können weitere Namen eingetragen werden. Falls du keine eintragen willst, kannst du diesen Abschnitt ab "geschrieben von" und bis hinter diesen blauen Text löschen.
\vfill \vfill
{Fotos von  \\ {Dein Fotograf}\par} %  XXX Hier kannst du deinen Fotografen eintragen oder den Abschnitt ab "{Fotos von" bis hinter diesen blauen Text löschen.

\vfill %

{\scshape 2020} \\[0.3\baselineskip] % XXX Hier kannst du das Jahr der Veröffentlichung / Fertigstellung eintragen, oder die Zeile ab "{\scshape" bis hinter diesen blauen Text löschen.

\endgroup}
