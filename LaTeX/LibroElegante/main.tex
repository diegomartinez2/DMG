\newif\iftablet    %\tablettrue % formato para tablets/pantalla
\newif\ifcalli     \callitrue  % caligrafía para letras
\newif\ifaudio     \audiotrue  % enlace para archivos de audio

\iftablet
\documentclass[12pt]{book}
\else
\documentclass[twoside,a4paper,12pt]{book}
\fi

\title{El Fanal: Relato de ciencia ficción}

\usepackage[utf8]{inputenc}
\usepackage[ngerman]{babel}
%\usepackage[T1]{fontenc}
\usepackage[object=vectorian]{pgfornament}
\usepackage{graphicx}
\usepackage{color}
\usepackage{wallpaper}
\usepackage[hang,splitrule]{footmisc}
%\usepackage{wrapfig}
\usepackage{bookmark}
\usepackage{hyperref}
%\usepackage{booktabs}
\usepackage{fancyhdr}
\usepackage{lettrine}
\usepackage{geometry}
\usepackage[final,stretch=10,protrusion=true,tracking=true,spacing=on,kerning=on,expansion=true]{microtype}
\usepackage{fourier-orns}
%\usepackage{epigraph}
%\usepackage{makeidx}
\usepackage{calligra}
% \usepackage{pgfornament}
\usepackage{chngcntr}
%\usepackage{lipsum}
\usepackage{titlesec,titletoc}
% \usepackage{tikz}
\usepackage[light]{antpolt}

\hyphenation{Mi-khay-lo-vna Do-lo-khov Be-zu-khov Kar-lo-vich Lange-ron
Co-pen-ha-gen Dmi-trie-vna Pav-lo-vich Na-ta-sha Kuz-mi-ni-chna Bol-kho-vi-ti-nov
Bo-ro-di-no  Ku-tu-zov Pav-lo-vna
Ger-mers-heim Ca-me-na Lou-ana Ka-rin Ka-rolin Ma-nu-el }


% babel
\selectlanguage{german}

% graphixs
\graphicspath{{img/}}

% chngcntr
\counterwithout{figure}{chapter}
\counterwithout{table}{chapter}

% color
\definecolor{ornament}{RGB}{0,80,0}

% \definecolor{marron}{RGB}{60,30,10}
% \definecolor{darkblue}{RGB}{0,0,80}
% \definecolor{lightblue}{RGB}{80,80,80}
% \definecolor{darkgreen}{RGB}{0,80,0}
% \definecolor{darkgray}{RGB}{0,80,0}
% \definecolor{darkred}{RGB}{80,0,0}
% \definecolor{shadecolor}{rgb}{0.97,0.97,0.97}

\input Acorn.fd
\newcommand*\initfamily{\usefont{U}{Acorn}{xl}{n}}

\newcommand{\nm}{}

\iftablet
\geometry{
  paperwidth=220mm,
  paperheight=150mm,
  left=2cm,
  right=2cm,
  top=1.5cm,
  bottom=2.5cm,
%  nohead
}
\else % periodicalaureo
\geometry{
 paperwidth=170mm,
 paperheight=240mm,
 right=18.9mm, %37.8mm,
 left= 37.8mm,%18.9mm,
 top=26.7mm,
 bottom=41mm, % should be 53.3mm
 }
\fi

\renewcommand{\headrule}{{\color{ornament}
\raisebox{-2.1pt}[10pt][10pt]{\leafright} \hrulefill
\raisebox{-2.1pt}[10pt][10pt]{~~~\decofourleft \decotwo\decofourright~~~} \hrulefill
\raisebox{-2.1pt}[10pt][10pt]{ \leafleft}}
}

\renewcommand{\footnoterule}{\vspace{-0.8em}
\noindent
 {\color{ornament}\pgfornament[width=5cm]{88} }
{\color{ornament}\includegraphics[width=5cm]{tikz/88} }
\vspace{.5em} }

\fancyfoot[RO,CO,LO,RE,CE,LE]{ }
\fancyhead[RO,CO,LO,RE,CE,LE]{ }

\fancyhead[RO,LE]{\scriptsize\thepage}
\fancyfoot[RE,LO]{{\scriptsize Dein Fußzeilentext -- identisch auf jeder linken Seite}} % XXX Hier änderst du den Fußzeilentext
\fancyfoot[RO,LE]{\scriptsize\thepage}
 \fancyfoot[C]{{\color{ornament}\pgfornament[width=1cm]{69}}}
\fancyfoot[C]{{\color{ornament}\includegraphics[width=1cm]{tikz/69}}}


\fancypagestyle{plain}{%
\fancyfoot[RO,LE]{\scriptsize\thepage}
\fancyhead[RO,LE]{\scriptsize\thepage}
\fancyhead[CO]{}
\fancyfoot[LO,CO]{}
}



\titleformat{\part}
{\hbox{\begin{minipage}{\textwidth}\color{ornament}
% \pgfornament[height=1cm]{63}\hfill%
% \pgfornament[height=1cm]{71}\iftablet\pgfornament[height=1cm]{71}\fi\hfill%
% \pgfornament[height=1cm]{64}\\[1em]
\includegraphics[height=1cm]{tikz/63}\hfill%
\includegraphics[height=1cm]{tikz/71}\iftablet\includegraphics[height=1cm]{tikz/71}\fi\hfill%
\includegraphics[height=1cm]{tikz/64}\\
\end{minipage}}
\centering \LARGE\bfseries}
{\newline \thepart.}{.5em}{}[\vspace{1ex}\hbox{%
\begin{minipage}{\textwidth}\color{ornament}
% \pgfornament[height=1cm,symmetry=h]{63}\hfill%
% \pgfornament[height=1cm,symmetry=v]{71}\iftablet\pgfornament[height=1cm]{71}\fi\hfill%
% \pgfornament[height=1cm,symmetry=h]{64}\\[1em]
\scalebox{1}[-1]{\includegraphics[height=1cm]{tikz/63}}\hfill%
\scalebox{1}[-1]{\includegraphics[height=1cm]{tikz/71}}\iftablet\scalebox{1}[-1]{\includegraphics[height=1cm]{tikz/71}}\fi\hfill%
\scalebox{1}[-1]{\includegraphics[height=1cm]{tikz/64}}\\
\end{minipage}
}]


\titleformat{\chapter}
{\iftablet\vspace{-20mm}\else\vspace{-15mm}\fi\hbox{\begin{minipage}[0pt]{\textwidth}\color{ornament}
% \pgfornament[height=1cm]{63}\hfill%
% \pgfornament[height=1cm]{87}\hfill%
% \pgfornament[height=1cm]{64}\\[1em]
\includegraphics[height=1cm]{tikz/63}\hfill%
\includegraphics[height=1cm]{tikz/87}\hfill%
\includegraphics[height=1cm]{tikz/64}\\
\end{minipage}}\vspace{-.5em}
\LARGE\bfseries}
{\newline \thechapter.}{0.5em}{}%
[\iftablet\vspace{-13mm}\else\vspace{-10mm}\fi]


\titleformat{\section}
{\hbox{\begin{minipage}{\textwidth}\color{ornament}
\hfill\parbox{4cm}{\pgfornament[width=2cm]{84}\pgfornament[width=2cm]{84}}%
\raisebox{2.6pt}{\pgfornament[width=1cm,symmetry=v]{14}}
\end{minipage}}
\Large\bfseries}
{\newline \thesection.}{.5em}{}



\ifcalli
\newcommand{\calli}{\calligra}
\else
\newcommand{\calli}{\itshape}
\fi


\let\oldquote\quote
\let\oldendquote\endquote
\def\quote{\begingroup \oldquote \flushleft }
\def\endquote{\endflushleft \oldendquote \endgroup \bigskip}






\makeatletter

\def\initial{\@ifstar\@initial\@@initial}
\def\@initial#1#2{
\lettrine[lines=2, loversize=0.3, lraise=0]{``\initfamily #1}{#2}
}
\def\@@initial#1#2{
\lettrine[lines=2, loversize=0.3, lraise=0]{\initfamily #1}{#2}
}

\makeatother

\hypersetup{
  bookmarks=true,         % ¿mostrar la barra de marcadores?
  bookmarksnumbered=true,
  linktocpage=true,
  pdfborder={0 0 0},
  unicode=false,          % caracteres no latinos en los marcadores de Acrobat
  pdftoolbar=true,        % ¿mostrar la barra de herramientas de Acrobat?
  pdfmenubar=true,        % ¿mostrar el menú de Acrobat?
  pdffitwindow=false,     % ¿ajustar la ventana a la página al abrir?
  pdfnewwindow=true,      % enlaces en una nueva ventana
  colorlinks=true,        % false: enlaces con bordes; true: enlaces coloreados
  linkcolor=blue,         % color de los enlaces internos
  citecolor=blue,         % color de los enlaces a la bibliografía
  filecolor=blue,         % color de los enlaces a archivos
  urlcolor=blue,          % color de los enlaces externos
  hyperfootnotes=false,
}

\ifpdf
  \pdfinfo{
    /Author (Diego MArtínez Gutiérrez) % XXX Ingresa tu nombre
    /Title  (El Fanal) % XXX esto puedes cambiarlo
    /Subject (Ciencia Ficción) % XXX esto puedes cambiarlo
      }
\fi


\begin{document}

\selectlanguage{spanish}
\pagestyle{empty}
%test\newpage
{\begingroup
%%TITELSEITE%%
\iftablet
\ThisCenterWallPaper{1.2}{Strand.jpg} % XXX aquí puedes insertar una imagen diferente para la vista de tablet, después de haberla subido a la carpeta img.
\else
\ThisLLCornerWallPaper{2.0}{Strand.jpg} % XXX aquí puedes insertar una imagen diferente para la vista en dispositivos que no sean tablets, después de haberla subido a la carpeta img.
\fi
\noindent
\hspace{-8mm}\parbox{12cm}{\rule{10cm}{1.6pt}\vspace*{-\baselineskip}\vspace*{2pt}\\ % Línea horizontal gruesa
\rule{10cm}{0.4pt}\\[0.2\baselineskip] % Línea horizontal fina
\parbox[c][2cm]{10cm}{\centering\textsc{\LARGE \textbf{El Fanal}}\par}\\ % XXX ¡aquí cambias tu título!]

 \noindent\rule{10cm}{0.4pt}\vspace*{-\baselineskip}\vspace{3.2pt}\\ % Línea horizontal fina
\rule{10cm}{1.6pt}}\\[\baselineskip] % Línea horizontal gruesa
\endgroup}
\cleardoublepage
{\begingroup
% INNENTITEL
\centering
\vfill
\parbox{\textwidth}{%
\rule{\textwidth}{1.6pt}\vspace*{-\baselineskip}\vspace*{2pt} % Línea horizontal gruesa
\rule{\textwidth}{0.4pt}\\[0.2\baselineskip] % Línea horizontal fina
\parbox{\textwidth}{%
\parbox[c][1.5cm][c]{1.5cm}{\includegraphics[height=1.5cm]{tikz/19}}
~ \hrulefill ~
\parbox[c][2cm][c]{7cm}{\centering\textsc{\Large \textbf{Mi diario de embarazo}}}
 ~\hrulefill ~
\parbox[c][1.5cm][c]{1.5cm}{\scalebox{-1}[1]{\includegraphics[height=1.5cm]{tikz/19}}}}\\[0.2\baselineskip] % Título
\rule{\textwidth}{0.4pt}\vspace*{-\baselineskip}\vspace{3.2pt} % Línea horizontal fina
\rule{\textwidth}{1.6pt}}\\[\baselineskip] % Línea horizontal gruesa
\scshape % Letras versalitas
Como parte de la crónica familiar \\[0.2\baselineskip] % XXX Puedes eliminar esto

\vspace*{2\baselineskip} % W
escrito por \\[0.5\baselineskip]
{\Large TU NOMBRE}  \\
\Large
TU NOMBRE % XXX Aquí puedes añadir un segundo nombre. Si no lo necesitas, puedes eliminar esta sección desde "\\" hasta después de este texto azul.
\vfill
{Con la colaboración de  \\ {Abuela / Abuelo, etc.}} % XXX Aquí puedes añadir más nombres. Si no quieres añadir ninguno, puedes eliminar esta sección desde "escrito por" hasta después de este texto azul.
\vfill \vfill
{Fotos de  \\ {Tu fotógrafo}\par} % XXX Aquí puedes añadir el nombre de tu fotógrafo o eliminar esta sección desde "{Fotos de" hasta después de este texto azul.

\vfill %

{\scshape 2020} \\[0.3\baselineskip] % XXX Aquí puedes insertar el año de publicación / finalización, o eliminar la línea desde "{\scshape" hasta después de este texto azul.

\endgroup}
\iftablet \newpage \else \cleardoublepage \fi

\pagestyle{fancy}
\pagenumbering{roman}
\tableofcontents \iftablet \newpage \else \cleardoublepage \fi
% XXX Si no deseas tener un índice, puedes eliminar esta línea desde \tableofcontents.

\pagenumbering{arabic}
\makeatletter
\let\Oldpart\part
\def\part{\@ifstar\@DASTpart\@@DASTpart}
\def\@DASTpart#1{
\cleardoublepage
\phantomsection
\bookmarksetup{startatroot}% this is it
\fancyhead[LE,LO]{} %Kapitel
\Oldpart*{#1}
\fancyhead[RE,RO]{}
\fancyhead[LE,LO]{\scriptsize #1}
}
\def\@@DASTpart#1{
\Oldpart{#1}
\fancyhead[LE,LO]{}
\fancyhead[RE,RO]{}
\fancyhead[LE,LO]{ \scriptsize \partname ~ \thepart : ~ #1}
}

\let\Oldchapter\chapter
\def\chapter{\@ifstar\@DASTchapter\@@DASTchapter}
\def\@DASTchapter#1{
\Oldchapter*{#1}
\fancyhead[RE,RO]{ \scriptsize #1}
\addcontentsline{toc}{chapter}{#1}
}
\def\@@DASTchapter#1{
\Oldchapter{#1}
\fancyhead[RE,CO]{ \scriptsize \chaptername ~ \thechapter : ~ #1}
}
\makeatother


% XXX Coloca un % antes de las partes que no quieras que aparezcan en el PDF. Esto hará que toda la línea se vuelva azul. El código ya no será leído.
\input{Inhalt/01- Introducción}
\input{Inhalt/02- Antecedentes}
\input{Inhalt/03- Más antecedentes}
\input{Inhalt/04- 1er mes de embarazo}
\input{Inhalt/05- 2do mes de embarazo}
\input{Inhalt/06- 3er mes de embarazo}
\input{Inhalt/07- 4to mes de embarazo}
\input{Inhalt/08- 5to mes de embarazo}
\input{Inhalt/09- 6to mes de embarazo}
\input{Inhalt/10- 7mo mes de embarazo}
\input{Inhalt/11- 8vo mes de embarazo}
\input{Inhalt/12- 9no mes de embarazo}
\input{Inhalt/13- 10mo mes de embarazo}
\input{Inhalt/14- El parto}
\input{Inhalt/15- El posparto}

\end{document}
