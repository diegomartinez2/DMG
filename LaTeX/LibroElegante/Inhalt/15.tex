\part*{Das Wochenbett}

% % % % % % % % % % % % % % % % % % % % % % % % % % % % % % % % %
% % % % % % % % % % % % % % % % % % % % % % % % % % % % % % % % %
% % % % % % % % % % % % % % % % % % % % % % % % % % % % % % % % %
% % % % % % % % % % % % % % % % % % % % % % % % % % % % % % % % %
% % % % % % % % % % % % % % % % % % % % % % % % % % % % % % % % %
% % % % % % % % % % % % % % % % % % % % % % % % % % % % % % % % %
% % % % % % % % % % % % % % % % % % % % % % % % % % % % % % % % %
% % % % % % % % % % % % % % % % % % % % % % % % % % % % % % % % %
% % % % % % % % % % % % % % % % % % % % % % % % % % % % % % % % %
% % % % % % % % % % % % % % % % % % % % % % % % % % % % % % % % %
% % % % % % % % % % % % % % % % % % % % % % % % % % % % % % % % %
% % % % % % % % % % % % % % % % % % % % % % % % % % % % % %

\chapter*{Erste Tage}
\lettrine[lines=2, loversize=0.3, lraise=0]{\initfamily U}{nsere}

\section*{Stillen}

\section*{Rückbildung}

\section*{Aufnahme in die Familie}

\subsection*{Oma und Opa}

\subsection*{Ältere Geschwister}

\subsubsection*{der ältere bruder}

\subsubsection*{die ältere Schwester}

\subsection*{Nachbarschaft}

\chapter*{Die ersten Wochen}
\lettrine[lines=2, loversize=0.3, lraise=0]{\initfamily B}{ereits } nach

\chapter[La Nueva Frontera Salvaje y la Fiebre Estelar]{Capítulo 15: La Nueva Frontera Salvaje y la Fiebre Estelar}
\lettrine[lines=2, loversize=0.3, lraise=0]{E}{l} caos desatado por el pulso chino no fue eterno, pero sus ecos transformaron el mundo como un río que, al desbordarse, esculpe nuevos valles. Mientras los gobiernos se enzarzaban en una danza frenética por los semiconductores, la humanidad, armada con los planos de código abierto del motor FTL, alzó los ojos al firmamento. El espacio dejó de ser un feudo de reyes y generales; se convirtió en una frontera salvaje, un lienzo donde los sueños y las pesadillas se tejían con hilos de luz estelar.

En garajes polvorientos, en foros digitales que zumbaban como colmenas, en talleres improvisados donde el olor a soldadura se mezclaba con el de la esperanza, la gente corriente forjaba naves con la audacia de quien no tiene nada que perder. No eran las superpotencias ni las corporaciones quienes lideraban esta estampida; eran los soñadores, los locos, los que veían en el vacío un refugio y en las estrellas un destino. La Tierra, con sus guerras y sus cenizas, se desvanecía en el horizonte, y el cosmos, vasto como un océano sin orillas, los llamaba con un canto que solo los valientes podían escuchar.

En un suburbio de São Paulo, bajo un cielo que parecía arder con promesas, un grupo de mecánicos e ingenieros retirados se reunía en torno a una nave hecha de restos de aviones y acero reciclado. Sus manos, curtidas por años de trabajo, trazaban líneas en los planos de Louis Martin, ahora un evangelio digital compartido por millones. João, un anciano con ojos que brillaban como luceros, dirigía el esfuerzo, su voz temblando de fervor. \begin{quote}
\calli
\glqq No necesitamos gobiernos,\grqq~ decía, mientras soldaba un panel bajo la luz titilante de una lámpara. \glqq El espacio es nuestro, un campo donde plantaremos un nuevo comienzo.\grqq
\end{quote}
Habían recaudado fondos en una campaña de crowdfunding que resonaba como un grito colectivo, uniendo a desconocidos desde Mumbai hasta Ciudad del Cabo. Su nave, bautizada \emph{Estrela Livre}, no buscaba gloria, sino un hogar lejos de la guerra y la pobreza, un planeta donde el viento no oliera a humo.

Pero no todos los soñadores eran tan pacientes. En un almacén olvidado de Detroit, un grupo de jóvenes, con más arrojo que cálculo, ensamblaba una nave que parecía un rompecabezas de chatarra. Sus ordenadores, rescatados de vertederos electrónicos, ejecutaban \emph{FluzoCalc} con la fe de quien lanza una moneda al aire. \begin{quote}
\calli
\glqq Si funciona, seremos los primeros en tocar una estrella,\grqq~ decía Maya, una chica de cabello teñido de azul, mientras ajustaba un condensador de Fluzo casero. \glqq Y si no, al menos lo intentamos.\grqq
\end{quote}
No tenían plan, solo fiebre, una sed que los impulsaba al vacío sin mapas ni promesas. El riesgo era su bandera, el fracaso su poema, y el cosmos, su lienzo en blanco.

Más allá de los pioneros, un murmullo más oscuro se alzaba, un coro de disidentes que veían en el FTL no un puente a las estrellas, sino un arca para la salvación o el caos. En un pueblo olvidado de los Andes, una secta conocida como \emph{Los Hijos de Raticulli} se reunía bajo la luz de antorchas, sus cánticos elevándose como humo hacia un cielo puntuado de constelaciones. Creían que el FTL era un regalo divino, un camino para escapar del apocalipsis terrestre y alcanzar a sus salvadores estelares. Su líder, una mujer de rostro afilado y ojos que parecían contener galaxias, dirigía la construcción de una nave que llamaban \emph{Ascensión}. \begin{quote}
\calli
\glqq No exploramos,\grqq~ susurraba, mientras sus seguidores martillaban metal bajo la luna. \glqq Ascendemos.\grqq
\end{quote}
Cada tornillo, cada cable, era un acto de fe, un paso hacia un paraíso que solo ellos podían imaginar.

En las sombras de ciudades fracturadas, otros grupos, los utopistas, tejían sueños distintos. Financiados por millonarios excéntricos que habían amasado fortunas en la era digital, planeaban colonias donde las fronteras se disolverían como niebla. En un loft de Berlín, iluminado por pantallas que proyectaban simulaciones de mundos nuevos, un colectivo de filósofos y programadores diseñaba \emph{Aurora}, una nave para un planeta sin gobiernos, sin clases, donde la humanidad renacería como un solo latido. \begin{quote}
\calli
\glqq No queremos repetir la Tierra,\grqq~ decía su líder, un joven con cicatrices en las manos y esperanza en la voz. \glqq Queremos un lienzo limpio, un lugar donde el hombre sea más que su codicia.\grqq
\end{quote}

Pero no todos los disidentes soñaban con utopías. En los callejones de Yakarta, un grupo que se autoproclamaba \emph{Sombras del Éter} veía en el FTL un arma, no un refugio. Con la capacidad de teletransportar bombas o plagas a cualquier rincón del mundo, planeaban sembrar el caos, derribar los pilares de un sistema que los había marginado. Sus naves, ensambladas en secreto, no buscaban las estrellas, sino el poder de destruir desde ellas. \begin{quote}
\calli
\glqq El mundo arderá,\grqq~ susurraba su líder, un hombre sin rostro que operaba desde la red oscura, \glqq y desde sus cenizas, nosotros decidiremos qué renace.\grqq
\end{quote}

En Hamburgo, Joris de Vries contemplaba \emph{El Sueño de Ícaro}, su nave erguida como un titán en el astillero, pero su corazón pesaba como plomo. La escasez de chips, un eco del pulso chino, había detenido la producción de los nuevos condensadores de Fluzo, y el núcleo robado seguía perdido, un fantasma en un casillero olvidado. Louis Martin, a su lado, trazaba ecuaciones en una servilleta, su mente danzando en un cosmos que no conocía fronteras. \begin{quote}
\calli
\glqq Joris,\grqq~ dijo, con la calma de quien conversa con las estrellas, \glqq el caos es solo una partitura mal tocada. Si ajustamos los campos, los pulsos pueden ser nuestro combustible, no nuestra ruina.\grqq
\end{quote}
Joris lo miró, atrapado entre la admiración y la desesperación. El mundo se fragmentaba, pero Louis veía solo armonías.

En un rincón del astillero, una paloma blanca alzó el vuelo, sus alas cortando el aire como un susurro del destino. Sus ojos, profundos como los de Miguel García Solís, parecían observar la fiebre estelar que consumía a la humanidad, una danza de sueños y pesadillas que se alzaba hacia el infinito. La Tierra no había cambiado; sus anhelos y temores seguían siendo los mismos. Pero ahora, con el FTL en sus manos, los llevaba al espacio, donde la nueva frontera salvaje no era un lugar, sino un reflejo del alma humana, un espejo roto que brillaba bajo las estrellas.
