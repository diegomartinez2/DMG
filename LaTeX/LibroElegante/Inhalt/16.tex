\chapter{La Sombra de Centauri}

A bordo del \textit{Pionero}, el silencio era un manto que envolvía la nave como el abrazo de un dios olvidado. Dos semanas de deriva por la trayectoria Hoffmann, alejándose de la Tierra como un exiliado que huye de su sombra, habían transformado la cápsula Dragon en una prisión flotante, un capullo de acero donde el tiempo se estiraba como un suspiro interminable. Los seis tripulantes —cuatro ingenieros y dos científicos, tejedores de sueños y ecuaciones— se movían con la cautela de quienes saben que el cosmos no perdona errores. El comandante Mark Collins, un ex piloto de la Fuerza Aérea con ojos que parecían haber visto demasiados horizontes, scrutaba las pantallas con un ceño que tallaba surcos en su rostro, como ríos secos en un desierto. Su misión, grabada en su alma como un juramento, era ser el primero en besar la luz de Alfa Centauri.

«Señal visual confirmada», dijo Elena Ramírez, ingeniera jefe, su voz un susurro que cortaba el silencio como una hoja de obsidiana. «Es el \textit{Dragón Ascendente}. Su firma energética es… ruidosa, un lamento que resuena en el vacío.»

En la pantalla, la nave china apareció como un lucero herido, envuelta en un halo de energía residual que parpadeaba como el aliento de un dragón moribundo. El pulso electromagnético que había sumido a la Tierra en tinieblas era el eco de su despegue imprudente, pero su audacia les había dado una ventaja fugaz. Los cálculos para el salto, destilados por el supercluster de la NASA, ya estaban cargados en el \textit{Pionero}. Solo debían esperar a que el \textit{Dragón Ascendente} desplegara su salto, un desafío que colgaba en el espacio como una estrella a punto de colapsar.

Collins, con la mirada fija en un reloj que parecía contar los latidos del universo, apretó los puños. «Tenemos minutos de ventaja», murmuró, su voz un trueno contenido. «No perderemos esta carrera, no cuando las estrellas nos miran.»

De pronto, la pantalla destelló. La firma energética del \textit{Dragón Ascendente} se intensificó, un incendio que devoraba la oscuridad, y en un instante, la nave china se desvaneció, como si el cosmos la hubiera tragado en un suspiro. El silencio en el puente de mando era un abismo, roto solo por el zumbido de los ventiladores, un canto mecánico que parecía burlarse de su espera. La nave china había saltado, un desafío lanzado al infinito.

Desde la Tierra, la voz de la sala de control llegó como un eco lejano: «Procedan con el salto, \textit{Pionero}. Repito, procedan con el salto. El tiempo es un ladrón que no espera.»

Collins no vaciló. «Estamos en el punto», dijo, su voz firme como el acero. «Preparen el motor FTL. Que las estrellas sean testigos.»

Los ingenieros —Elena, con sus manos precisas como las de un cirujano; Jamal Carter, cuyo humor escondía un genio para las máquinas; Li Wei, silencioso pero infalible; y Sarah Nguyen, que leía los datos como un poeta lee el viento— ajustaron el condensador de Fluzo. Los científicos, Clara Voss y Amit Patel, revisaban las ecuaciones con la devoción de monjes ante un texto sagrado. El motor cobró vida, un latido que resonaba en el corazón de la nave.

El mundo se torció. Los ojos de Collins se estiraron, como si el universo los jalara hacia su interior. Las paredes de la cápsula se curvaron, el tiempo se volvió elástico, un río que fluía en todas direcciones. Un murmullo, más sensación que sonido, llenó el aire, un canto que parecía nacer de las entrañas del cosmos, como si las estrellas mismas susurraran secretos olvidados. Los colores se desdibujaron en un espectro que no tenía nombre, un lienzo donde la realidad se deshacía como arena en el viento.

El salto fue un instante, pero también una eternidad.

El \textit{Pionero} emergió en un espacio nuevo, donde el tiempo retomó su ritmo y los colores volvieron a su lugar, como un cuadro que recupera su forma. Frente a ellos, Alfa Centauri, un dúo estelar de fuego blanco y ámbar, brillaba con una luz que no era luz, sino un tapiz de sueños tejidos en el vacío. Habían llegado.

Un grito de júbilo estalló en el puente. Clara abrazó a Amit, sus gafas empañadas por lágrimas de incredulidad. Jamal soltó una risa que resonó como un tambor, mientras Li Wei, siempre reservado, permitió una sonrisa fugaz. Sarah, con los dedos aún sobre el panel, susurró un «lo logramos» que parecía una oración. Collins, por un instante, dejó que la tensión abandonara su rostro, una grieta en su armadura de comandante.

«Escaneen el sistema», ordenó, su voz teñida de una sonrisa que no duraría. «Busquen la firma del \textit{Dragón Ascendente}. Si llegaron, quiero felicitarlos, aunque sea con los dientes apretados.»

Los instrumentos zumbaron, buscando en el vacío el rastro de los rivales. Pero el espacio respondió con un silencio que pesaba como el destino. No había firma energética, ni eco, ni susurro de la nave china. El \textit{Pionero} flotaba solo en el sistema de Alfa Centauri, un náufrago en un mar de estrellas. La victoria, que debería haber sido un incendio de gloria, se sentía hueca, un trofeo forjado en ausencia.

«¿Dónde están?», murmuró Clara, sus ojos recorriendo las pantallas como si pudieran conjurar a los desaparecidos. «El salto… ¿falló? ¿O encontraron una forma de no dejar rastro, de deslizarse como fantasmas?»

Collins, con la mandíbula tensa, no respondió. La pregunta era un peso que colgaba en el aire, más grande que la victoria, más oscura que el espacio. ¿Había errado el \textit{Dragón Ascendente} su destino, perdido en un pliegue del cosmos? ¿O habían dominado un secreto que el \textit{Pionero} aún no alcanzaba? En el horizonte, una paloma blanca, imposible en el vacío, pareció cruzar el cristal de la nave, sus alas un destello que llevaba el eco de Miguel García Solís. Sus ojos, profundos como el abismo, parecían saber la respuesta, pero el cosmos, como siempre, guardaba su silencio.

\end{document}
