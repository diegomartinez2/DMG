

\chapter*{Der siebte Schwangerschaftsmonat}

\lettrine[lines=2, loversize=0.3, lraise=0]{\initfamily U}{nsere}

\chapter{La Escalada Global y la Guerra de Mentes}

\lettrine[lines=2, loversize=0.3, lraise=0]{E}{l} mundo, aún tambaleándose bajo el peso de la explosión de Shenzhen, se precipitó hacia una nueva era, un torbellino de ambición donde las superpotencias, armadas con planos robados y fragmentos de un sueño roto, corrían tras el espejismo del FTL. No era la falta de recursos lo que las frenaba, ni el acero ni los billones que fluían como ríos desbordados. Era la ausencia de un alma, de una chispa que solo un hombre sin rostro, un genio errante en las calles de Luxemburgo, podía encender. La humanidad, con sus ejércitos de ingenieros y sus arsenales de poder, jugaba ahora como un mono con dinamita, ciego ante la chispa que podía incendiar el cosmos.

\section{El Maná del Pentágono}

En los laboratorios de la NASA, donde el aire olía a cables quemados y esperanzas renovadas, los científicos se movían como sacerdotes en un templo profanado. El fiasco de Shenzhen había sido un relámpago que iluminó el camino para Estados Unidos, y de la noche a la mañana, los recortes presupuestarios se desvanecieron como niebla al amanecer. El General Thompson, un hombre cuya voz resonaba como el martillo de un herrero sobre el yunque, lideraba un consorcio de agencias de inteligencia y defensa, blandiendo un cheque en blanco que parecía escrito con la tinta de la urgencia.

\begin{quote}
\calli
\glqq Tenemos la teoría,\grqq~ proclamó Thompson, su confianza un faro en la tormenta, \glqq tenemos los planos robados, y ustedes tienen el conocimiento. Solo nos falta el corazón del motor, esa pieza que el genio de Luxemburgo llamó ‘condensador de fluzo’. No me importa si es un chiste de una película ochentera; lo que me importa es que no sabemos cómo funciona.\grqq
\end{quote}

Los científicos, con rostros pálidos y ojeras que contaban noches de ecuaciones sin respuesta, intercambiaron miradas cargadas de impotencia. Habían intentado descifrar el artículo de Louis Martin, un laberinto de geometría de Riemann y haces fibrados que se burlaba de sus mentes como un jeroglífico cósmico. \glqq General,\grqq~ dijo un ingeniero, el mismo que había reconocido los planos robados, con un tono que temblaba entre la frustración y la verdad, \glqq no podemos hacerlo solos. Necesitamos a Louis Martin. Él es la clave, el alma que da vida a este rompecabezas.\grqq

Thompson frunció el ceño, su rostro una máscara de granito resquebrajado. \glqq No lo tendremos. Está en Europa, protegido por la diplomacia como un santo en su altar. La Unión Europea lo usa como moneda de cambio, un trofeo que no cederán. Pero si construimos una nave sin él, seremos los primeros en surcar las estrellas. Y eso, señores, es la única victoria que importa.\grqq

\section{El Silencio del Dragón}

En China, bajo las montañas de la región autónoma del Tíbet, donde el viento susurraba secretos antiguos entre picos nevados, un búnker secreto albergaba el eco del fracaso de Shenzhen. El general del MSS, con el rostro endurecido por la presión que pesaba como un yugo de hierro, había trasladado su equipo a este refugio oculto, donde los errores no iluminarían el cielo ni alertarían al mundo. La explosión, borrada de la historia oficial con la precisión de un cirujano, era una cicatriz que ardía en su memoria.

Los científicos chinos, con la tenacidad de hormigas que construyen un imperio, adoptaron un enfoque brutal: construir cientos de prototipos, alterar una variable a la vez, observar los resultados como alquimistas ciegos buscando oro. No intentaban comprender la teoría de Louis; la aplicaban como un martillo, golpeando el yunque de la física hasta que cediera. Pero el “condensador de fluzo” seguía siendo una caja negra, un misterio sellado que desafiaba sus esfuerzos. Cada prototipo era un cuerpo sin alma, un cascarón que rugía pero no cantaba. Y la lección de Shenzhen pesaba como un presagio: un error podía convertir una mota de polvo en un dios de destrucción.

El verdadero obstáculo no era la falta de recursos, ni de acero ni de manos expertas. Era la ausencia de un genio, de un hombre que se movía por el mundo sin pasaporte, un espectro que danzaba entre las sombras de Europa.

\section{La Guerra de Mentes}

En el astillero alemán, donde las grúas se alzaban como titanes que custodiaban un secreto estelar, Joris de Vries trabajaba en la penumbra, su mente un campo de batalla entre la esperanza y el temor. Los videos de Anonymous, que aún circulaban en la red como fantasmas digitales, le habían mostrado los fracasos de las superpotencias, pero también la chispa de posibilidad que Louis había vislumbrado. Mientras las naciones construían naves con acero y ambición, carecían del corazón que solo Louis podía forjar. El “condensador de fluzo” no era una pieza, ni un chiste; era la partitura de un genio, un canto que solo él podía entonar.

Louis, ajeno al clamor del mundo, trabajaba en su laboratorio de Luxemburgo, rodeado de pantallas que parpadeaban como constelaciones atrapadas. Sus ecuaciones, garabateadas en servilletas y cuadernos, parecían danzar con una vida propia, como si el universo susurrara sus secretos en su oído. No le importaba la guerra de las superpotencias, ni las filtraciones de Anonymous, ni las amenazas que pesaban sobre Joris. Su mente estaba en las estrellas, en el latido del cosmos que solo él podía escuchar.

El verdadero desafío no era construir naves, ni sensores, ni inteligencia artificial. Las superpotencias podían forjar cuerpos de metal, pero no podían darles alma. La guerra se había transformado en una danza de mentes, un duelo donde el vencedor sería quien descifrara el enigma de Louis Martin. Pero Louis, con su calma etérea, era un adversario que no jugaba. Su fuerza no radicaba en la estrategia, sino en su indiferencia al poder, en su devoción a un sueño que trascendía fronteras y ejércitos.

En las calles de Luxemburgo, mientras Louis caminaba hacia el laboratorio, un agente del MSS intentó seguirlo, su cámara oculta bajo un abrigo. Pero una ráfaga de viento, levantada por un grupo de niños que corrían tras un balón, alzó una nube de polvo que cegó el lente. Cuando el aire se aclaró, Louis había desaparecido, y en su lugar, una paloma blanca, con alas que brillaban como un reflejo de la luna, alzó el vuelo. Sus ojos, profundos como los de Miguel García Solís, parecían observar el mundo con una sabiduría antigua, un eco del pasado que protegía el futuro.
