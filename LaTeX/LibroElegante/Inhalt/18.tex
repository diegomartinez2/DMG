\chapter[La Paradoja de Centauri]{La Paradoja de Centauri}

\lettrine[lines=2, loversize=0.3, lraise=0]{E}{n} el puente de mando de las naves, suspendidas a cuatro años luz de la Tierra, el silencio era un lienzo donde el cosmos pintaba su indiferencia. El \textit{Pionero} y el \textit{Dragón Ascendente}, como náufragos en el océano estelar de Alfa Centauri, flotaban en un mutismo que no era temor, sino asombro, un eco de corazones que latían al ritmo de las estrellas. Habían desafiado la luz, pero el universo, con su geometría caprichosa, los separaba por horas luz en un sistema triple donde las órbitas danzaban como hilos en un telar roto. Buscándose sin hallarse, eran fantasmas en un espejo fracturado, atrapados en la paradoja del tiempo.

\section{El Pionero: El Mapa de las Estrellas}

A bordo del \textit{Pionero}, el comandante Mark Collins, con ojos que parecían haber robado el fulgor de las constelaciones, contemplaba Alfa Centauri A, un sol blanco que ardía como una promesa susurrada. La cápsula Dragon, su refugio en el vacío, vibraba con el canto mecánico de los instrumentos, un coro que acompañaba a los seis tripulantes: los ingenieros Elena Ramírez, Jamal Carter, Li Wei, Sarah Nguyen, y los científicos Clara Voss y Amit Patel, cuyas almas temblaban con la fiebre de lo imposible.

«Hemos cruzado cuatro años luz», murmuró Collins, su voz un río que se quebraba de emoción. «Estamos aquí.»

Clara Voss, con la precisión de quien lee las estrellas como un manuscrito antiguo, ajustó las cámaras de alta resolución.

«Centauri A», dijo, su tono un puente entre la ciencia y la maravilla. «Las firmas espectrales son un poema que confirma nuestro lugar.»

Las pantallas se llenaron de imágenes: asteroides que trazaban estelas como pinceladas en un lienzo negro, planetas improbables en órbitas que desafiaban la lógica de un sistema triple, un caos estelar donde la estabilidad era un sueño fugaz. Su misión era mapear este rincón del cosmos, tejer un tapiz de pruebas que la Tierra, a cuatro años de distancia, algún día descifraría.

Un pitido agudo, como el lamento de un pájaro perdido, rompió el silencio. Elena Ramírez, con dedos que parecían danzar sobre ecuaciones invisibles, señaló la pantalla.

«Firma energética residual», dijo, «con un retraso de cuatro horas y media, el tiempo que la luz tarda en cruzar este abismo.»

Los chinos habían llegado, no a Centauri A, sino a otro rincón del sistema, separados por la paradoja relativista que convertía el espacio en un río de tiempo torcido. Jamal, con una risa que ocultaba su alivio, murmuró:

«Vivos, pero lejos. El cosmos juega a las escondidas.»

Collins, con el ceño tallado por la tensión, asintió.

«Mapeen todo: asteroides, planetas, lo que sea. Necesitamos pruebas, aunque nuestras voces tarden cuatro años en cantarlas.»

La comunicación con la Tierra era un eco imposible, un mensaje atrapado en el lento viaje de la luz. Y el regreso, un desafío aún más grande: sin supercomputadoras, debían calcular múltiples saltos de un año luz, horas de espera entre cada uno, confiando en chips blindados pero lentos, tortugas de silicio que gateaban en un desierto estelar.

\section{El Dragón Ascendente: El Peso del Silicio Herido}

En el \textit{Dragón Ascendente}, el capitán Zhang Wei, cuyo nombre evocaba la fuerza de montañas eternas, miraba por la ventana hacia Alfa Centauri B, un sol ámbar que pintaba el vacío con un resplandor melancólico. Su tripulación —seis almas forjadas en la disciplina de Pekín: los ingenieros Liu Chen y Mei Ling, el navegante Wang Jun, el científico Xu Tao, la oficial de comunicaciones Lin Hua, y el técnico Zhao Peng— trabajaba con la urgencia de quienes llevan el peso de una nación. Pero el silencio en el puente era un yugo, no una promesa.

Wang Jun, con el rostro pálido como la luna sobre el Tíbet, revisó los cálculos en una pantalla que titilaba con datos fracturados.

«Capitán», dijo, su voz un susurro que temblaba como una hoja en el viento, «hemos llegado, pero no al lugar correcto. Esto es Centauri B, no A. El modelo del indio… era demasiado simple. El campo de Riemann se curva aquí, como un río que se retuerce.»

Zhang Wei frunció el ceño, su mirada un filo que cortaba el aire.

«Los cálculos eran precisos», gruñó, aunque una sombra de duda cruzó su alma.

No era el modelo; era el silicio. El pulso electromagnético que había oscurecido la Tierra había rozado su nave, dejando cicatrices invisibles en los chips. Diseñados para resistir la radiación cósmica, eran caros y lentos, y las partículas de alta energía de las detonaciones nucleares habían sembrado errores, como semillas de caos en un campo estéril. Cada cálculo era una apuesta, cada salto un salto de fe.

Lin Hua, con ojos que capturaban el resplandor ámbar de Centauri B, interrumpió.

«Capitán, hemos detectado una nave. El \textit{Pionero}, en Centauri A, a horas luz de nosotros. Su señal llega como un eco, retrasada por el tiempo que la luz tarda en cruzar el abismo.»

La pantalla mostró un destello débil, un faro lejano en la vastedad.

Wang Jun, con dedos temblorosos, se lanzó a recalcular.

«Puedo ajustar el salto», dijo, más para sí mismo que para el capitán.

Pero Zhang Wei sabía la verdad: los chips, heridos por el pulso, eran un enemigo silencioso. No lo dijo. El honor de China, un peso más grande que las estrellas, lo obligaba a callar.

\section{La Tierra: El Susurro de la Espera}

En la Tierra, las pantallas se habían apagado, pero el fervor seguía ardiendo. Las imágenes del salto, repetidas como un evangelio, unían a la humanidad en un sueño compartido. Pero la espera por el regreso de las naves era un vacío que devoraba la esperanza. Cuatro años luz separaban sus voces; ningún mensaje llegaría hasta que las estrellas lo permitieran.

En Hamburgo, Joris de Vries, en el astillero donde \textit{El Sueño de Ícaro} aguardaba como un titán sin latido, miraba el cielo con ojos cansados. La patente de \textit{El Fanal}, un faro que había iluminado el mundo, no llenaba el hueco del núcleo robado, un secreto perdido en un casillero que nadie encontraba.

Louis Martin, a su lado, trazaba ecuaciones en un cuaderno, su mente un faro en la tormenta.

«Joris», dijo, con la calma de quien conversa con el cosmos, «el campo de Riemann en un sistema triple es un rompecabezas. Si los saltos fallan, no es solo el modelo; es el silicio, tocado por el pulso o por la radiación estelar. Pero \textit{El Fanal} puede estabilizarlo. Podemos calcular saltos precisos, incluso aquí.»

Sus ojos brillaban con una idea que danzaba con las estrellas.

«Si el núcleo regresa, construiremos un puente al cielo.»

Joris, atrapado entre la esperanza y el peso de su nuevo poder, no respondió. El mundo celebraba la abundancia, pero él sentía el vacío de un futuro incierto. En el horizonte, una paloma blanca, con los ojos profundos de Miguel García Solís, alzó el vuelo desde el astillero, sus alas un susurro que parecía llevar el eco de Centauri.

Las naves, separadas por el tiempo y el espacio, mapeaban un sistema que no les pertenecía, mientras la Tierra aguardaba, ciega a su destino, en un silencio que era más pregunta que respuesta.
