\chapter{La Nueva Era de la Abundancia}
\addcontentsline{toc}{chapter}{Capítulo 17: La Nueva Era de la Abundancia}

\lettrine[lines=2, loversize=0.3, lraise=0]{E}{l} mundo, atrapado en un frenesí estelar, contenía el aliento mientras las pantallas destellaban con las imágenes de la \textit{Pionero} y el \textit{Dragón Ascendente} desvaneciéndose en el vacío, un acto de magia que hacía palidecer el alunizaje de antaño. Las calles de São Paulo, Yakarta, Berlín y Nairobi estallaron en un carnaval de luces y gritos, como si la humanidad hubiera tocado las estrellas con las yemas de los dedos. En los bares, los televisores mostraban repeticiones del salto FTL, un relámpago que partía la realidad en dos. Viejos con lágrimas en los ojos alzaban vasos de cerveza, niños corrían con modelos de naves hechos de cartón, y en los templos, los cánticos se elevaban como ofrendas al cosmos. El espacio ya no era un sueño lejano; era un patio donde los audaces jugaban, un lienzo donde la humanidad pintaba su futuro.

Pero la euforia, como un cometa fugaz, tenía su sombra. La \textit{Pionero} había emergido en Alfa Centauri, sola, sin rastro del \textit{Dragón Ascendente}, cuyo silencio pesaba como un presagio. ¿Habían errado su salto, perdidos en un pliegue del cosmos? ¿O acechaban, invisibles, con un secreto que el mundo aún no descifraba? En la Tierra, los gobiernos, enredados en la danza de la geopolítica tras el pulso chino, apenas notaban la ausencia, demasiado ocupados reconstruyendo cadenas de suministro rotas y tejiendo alianzas frágiles. Fue en este torbellino, bajo un cielo que aún vibraba con el eco de las estrellas, donde Stellarius Lux lanzó su golpe maestro, un acto que transformaría el mundo como un río que, al romper sus diques, redibuja el paisaje.

En un auditorio de Bruselas, donde la luz se filtraba por vitrales como un susurro de épocas pasadas, Joris de Vries, flanqueado por Louis Martin y un ejército de abogados que parecían tejer hechizos con sus plumas, anunció la patente internacional de \textit{El Fanal}. No era solo un motor FTL; era un conversor de materia en energía, un milagro que destilaba la esencia del universo. «Hemos descifrado los errores de las superpotencias», dijo Joris, su voz firme como el acero pero teñida de una tristeza que solo Louis entendía. «Un gramo de materia, comprimido hasta los hilos de los quarks, puede iluminar una ciudad por un año. Este es el fin de la escasez, el amanecer de la abundancia.»

La patente, un tapiz burocrático tejido con la astucia de los abogados de Stellarius Lux, era un escudo impenetrable. Mientras los gobiernos se ahogaban en la crisis del EMP y la fiebre estelar, Joris había asegurado que ningún país, ninguna corporación, pudiera replicar \textit{El Fanal} sin su bendición. Era un jaque mate silencioso, un faro que brillaba en la tormenta mientras el mundo aún miraba al cielo, esperando el regreso de sus naves.

La noticia se propagó como un incendio en un bosque seco. En las bolsas de valores, las acciones de las petroleras y mineras se desplomaron como torres de naipes, mientras las de Stellarius Lux se alzaban como cohetes, un cometa financiero que redefinía el poder. Google, Amazon, Nvidia, esos titanes del silicio, enviaban emisarios con maletines llenos de promesas, suplicando licencias para un futuro que ya no controlaban. En las calles, la gente danzaba bajo farolas que pronto podrían brillar con la energía de un puñado de arena. Las guerras por el petróleo, el agua, la tierra, se desvanecían como espejismos, reemplazadas por visiones de un mundo sin hambre, sin fronteras, donde la abundancia era un río que fluía para todos.

En un mercado de Delhi, donde el aroma de especias se mezclaba con el humo de los generadores, un anciano vendía modelos de \textit{El Fanal} hechos de madera, mientras niños gritaban que el futuro había llegado. En Nairobi, una madre joven, con un bebé en brazos, miraba un televisor que repetía el anuncio de Joris, y sus ojos brillaban con una esperanza que no necesitaba palabras. En Yakarta, los \textit{Sombras del Éter}, aquellos que soñaban con el caos, pausaron sus planes, desconcertados por un mundo donde la escasez, su arma, se desmoronaba.

Pero en Hamburgo, en el astillero donde \textit{El Sueño de Ícaro} aguardaba como un titán dormido, Joris no celebraba. La patente era un triunfo, pero el núcleo robado, aquel corazón de metal lunar, seguía perdido en un casillero olvidado, un secreto que lo atormentaba como un espectro. Louis, sentado en una silla junto a un ventanal, garabateaba ecuaciones en un cuaderno, su rostro iluminado por una calma que parecía desafiar el caos. «Joris», dijo, sin alzar la vista, «la energía no solo ilumina ciudades. Puede protegerlas. Si ajustamos \textit{El Fanal}, podríamos crear escudos, barreras que detengan incluso un pulso como el chino. ¿No lo ves? El cosmos nos dio la llave, pero nosotros elegimos la puerta.»

Joris lo miró, atrapado entre la admiración y el peso de su nuevo poder. Stellarius Lux, una empresa nacida en las sombras, ahora era el eje del mundo, una fuerza que los gobiernos, ocupados en sus rencillas, no podían contener. Pero el poder traía preguntas, susurros que resonaban en su alma como el eco de las estrellas. ¿Usaría \textit{El Fanal} para el bien, para iluminar un mundo fracturado? ¿O se convertiría, sin quererlo, en el nuevo titán que los demás temerían? En el ventanal, una paloma blanca, con ojos profundos como los de Miguel García Solís, ladeó la cabeza, como si supiera la respuesta. El mundo celebraba la abundancia, pero Joris sentía el peso de un futuro que aún no podía descifrar, un horizonte donde las estrellas eran tan promesa como amenaza.

\end{document}
