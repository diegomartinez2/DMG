

\chapter*{Der achte Schwangerschaftsmonat}

\lettrine[lines=2, loversize=0.3, lraise=0]{\initfamily U}{nsere}

\chapter{La Partitura Completa}
\label{cap:partitura-completa}

En la penumbra de una oficina de Luxemburgo, donde el aire olía a papel antiguo y promesas frágiles, Joris de Vries abrió un sobre grueso, su superficie adornada con el relieve plateado del logo de la Agencia Espacial Europea. Sus manos, temblorosas como hojas en un viento otoñal, extrajeron una carta escrita con la precisión de un juramento diplomático. Las palabras del Director General, firmadas con tinta que parecía sangrar autoridad, eran un faro en la tormenta que lo rodeaba.

\begin{quote}
    \textbf{Estimado Sr. de Vries:}
    \vspace{0.5em}

    Me dirijo a usted en referencia a las oportunidades de cooperación bajo el marco de nuestra iniciativa \emph{Eurociencia 2030}. Como usted sabe, esta iniciativa está diseñada para apoyar proyectos que prometan avances significativos en el sector energético y de transporte de la Unión Europea. El satélite de prueba que Stellarius Lux puso en órbita hace tres años ha demostrado un potencial que nuestra agencia considera prometedor.
    \vspace{0.5em}

    Nos gustaría invitarle a usted y a su equipo de diseño a una reunión en la sede del Centro Europeo de Tecnología y Ciencias (ESTEC) en los Países Bajos. El objetivo es explorar la posibilidad de concederle un subsidio que, si bien no es de naturaleza militar, le permitiría continuar su trabajo de forma independiente y sin las presiones que, lamentablemente, está experimentando. Este subsidio se formalizará como una ayuda para el desarrollo de un “impulsor no inercial” y un “sistema de conversión de energía sostenible”, de los cuales su trabajo, según nuestra evaluación, muestra un potencial sin precedentes.
    \vspace{0.5em}

    Esperamos su respuesta. La agenda de la reunión está adjunta.
    \vspace{0.5em}

    Atentamente, \quad
    El Director General de la ESA.
\end{quote}

Joris dejó caer la carta sobre el escritorio, una sonrisa asomando en su rostro como un amanecer tímido. Europa había movido su ficha, un gambito envuelto en la sutileza de la diplomacia. No era solo una oferta de fondos; era un escudo contra las garras de las superpotencias, un pacto que disfrazaba su ambición estelar como un proyecto de energía renovable. Joris no dudó: aceptó la invitación, sabiendo que el juego apenas comenzaba.

\section{El Escenario del ESTEC}
\label{sec:escenario-estec}

En el vestíbulo del ESTEC, en Noordwijk, donde maquetas de cohetes y satélites se alzaban como reliquias de un futuro soñado, Louis Martin se movía con la alegría de un niño en un bosque encantado. Sus ojos, brillantes como luceros, recorrían las vitrinas, deteniéndose en cada curva de metal y cada destello de vidrio. Joris, en cambio, caminaba con el peso de un general antes de la batalla, sus sentidos alerta ante los trajes impecables de los burócratas de la ESA, que parecían tejidos con hilos de autoridad y cautela.

La sala de conferencias era un santuario futurista, con una mesa de cristal que reflejaba la luz como un lago helado. Allí, un trío de científicos aguardaba: Duncan, un ingeniero escocés de cabello rojo como un incendio y barba de tres días; un finlandés alto y silencioso, cuya boca apenas se movía, como si las palabras fueran un recurso escaso; y Giancarlo, un italiano que reía con estruendo y hablaba con las manos, como si dirigiera una orquesta invisible. Al frente, Kalle, el director estonio, un hombre de rostro pétreo y mirada afilada, como un halcón que observa desde un risco.

“Señor de Vries, bienvenido,” dijo Kalle, su inglés cortado con un acento del este que resonaba como el eco de un bosque nevado. “Vamos al grano. La Unión Europea ofrece a Stellarius Lux una subvención de quinientos millones de euros. A cambio, el prototipo del ‘impulsor’ debe desarrollarse en suelo europeo. Y necesitamos los planos y la teoría, para certificar su seguridad.”

Joris se sentó, su sonrisa un escudo que ocultaba su estrategia. “Debo rechazar su oferta.”

El rostro de Kalle se endureció, como si la piedra misma se hubiera congelado. Giancarlo soltó una risita, mientras Duncan palidecía, su rostro una máscara de incredulidad. “¿Por qué, \emph{kuradi vittu}?” espetó Kalle, su voz un trueno contenido. “Quinientos millones es una fortuna, una oportunidad que no volverá.”

Joris se inclinó, sus ojos fijos en el estonio. “No lo dudo, Kalle. Pero no entienden. No necesito quinientos millones; necesito dos mil. Estados Unidos y China han movido sus piezas, y no se detendrán. Necesito fondos para construir la nave, contratar a los mejores ingenieros, proteger a mi equipo. Si \emph{El Sueño de Ícaro} despega, el impacto económico será un terremoto, y las superpotencias harán lo imposible por detenernos. Quinientos millones no bastan para enfrentar a dos titanes.”

Kalle soltó una risa seca, como el crujir de ramas en un invierno olvidado. “Tienes cara de \emph{palo}, Joris. Lo has calculado todo.” Le dio una palmada en la espalda, un gesto que resonó como un pacto sellado. “Está bien. Dos mil millones. Es una locura, pero por la patria y la ciencia, lo haremos.”

\section{La Danza de las Ideas}
\label{sec:danza-ideas}

Mientras los burócratas de la ESA se retiraban a deliberar, Louis, Joris y los científicos se quedaron en la sala, un oasis donde las ideas fluían como ríos desbordados. Duncan, el escocés, se sentó junto a Louis, su rostro una mezcla de admiración y temor, como un marinero ante una tormenta. “Escucha, \emph{bampot},” dijo, su acento áspero como el whisky, “he visto los planos. ¿Cómo funciona? ¿Es como el motor de curvatura de \emph{Star Trek}?”

Louis negó con la cabeza, una sonrisa suave curvando sus labios. “No. La curvatura espacial es un truco de principiantes, un atajo para moverse en el espacio. Nosotros no nos movemos; creamos un túnel, un puente entre el espacio y el tiempo. Y podemos controlarlo, como un escultor que moldea la arcilla del cosmos.”

Giancarlo, con sus manos danzando como si dirigiera una sinfonía, se unió con entusiasmo. “¡Es como tejer una tela de araña, donde el espacio y el tiempo se entrelazan! \emph{Magnifico!}”

“Exacto,” asintió Louis, sus ojos brillando como si vieran el universo mismo. “Pero si no lo haces bien, la tela se desgarra, y los hilos de la realidad se rompen. Los rusos, los americanos, los chinos\dots no son tontos. Tienen el instrumento, pero no la partitura. Sin la música, solo hay caos.”

Duncan, con el ceño fruncido, insistió: “¿Y los rusos? ¿Por qué su haz se desvió?”

“No se desvió,” corrigió Louis, su voz un susurro que parecía contener el peso de las estrellas. “El haz se desenfocó, un error mínimo. Creó un túnel, pero emergió a miles de kilómetros. Si lo perfeccionamos, podríamos teletransportar un coche, un tren, un avión, conectar ciudades como hilos en un tapiz. Y la conversión de materia en energía\dots” Hizo una pausa, sus manos dibujando en el aire. “Si enfocamos bien, podemos alimentar la nave, o el mundo. Es el fin de los combustibles fósiles, el amanecer de una nueva era.”

Kalle, que había escuchado en silencio, intervino, su voz cortante como el viento del Báltico. “Es elegante, pero peligroso. Si Estados Unidos o China descubren que tienen la tecnología, \emph{tõbras}, harán cualquier cosa: sabotaje, sanciones, espionaje. Es una carrera armamentista, pero con el alma del mundo en juego.”

Duncan asintió, su rostro sombrío. “Es como en \emph{Blade Runner}: una tecnología que crea vida, pero que también puede destruirla.”

Louis, imperturbable, respondió: “La física es más fuerte que la política. Siempre gana.” Joris lo miró, su corazón dividido entre la admiración y la ansiedad. Louis era un soñador, un poeta del cosmos; Joris, un ingeniero, sabía que los sueños necesitaban un plan, y ningún plan podía detener los ejércitos que los acechaban.

\section{El Sueño de Ícaro y el Robo}
\label{sec:sueno-icaro-robo}

El acuerdo con la ESA se firmó bajo la luz tenue de un amanecer europeo, un pacto sellado con la promesa de dos mil millones de euros disfrazados de “inversión en sostenibilidad”. En un astillero de Hamburgo, donde las grúas danzaban como titanes y el aroma salado del Báltico impregnaba el aire, comenzó la construcción de \emph{El Sueño de Ícaro}. La antigua base naval, un laberinto de acero y secretos, era el refugio perfecto: vasto, privado, con el mar como testigo. Durante dos años, las manos de los ingenieros moldearon el fuselaje, un coloso que parecía soñar con las estrellas. Cuando el núcleo del motor FTL, una esfera de un metal desconocido que brillaba como un fragmento de luna, fue colocado en su corazón, el proyecto alcanzó su \emph{crescendo}.

Pero entonces, en la penumbra de una noche sin luna, un grupo que se hacía llamar \emph{Los Custodios de la Tierra} irrumpió en el astillero. No eran idealistas, ni fanáticos; eran sombras al servicio de un propósito oculto. El ataque fue un relámpago: un incendio rugió, una explosión estremeció el suelo, y el caos se alzó como un dragón despertado. No buscaban destruir la nave, sino robar su alma. Cuando el humo se disipó y las sirenas callaron, \emph{El Sueño de Ícaro} seguía en pie, intacto en su cuna de acero. Pero el núcleo, el corazón palpitante del FTL, había desaparecido, arrancado como un tesoro de un mito antiguo.

Joris, de pie entre los escombros, sintió el peso del mundo aplastarlo. Louis, a su lado, miró el espacio vacío donde el núcleo había brillado, y una paloma blanca, posada en una viga retorcida, ladeó la cabeza. Sus ojos, profundos como los de Miguel García Solís, parecían susurrar una verdad: el robo no era el fin, sino un compás más en la partitura del cosmos.
