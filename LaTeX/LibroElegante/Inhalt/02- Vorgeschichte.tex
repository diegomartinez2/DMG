\part*{Die Vorgeschichte}

% % % % % % % % % % % % % % % % % % % % % % % % % % % % % % % % %
% % % % % % % % % % % % % % % % % % % % % % % % % % % % % % % % %
% % % % % % % % % % % % % % % % % % % % % % % % % % % % % % % % %
% % % % % % % % % % % % % % % % % % % % % % % % % % % % % % % % %
% % % % % % % % % % % % % % % % % % % % % % % % % % % % % % % % %
% % % % % % % % % % % % % % % % % % % % % % % % % % % % % % % % %
% % % % % % % % % % % % % % % % % % % % % % % % % % % % % % % % %
% % % % % % % % % % % % % % % % % % % % % % % % % % % % % % % % %
% % % % % % % % % % % % % % % % % % % % % % % % % % % % % % % % %
% % % % % % % % % % % % % % % % % % % % % % % % % % % % % % % % %
% % % % % % % % % % % % % % % % % % % % % % % % % % % % % % % % %
% % % % % % % % % % % % % % % % % % % % % % % % % % % % % %

\chapter*{Wie wir beschlossen, ein Baby zu bekommen}

\lettrine[lines=2, loversize=0.3, lraise=0]{\initfamily U}{nsere}

\section*{Teil 1}
\lettrine[lines=2, loversize=0.3, lraise=0]{\initfamily U}{nsere}
\section*{Teil 2}

Und noch ein zweiter Teil. Mit Buchstabe. \lettrine[lines=2, loversize=0.3, lraise=0]{\initfamily U}{nsere}

\chapter{El Fanal Brilla en Urano}

\lettrine[lines=2, loversize=0.3, lraise=0]{E}{n} la sala de crisis de la Casa Blanca, el aire estaba cargado de una furia contenida, como si las paredes mismas exhalasen el peso de un agravio cósmico. El presidente de los Estados Unidos, con el rostro endurecido por la incredulidad, golpeó un informe contra la mesa de roble, un sonido seco que resonó como el crujir de un trueno lejano.

\begin{quote}
\calli
\glqq ¿Me dicen que este\ldots \emph{Fanal}, un artilugio de una empresa europea insignificante, llegó a Urano en un suspiro de cero coma dieciséis segundos?\grqq~ Su voz no era un grito, sino un murmullo afilado, más peligroso que cualquier alarido. \glqq ¿Y que la NASA, la ESA, e incluso los chinos han confirmado los datos? ¡Hemos derramado océanos de dinero en propulsión durante décadas, y ni un maldito telegrama nos avisa de esto!\grqq
\end{quote}

El asesor de seguridad nacional, un hombre de gafas gruesas y nervios crispados, ajustó sus lentes como si intentara alinear las estrellas mismas. \glqq Señor, los datos son incontrovertibles. Las mediciones de desplazamiento espacio-temporal coinciden en las bases de todas las agencias. El satélite, registrado en Luxemburgo, se desvaneció de nuestro radar subluminal al encenderse y reapareció en la órbita de Urano, en el tiempo y lugar exactos predichos. El salto es real.\grqq

El presidente se dejó caer en su silla, como un titán abatido por el peso de su propia corona. \glqq Es una humillación. Una afrenta de proporciones estelares. Y para colmo, leo que parte del dinero vino de la Unión Europea, de un tal `Eurociencia 2030'. ¿Dónde estaban nuestros espías? ¿Qué hacíamos nosotros?\grqq~ El almirante de la Marina, mudo hasta entonces, se removió inquieto, su rostro teñido de una vergüenza que parecía manar de las profundidades del océano.

Mientras tanto, al otro lado del mundo, en un búnker subterráneo en las afueras de Pekín, la escena era un espejo de furia, aunque envuelta en una calma gélida. Un oficial del Ministerio de Seguridad del Estado, con la precisión de un cirujano, presentó una pantalla de datos a su superior.

\begin{quote}
\calli
\glqq La tecnología es real. El motor FTL funciona. Nuestro satélite de reconocimiento captó el evento, aunque no pudimos seguir la trayectoria. La firma energética es distinta a todo lo que conocemos.\grqq
\end{quote}

El general, con una calma que parecía tallada en jade, asintió. \glqq No me inquieta la tecnología; ya la descifraremos. Me inquieta su origen. ¿Quién la creó? ¿Quién es el hombre detrás de esto?\grqq~ El oficial tecleó con dedos rápidos, y el rostro de Joris de Vries, el CEO holandés de Stellarius Lux, apareció en la pantalla, pálido como un amanecer invernal. \glqq Es el rostro público. Pero el cerebro, el artífice del motor, solo lo conocemos por un nombre: Louis. Ningún agente en Europa ha podido identificarlo.\grqq

El general frunció el ceño, su mirada cortante como una espada samurái. \glqq ¿Quién es este Louis? Quiero su rostro, su historia, su sombra. ¡Ahora!\grqq

\section{La Caza del Fantasma}

Lo que la CIA y el MSS ignoraban, y lo que Louis Martin había logrado sin proponérselo, era convertirse en un espectro en un mundo de ojos vigilantes. No era un maestro del subterfugio ni un espía adiestrado; era simplemente un hombre tocado por la casualidad, un alma que el destino había envuelto en un manto de invisibilidad.

En Langley, un analista de la CIA, con los ojos enrojecidos por noches sin dormir, se mesaba el cabello frente a un informe. \glqq No hay fotos. Ni una. Ni un selfie en redes, ni un rastro en blogs de viajeros. Es como si no existiera.\grqq~ Había intentado una búsqueda facial inversa con la imagen borrosa de Louis en la videoconferencia con el banco, pero la pantalla de Joris, pixelada como un mosaico antiguo, frustraba cualquier esfuerzo.

En Berlín, un equipo del MSS enfrentaba el mismo enigma. Habían rastreado a Joris de Vries, fotografiándolo al salir del edificio de Stellarius Lux. En la imagen, Louis caminaba detrás, pero en el instante preciso del disparo, una paloma descomunal, como enviada por un dios bromista, cruzó volando, su ala cubriendo el rostro del científico. No había otra toma. El agente juró que era la paloma más grande de su vida, pero sus colegas solo rieron, incrédulos.

Los intentos de los espías se volvieron una comedia de desatinos:

\begin{itemize}
    \item Una cámara de vigilancia en un café vienés, donde Louis y Joris discutían negocios, se empañó con el vaho de un café recién servido, como si el aliento de la bebida conspirara con el destino.
    \item Un agente estadounidense en un aeropuerto intentó escanear a Louis desde lejos, pero un turista, absorto en un selfie, bloqueó la toma, y la cámara solo captó el destello del sol en su teléfono.
    \item La CIA hurgó en los registros de licencias de conducir de Luxemburgo, pero Louis no conducía; pedaleaba en bicicleta o se perdía en el transporte público, sin dejar huella.
    \item El MSS, en un último esfuerzo, accedió a los datos de inmigración, pero el nombre \glqq Louis Martin\grqq~ no tenía foto. Sus viajes, confinados a la zona Schengen, eran un susurro sin rostro en un mundo sin fronteras.
\end{itemize}

Mientras tanto, en las oficinas de Stellarius Lux, Louis, ajeno al torbellino que su invisibilidad desataba, servía un café a Joris con la calma de un monje. \glqq Joris,\grqq~ dijo, su voz suave como el rumor de un río, \glqq he estado pensando en el astillero. Tal vez necesitemos una aleación más resistente para los estabilizadores inerciales. Las fuerzas podrían ser\ldots\grqq

Joris, con un suspiro que parecía cargar el peso del firmamento, se frotó el puente de la nariz. \glqq Louis, por favor, solo tómate el café.\grqq

Y así, Louis Martin, sin saberlo, había logrado lo que los espías más astutos no podían: una invisibilidad perfecta en un mundo donde cada sombra era escrutada. Lo había hecho sin intentarlo, como si el universo, en un guiño travieso, lo hubiera elegido para ser su secreto mejor guardado.
