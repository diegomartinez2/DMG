\part*{Die Geburt}

% % % % % % % % % % % % % % % % % % % % % % % % % % % % % % % % %
% % % % % % % % % % % % % % % % % % % % % % % % % % % % % % % % %
% % % % % % % % % % % % % % % % % % % % % % % % % % % % % % % % %
% % % % % % % % % % % % % % % % % % % % % % % % % % % % % % % % %
% % % % % % % % % % % % % % % % % % % % % % % % % % % % % % % % %
% % % % % % % % % % % % % % % % % % % % % % % % % % % % % % % % %
% % % % % % % % % % % % % % % % % % % % % % % % % % % % % % % % %
% % % % % % % % % % % % % % % % % % % % % % % % % % % % % % % % %
% % % % % % % % % % % % % % % % % % % % % % % % % % % % % % % % %
% % % % % % % % % % % % % % % % % % % % % % % % % % % % % % % % %
% % % % % % % % % % % % % % % % % % % % % % % % % % % % % % % % %
% % % % % % % % % % % % % % % % % % % % % % % % % % % % % %

\chapter*{Ankündigungen}
\initial{D}{ie} ersten Anzeichen

\chapter*{Falscher Alarm}
\initial{D}{och}

\chapter[Un Mosaico de Caos]{Capítulo 14: Un Mosaico de Caos}
\lettrine[lines=2, loversize=0.3, lraise=0]{E}{l} mundo no desentrañó por completo el enigma de lo ocurrido en la estratosfera, pero sus efectos se filtraron como un río invisible que inunda valles ocultos. La información llegó en fragmentos, un rompecabezas tejido con hilos de terror y oportunismo, desplegándose a través de pantallas que parpadeaban como ojos febriles en la oscuridad global.
\section{BREAKING NEWS}
\textbf{REUTERS.COM - 14:27 CEST}\\
APAGÓN MASIVO EN CHINA PROVOCA CAOS Y PÁNICO EN DIVERSAS CIUDADES

Pekín, China – Un fallo inexplicable en la red eléctrica ha sumido a la costa este de China y a gran parte del sudeste asiático en una oscuridad que parece devorar el día. Las autoridades chinas guardan silencio como un secreto ancestral, pero fuentes militares susurran de \glqq anomalías energéticas\grqq~ ligadas a un ensayo balístico que se torció como un sueño roto. Los mercados de valores se desploman como torres de Babel, y las comunicaciones se evaporan, dejando un vacío donde antes latía el pulso del mundo.
\section{@CITIZENJOURNALIST123}
\textbf{YOUTUBE.COM - 15:01 CEST}\\
Toma 1: Caos en Shanghái

(Video grabado con un teléfono. Las luces de Shanghái parpadean frenéticamente, como estrellas agonizantes, antes de extinguirse por completo. Los coches se detienen en un coro de frenos chirriantes, el tráfico se congela en un tableau de metal inmóvil. Gritos se elevan como lamentos de un coro invisible, y sirenas de emergencia aúllan en la distancia. La pantalla del teléfono se vuelve negra, un abismo que traga la luz)

\begin{quote}
\calli
\glqq Mi teléfono\ldots~ mi teléfono se apagó. Todo se ha ido a la mierda. ¿Qué está pasando? Es como si el cielo hubiera caído sobre nosotros, un velo de oscuridad que ahoga la ciudad.\grqq
\end{quote}
\section{ENTREVISTA EN LA CALLE}
\textbf{WASHINGTON D.C. - DÍA SIGUIENTE}

(Una periodista de la CNN interroga a un analista militar retirado bajo el sol implacable de la capital, donde las sombras de los monumentos parecen guardianes de secretos olvidados)

\textbf{Periodista:} General, la Casa Blanca anuncia un plan de ayuda humanitaria a China, pero los rumores susurran que la verdadera preocupación es la escasez de chips. ¿Es cierto que Taiwán también ha sido tocada por esta plaga invisible?

\textbf{General:} No puedo desentrañar los hilos de la inteligencia, pero le diré esto: la tecnología que nutre nuestra economía, nuestros cazas F-35, nuestros ordenadores que laten como corazones mecánicos\ldots~ todo fluye del Pacífico. Si Taiwán cae, América cae con ella. Esto no es solo una crisis humanitaria; es un golpe al alma de nuestra seguridad nacional, un pulso que apaga el fuego de nuestra supremacía.
\section{La Oportunidad Europea}
\textbf{COMUNICADO DE PRENSA DE LA COMISIÓN EUROPEA}\\
\textbf{BRUSELAS, BÉLGICA - DÍA 3}\\
EUROPA LANZA EL \glqq ACTA DE SOBERANÍA TECNOLÓGICA\grqq~ PARA GARANTIZAR EL SUMINISTRO DE CHIPS

\emph{La Unión Europea anuncia hoy una iniciativa de emergencia para tejer su propio ecosistema de producción de chips de vanguardia. La decisión nace de la crisis energética que ha devorado la producción asiática, un pulso invisible que ha dejado al mundo a oscuras. La UE invertirá en una gigafactoría en un país miembro, un bastión para que el continente no vuelva a depender de cadenas de suministro frágiles como hilos de araña. El proyecto, liderado por la empresa holandesa ASML, busca asegurar el futuro tecnológico de Europa, un faro en la noche que ilumine nuestra soberanía.}
\subsection{Entrevista con un ejecutivo de ASML}
\textbf{Periodista:} ¿Qué significa esta inversión para la compañía, en medio de este mosaico de caos?

\textbf{Ejecutivo:} Significa que Europa despierta de un sueño largo, como un gigante que estira sus miembros. Siempre hemos sido los maestros de la litografía, pero nos faltaba el aliento político. Ahora lo tenemos. El mundo nos necesita, como un náufrago necesita un faro. Y el mundo, a partir de ahora, vendrá a nosotros, arrastrado por el río de la necesidad.
\section{La Jugada Británica y el Ascenso de la India}
\textbf{BBC News: Reportaje Especial}\\
\textbf{ISLAS SHETLAND, ESCOCIA - DÍA 4}

(Se muestra una imagen de vastos campos de hierba ondeando como un mar verde, y miles de aerogeneradores girando al compás del viento fuerte y constante, como guardianes eternos de una isla perdida)

\textbf{Reportero:} El gobierno británico anuncia planes para erigir una planta de semiconductores en las Islas Shetland, Escocia. Alimentada por el aliento inagotable del viento, se considera una inversión estratégica para el Reino Unido, un intento de liberarse de las cadenas de la UE y EE.UU. El proyecto enfrenta desafíos titánicos: ingenieros deben tejer puertos y carreteras para mover los pesados equipos de litografía. El gobierno británico afirma que esta es la única forma de garantizar el futuro tecnológico del país, un faro en la tormenta que ilumine su independencia.
\subsection{BLOG DE TECNOLOGÍA}
\textbf{NEWDELHIHACKER.BLOGSPOT.COM - DÍA 5}\\
Somos el futuro, no el pasado

No les mentiré. El EMP chino nos ha tocado, como un pulso invisible que roza las orillas de nuestro vasto río. No tanto como a ellos, pero suficiente para despertar al gobierno, que ahora ve el talento que late en nuestras venas. El video de un joven en YouTube ha hecho más por nuestra industria que cualquier decreto: un faro que ilumina el camino. China está atrapada en su caos, tejiendo narrativas para contener el diluvio, mientras nosotros discutimos la construcción de nuestra propia industria de chips. No solo igualaremos; superaremos, como un río que se desborda y nutre nuevos valles.
\section{El Sueño de Ícaro y el Precio de la Victoria}
\textbf{DIARIO PERSONAL DE JORIS DE VRIES}\\
\textbf{Hamburgo, Alemania - Día 6}

\emph{El Sueño de Ícaro se alza majestuoso en el astillero, como un titán dormido que sueña con las estrellas. Lo miro cada día, y me susurra promesas de vuelo. La nave ya está completa: la aviónica, los sensores, los sistemas de control\ldots~ Los chips que necesitábamos los teníamos ya, un tesoro guardado antes del pulso. Los retrasos fueron por el ataque terrorista y la burocracia política, un velo de sombras que nos demoró. El único problema es que nos falta el núcleo, la pieza más valiosa del mundo, robada como un secreto arrancado del corazón de la noche. Lo que es peor, nos faltan los nuevos condensadores de Fluzo que pedimos a Louis, diseñados de los errores filtrados por Anonymous, un regalo inesperado del caos.

Y ahora, el EMP chino provoca una escasez que devora el mundo, un pulso que iba a ser nuestro aliado se convierte en enemigo. Si encontramos el núcleo, si lo recuperamos, no podremos construir las nuevas piezas; el río de chips se ha secado. Por si fuera poco, los políticos en Bruselas discuten quiénes irán en la nave: una tripulación de cuatro. Los italianos proponen a una de las suyas, una mujer famosa por sus videos en YouTube, que nos muestra su vida en la Estación Espacial Internacional como un cuento estelar. El médico, un hombre serio como la gravedad, la acompañará. Y los otros dos\ldots~ Louis y yo. Al principio me negué, pero Louis me dijo que somos los únicos que sabemos cómo funciona la nave. Él, el genio que la creó; yo, el que la construyó. Solo nosotros podemos solucionar un problema en el espacio, si el cosmos nos desafía. Pero ¿qué pasará si no regresamos? La nave está construida, pero en este caos global, siento que nunca tocaremos las estrellas. Nos quedamos sin corazón, sin cerebro, y por primera vez, sin futuro, como un río seco en un desierto eterno.}
