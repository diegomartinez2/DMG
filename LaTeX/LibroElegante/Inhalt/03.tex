

% % % % % % % % % % % % % % % % % % % % % % % % % % % % % % % % %
% % % % % % % % % % % % % % % % % % % % % % % % % % % % % % % % %
% % % % % % % % % % % % % % % % % % % % % % % % % % % % % % % % %
% % % % % % % % % % % % % % % % % % % % % % % % % % % % % % % % %
% % % % % % % % % % % % % % % % % % % % % % % % % % % % % % % % %
% % % % % % % % % % % % % % % % % % % % % % % % % % % % % % % % %
% % % % % % % % % % % % % % % % % % % % % % % % % % % % % % % % %
% % % % % % % % % % % % % % % % % % % % % % % % % % % % % % % % %
% % % % % % % % % % % % % % % % % % % % % % % % % % % % % % % % %
% % % % % % % % % % % % % % % % % % % % % % % % % % % % % % % % %
% % % % % % % % % % % % % % % % % % % % % % % % % % % % % % % % %
% % % % % % % % % % % % % % % % % % % % % % % % % % % % % %

\chapter*{Mehr zur Vorgeschichte}


\lettrine[lines=2, loversize=0.3, lraise=0]{\initfamily N}{ach }

\chapter{El Gran Bloqueo y la Pequeña Paloma}

\lettrine[lines=2, loversize=0.3, lraise=0]{E}{n} la penumbra de una sala de mando en el Pentágono, los monitores parpadeaban como luciérnagas atrapadas en una red de acero, proyectando un resplandor frío sobre los rostros de los generales y analistas. El aire olía a café rancio y a decisiones aplazadas, mientras un mapa estelar giraba en una pantalla central, trazando la órbita imposible del satélite de Stellarius Lux, que había danzado desde la Tierra hasta Urano en un suspiro de tiempo. El almirante Jackson, un hombre de mandíbula cuadrada y ojos como mares en tormenta, señaló el mapa con un dedo que parecía tallado en granito.

\begin{quote}
\calli
\glqq Esto no es un satélite, señores,\grqq~ gruñó, su voz un trueno contenido. \glqq Es una declaración de guerra. Una empresa de Luxemburgo, un ducado de opereta, ha humillado a nuestras agencias espaciales. ¡Y ahora sabemos que están construyendo una nave!\grqq
\end{quote}

El general Ramírez, de la Fuerza Espacial, se inclinó hacia adelante, su uniforme crujiendo como hojas secas bajo el peso de su autoridad. \glqq Hemos confirmado el astillero en Alemania. Están forjando algo grande, algo que desafía nuestras leyes físicas. Pero el bloqueo\ldots el bloqueo será complicado. La Unión Europea no tomará a bien que cerremos sus puertos.\grqq

El almirante Jackson esbozó una sonrisa torcida, como si el destino le hubiera servido un trago amargo. \glqq No necesitamos cerrar puertos. Solo necesitamos\ldots persuadir. Una inspección sorpresa, una auditoría de seguridad. Los alemanes son meticulosos; un solo papel fuera de lugar y el astillero se detendrá como un reloj roto.\grqq

En Bruselas, sin embargo, los vientos de la intriga soplaban en otra dirección. En una sala de techos altos y lámparas de cristal, donde el eco de los pasos resonaba como un cántico olvidado, los líderes de la Unión Europea se reunían en secreto. La presidenta de la Comisión, una mujer de mirada afilada y voz como el susurro de un río, sostenía un informe marcado con sellos confidenciales.

\begin{quote}
\calli
\glqq Stellarius Lux es nuestra joya,\grqq~ dijo, su tono firme pero cargado de un orgullo maternal. \glqq Si los americanos o los chinos intentan bloquear el astillero, será un acto de agresión económica. Pero no podemos protegerlo abiertamente; el mundo no está listo para conocer \emph{El Fanal}.\grqq
\end{quote}

Un diplomático francés, con un pañuelo de seda asomando como una flor en su bolsillo, alzó una mano. \glqq Propongo una maniobra sutil. Una inspección ambiental, tal vez. Decimos que el astillero contamina el Báltico. Eso detendrá cualquier intervención extranjera sin levantar sospechas.\grqq

La presidenta asintió, sus ojos brillando como el reflejo de la luna en un estanque. \glqq Y mientras tanto, protegemos a Louis Martin. Ese hombre es un milagro, un cometa que cruza nuestro cielo sin que nadie lo vea.\grqq

\section{La Paloma que Desafió al Mundo}

Mientras los titanes de la Tierra conspiraban, en un rincón olvidado de Luxemburgo, una paloma blanca como la nieve posada en un alféizar parecía observar el mundo con una sabiduría antigua. Louis Martin, ajeno a las tormentas que su invención había desatado, trabajaba en una oficina desordenada, rodeado de papeles garabateados con ecuaciones que parecían hechizos. La luz del atardecer se filtraba por la ventana, pintando su rostro con tonos dorados, como si el sol mismo lo bendijera.

Joris de Vries irrumpió en la habitación, con el rostro pálido y los pasos apresurados, como un hombre perseguido por sombras. \glqq Louis, ¡escucha! Los americanos han enviado inspectores al astillero. Dicen que es una auditoría de seguridad, pero sabemos lo que buscan. Y los chinos\ldots han hackeado nuestros servidores. No encontraron nada, pero están cerca.\grqq

Louis alzó la vista, sus ojos brillando con una calma que parecía desafiar el caos del universo. \glqq ¿Cerca? Joris, nadie está cerca. \emph{El Fanal} no está en los servidores, ni en el astillero. Está aquí,\grqq~ dijo, tocándose la frente con un dedo, como si señalara un templo sagrado.

Joris se dejó caer en una silla, el peso de la conspiración aplastándole los hombros. \glqq No entiendes, Louis. Si te encuentran, si descubren quién eres, no habrá lugar en la Tierra donde puedas esconderte.\grqq

La paloma, como si hubiera escuchado la advertencia, alzó el vuelo, su silueta recortada contra el crepúsculo. Louis la siguió con la mirada, sonriendo como un niño que descubre un secreto. \glqq Tal vez no necesite esconderme en la Tierra,\grqq~ murmuró, y sus palabras flotaron en el aire, ligeras como plumas.

En ese momento, en un callejón de Berlín, un agente del MSS aguardaba frente al astillero, con una cámara oculta en su abrigo. Había recibido órdenes de capturar una imagen de Louis Martin, el hombre sin rostro. Pero cuando el científico salió del edificio, acompañado de Joris, una ráfaga de viento levantó un remolino de hojas secas, y la cámara del agente captó solo un borrón, como si el mundo mismo conspirara para proteger a Louis.

En Washington, un analista de la CIA, revisando imágenes satelitales, maldijo en voz baja. Había localizado a Joris en el astillero, pero Louis, una vez más, era un espectro. En la imagen, una paloma blanca cruzaba el encuadre, su ala desplegada como un velo sobre el rostro del científico. \glqq Otra maldita paloma,\grqq~ gruñó el analista, arrojando el informe al escritorio.

Y así, mientras las potencias del mundo tejían sus redes de intriga, Louis Martin seguía siendo un enigma, protegido no por ejércitos ni cifrados, sino por una paloma que parecía llevar en sus alas el aliento del destino. En su oficina, Louis garabateó una nueva ecuación, y el papel, iluminado por la luz del crepúsculo, pareció brillar como si contuviera las llaves del universo.
