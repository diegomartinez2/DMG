
% % % % % % % % % % % % % % % % % % % % % % % % % % % % % % % % %
% % % % % % % % % % % % % % % % % % % % % % % % % % % % % % % % %
% % % % % % % % % % % % % % % % % % % % % % % % % % % % % % % % %
% % % % % % % % % % % % % % % % % % % % % % % % % % % % % % % % %
% % % % % % % % % % % % % % % % % % % % % % % % % % % % % % % % %
% % % % % % % % % % % % % % % % % % % % % % % % % % % % % % % % %
% % % % % % % % % % % % % % % % % % % % % % % % % % % % % % % % %
% % % % % % % % % % % % % % % % % % % % % % % % % % % % % % % % %
% % % % % % % % % % % % % % % % % % % % % % % % % % % % % % % % %
% % % % % % % % % % % % % % % % % % % % % % % % % % % % % % % % %
% % % % % % % % % % % % % % % % % % % % % % % % % % % % % % % % %
% % % % % % % % % % % % % % % % % % % % % % % % % % % % % %

\chapter*{Der fünfte Schwangerschaftsmonat}
\lettrine[lines=2, loversize=0.3, lraise=0]{\initfamily U}{nsere}

\chapter{El Anónimo y el Millón de Dólares}

\lettrine[lines=2, loversize=0.3, lraise=0]{E}{n} una sala de monitoreo de la CIA, oculta bajo las entrañas de Langley, donde la luz fluorescente parpadeaba como un corazón enfermo, una docena de analistas permanecían en silencio, sus rostros bañados por el resplandor azulado de una pantalla gigante. En ella, un canal encriptado de YouTube mostraba una figura envuelta en sombras, con una máscara de Guy Fawkes que parecía sonreír con la burla de un dios travieso. Su voz, distorsionada como el eco de un oráculo mecánico, cortaba el aire con una claridad que helaba la sangre.

\begin{quote}
\calli
\glqq Saludos, ciudadanos del mundo,\grqq~ proclamó la figura, sus palabras resonando como un trueno en un cielo sin estrellas. \glqq Somos Anonymous. No servimos a ningún país, no obedecemos a ningún rey. El proyecto FTL, \emph{El Fanal}, no es un arma ni un trofeo para los gobiernos. Es un faro para la humanidad, y no permitiremos que lo apaguen.\grqq
\end{quote}

La pantalla destelló con documentos robados: correos electrónicos de la CIA, garabateados con planes para atrapar a Joris de Vries; la transcripción de la videoconferencia con el banco, donde Louis había dejado caer el nombre de \emph{El Fanal}; y un informe que detallaba la estrategia estadounidense para encerrar a Louis en una jaula de oro. Luego, gráficos de la explosión en Shenzhen y la deformación espacial en Siberia aparecieron, como visiones de un apocalipsis que aún no llegaba. \glqq Los gobiernos juegan a ser dioses con una tecnología que no comprenden,\grqq~ continuó la voz. \glqq No son científicos, son monos con dinamita. Y la dinamita explota. La única forma de detenerlos es el conocimiento, la verdad que ilumina. Por eso, hemos lanzado una campaña: un dólar, un susurro de apoyo para que Stellarius Lux no caiga bajo el yugo de los poderosos.\grqq

El video terminó con el logo de Anonymous, un círculo roto que parecía desafiar el orden del mundo, y una dirección web que brillaba como un faro en la penumbra digital. El analista jefe, un hombre de rostro surcado por años de secretos, se pasó la mano por la cara, como si quisiera borrar la realidad. \glqq ¿Cómo saben todo esto?\grqq~ gruñó, su voz un murmullo de derrota. \glqq ¿Quién les filtró los documentos?\grqq

Un joven analista, con los ojos encendidos por una mezcla de temor y fascinación, se encogió de hombros. \glqq No lo sé, señor. Pero la campaña ya tiene más de un millón de donaciones. La mayoría son de un dólar, pero algunas alcanzan miles. Han recaudado doce millones en horas.\grqq

El jefe, con un suspiro que parecía exhalar el peso del mundo, murmuró: \glqq Los monos han aprendido a encender la mecha.\grqq

\section{La Guerra del Caos}

En las calles digitales, donde las ideas se movían como ríos subterráneos, la campaña de crowdfunding se convirtió en un huracán. No era solo dinero; era un arma, una rebelión silenciosa contra los titanes del poder. Anonymous, ese colectivo sin rostro ni bandera, era una fuerza de la naturaleza, un vendaval de anarquía que mezclaba a soñadores, conspiradores y bromistas en un torbellino impredecible. Algunos buscaban la verdad del FTL, anhelando un futuro donde el cosmos estuviera al alcance de todos. Otros, simplemente, reían ante el caos, deleitándose en la humillación de los gobiernos que tropezaban en su propia arrogancia.

Un día, filtraban correos que desnudaban el bloqueo de los fondos de Joris; al siguiente, inundaban la red con memes sobre la explosión en Shenzhen, caricaturas de generales chinos huyendo de una mota de polvo. Luego, robaban datos de la NASA, exponiendo los planos incompletos del \emph{Condensador de Fluzo} con un guiño burlón al cine de los ochenta. No había un plan claro, solo el placer de ver a los poderosos tambalearse, como titanes sorprendidos por un enjambre de abejas.

Para la CIA y el MSS, Anonymous era una plaga, un fastidio que no podían atrapar. Sus ataques eran como relámpagos: impredecibles, fugaces, imposibles de contener. Mientras los agentes rastreaban a Louis Martin, Anonymous lanzaba distracciones, memes, filtraciones, convirtiendo la cacería en una comedia absurda. Louis, con su invisibilidad casi mística, era un espejo de Anonymous: ambos, anónimos e intocables, desafiaban el sistema con una existencia que no podían comprender. Pero donde Louis era un alma solitaria, movida por la música del cosmos, Anonymous era un coro caótico, cantando por el puro gozo de desordenar el mundo.

\section{El Dilema del Poder}

En los pasillos del poder estadounidense, donde las decisiones se tomaban bajo el peso de banderas y juramentos, el gobierno buscaba desesperadamente apagar el incendio de la campaña. Bloquear la plataforma de crowdfunding, como Kickstarter, parecía la solución más simple, pero era un arma de doble filo. Si cerraban la campaña, el dinero volvería a los donantes, que abrirían otra en una plataforma extranjera, haciendo que el gobierno pareciera un tirano que teme la verdad. La indignación pública crecería, y el nombre de Joris brillaría aún más, como un mártir coronado por la luz de un millón de dólares.

Presionar a la plataforma era otra opción, pero también un riesgo. Si se negaban, el gobierno tendría que clausurarla, hundiendo un pilar de la economía digital y alimentando las teorías de conspiración. La tercera vía, la desinformación, era un juego viejo: artículos que tildaran a Stellarius Lux de estafa, rumores de que los fondos se usaban para fines oscuros. Pero el público, olfateando la mano del poder, podría volverse contra ellos, convirtiendo cada mentira en un ladrillo más en el pedestal de Joris.

En Luxemburgo, Joris de Vries, atrapado en su oficina donde el aire olía a café amargo y promesas rotas, leía los titulares y veía los números de la campaña crecer como un río desbordado. Cada dólar era un voto, cada donación un grito que resonaba en el vacío. Pero él sabía que el dinero no detendría a los gobiernos, solo los enfurecería más. \glqq Louis,\grqq~ murmuró, mirando al genio que garabateaba ecuaciones en un cuaderno, \glqq esto se nos escapa de las manos.\grqq

Louis alzó la vista, sus ojos brillando con la calma de un lago bajo la luna. \glqq Joris, el universo no se preocupa por los gobiernos. \emph{El Fanal} no es para ellos, ni para Anonymous. Es para las estrellas.\grqq~ En ese momento, una paloma blanca se posó en el alféizar, sus alas capturando un rayo de sol que iluminó el rostro de Louis, como si el cosmos mismo le guiñara un ojo.

Fuera, en las calles de Luxemburgo, un agente de la CIA intentó capturar una imagen de Louis mientras caminaba hacia el laboratorio. Pero una ráfaga de viento, levantada por un autobús que pasaba rugiendo como un dragón, lanzó un remolino de hojas que cubrió su lente. La cámara solo captó la silueta de la paloma, volando libre, como un guardián que reía ante los ojos ciegos del mundo.
