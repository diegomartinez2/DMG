\chapter[El Despegue Silencioso y la Energía del Futuro]
        {El Despegue Silencioso y la Energía del Futuro}

\lettrine[lines=2, loversize=0.3, lraise=0]{M}{ientras} el mundo alzaba los ojos al cielo, aguardando un susurro de Alfa Centauri que no llegaría en cuatro años, Europa tejía su destino con la sutileza de un poeta que escribe en la penumbra. En el astillero de Hamburgo, donde el viento del Báltico llevaba ecos de mares antiguos, la prensa se apiñaba como un coro expectante, cámaras destellando como luciérnagas en la noche. Esperaban el rugido de un cohete, el fuego que había marcado los saltos de la \textit{Pionero} y el \textit{Dragón Ascendente}. Pero \textit{El Sueño de Ícaro}, un titán de acero y sueños, no rugió. Se alzó en un silencio que era más milagro que mecánica, un desafío al peso de la Tierra.

Joris de Vries y Louis Martin, arquitectos de lo imposible, habían afinado el motor FTL con la precisión de un laudero que tensa las cuerdas de un arpa. El impulsor no inercial, un susurro de física que burlaba la gravedad, emitió pulsos suaves, como el latido de un corazón estelar. La nave, un coloso que parecía flotar como una pluma en el viento, se elevó sin llamas, sin estruendo, solo con el murmullo del aire que se apartaba como un río ante un profeta. Los periodistas, con plumas inmóviles y cámaras temblando, quedaron mudos, testigos de un espectáculo que parecía más magia que ciencia. \textit{El Sueño de Ícaro} ascendió, un cisne de metal que danzaba hacia el cielo, dejando tras de sí un silencio que resonaba como un himno.

A bordo, la tripulación de cuatro —la capitana Agnese Lombardi, una italiana cuya voz en los vídeos de la Estación Espacial Internacional había inspirado a millones; el médico Ricardo Torres, un español de mirada serena que leía el cuerpo humano como un mapa estelar; Joris, el ingeniero que había dado forma al sueño; y Louis, el genio cuyas ecuaciones eran un puente al cosmos— se preparaba en el puente. Agnese, con manos firmes sobre los controles, susurró:

«Siamo pronti.»

Ricardo, ajustando un monitor médico, asintió en silencio. Joris, con el corazón apesadumbrado por el núcleo robado que aún no recuperaban, miraba por la ventana, donde Hamburgo se desvanecía como un recuerdo.

Louis, con ojos que parecían contener galaxias, activó el motor FTL. No se dirigían a Alfa Centauri, el campo de batalla de las superpotencias. Su destino era Epsilon Eridani, a diez años luz, un sistema que Louis, en sus cálculos nocturnos, había declarado «un poema más rico, un lienzo donde la vida podría haber tejido su propio tapiz». La nave se desvaneció en un instante, sin rastro, sin eco, como si el cosmos la hubiera abrazado con ternura. La prensa, atónita, garabateó titulares: «Europa elige un nuevo horizonte. Epsilon Eridani, la apuesta silenciosa». La humanidad tenía una nueva historia, no de competencia, sino de exploración.

En la Tierra, mientras el eco del despegue resonaba, \textit{El Fanal} comenzaba a transformar el mundo como un río que, al desbordarse, fertiliza valles secos. Las primeras unidades de conversión de materia a energía, pequeños faros de abundancia, llegaron a las zonas devastadas por el pulso electromagnético chino. En Shanghái, donde las luces habían muerto, las unidades brillaban como luciérnagas, alimentando hospitales, escuelas, hogares. En Taiwán, los generadores de Stellarius Lux resucitaban fábricas de chips, un latido que prometía sanar la cadena rota del mundo.

El gobierno chino, acorralado por la vergüenza del pulso, tejió una narrativa audaz. Los medios estatales proclamaban que la tecnología era un triunfo nacional, un «Dragón de Luz» que iluminaba el camino. Los ciudadanos, con teléfonos restaurados y calles iluminadas, sabían la verdad: \textit{El Fanal} era europeo, un regalo de Stellarius Lux, protegido por una patente que ningún gobierno podía tocar. Pero celebraban igual, no por el orgullo nacional, sino por la libertad que la energía limpia les devolvía, un futuro sin el yugo del petróleo o el carbón. En las calles de Pekín, los niños corrían con linternas alimentadas por un gramo de materia, riendo como si las estrellas mismas hubieran bajado a jugar.

En Hamburgo, Joris, de regreso en el astillero tras el lanzamiento, contemplaba el horizonte donde \textit{El Sueño de Ícaro} había desaparecido. La patente de \textit{El Fanal} lo había convertido en un titán, pero el núcleo robado, aquel corazón lunar perdido en un casillero ignoto, seguía siendo un espectro que lo atormentaba. Louis, sentado junto a un ventanal, garabateaba ecuaciones con la calma de quien conversa con el tiempo.

«Joris», dijo, su voz un susurro que parecía nacer del viento, «Epsilon Eridani nos espera con secretos que Alfa Centauri no conoce. Pero el núcleo… si lo recuperamos, \textit{El Fanal} no solo iluminará ciudades. Puede protegerlas, guiarlas, incluso llevarnos a casa.»

Joris, atrapado entre la maravilla y el peso de su poder, no respondió. El mundo celebraba un nuevo amanecer, pero él sentía el murmullo de un futuro incierto. En el ventanal, una paloma blanca, con los ojos profundos de Miguel García Solís, ladeó la cabeza, sus alas un destello que parecía llevar el eco de Epsilon Eridani. La humanidad, liberada de la escasez, danzaba en un lejano oeste estelar, pero Joris sabía que el cosmos, como un dios caprichoso, guardaba sus secretos con celosía.
