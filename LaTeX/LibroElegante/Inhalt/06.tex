\chapter*{Der dritte Schwangerschaftsmonat}
\lettrine[lines=2, loversize=0.3, lraise=0]{\initfamily E}{in} Beispiel für Kalligrafie:


\begin{quote} \calli

Dies ist ein Beispiel für einen Paragraph geschrieben mit dem Befehl CALLI. Es ist etwas schwierig zu lesen, aber vielleicht sinnvoll für kurze Abschnitte.

Der Befehl ist zusammen mit dem QUOTE-Befehl genutzt, so dass auch die Formatierung anders ist.
\end{quote}

% % % % % % % % % % % % % % % % % % % % % % % % % % % % % % % % %
% % % % % % % % % % % % % % % % % % % % % % % % % % % % % % % % %
% % % % % % % % % % % % % % % % % % % % % % % % % % % % % % % % %
% % % % % % % % % % % % % % % % % % % % % % % % % % % % % % % % %
% % % % % % % % % % % % % % % % % % % % % % % % % % % % % % % % %
% % % % % % % % % % % % % % % % % % % % % % % % % % % % % % % % %
% % % % % % % % % % % % % % % % % % % % % % % % % % % % % % % % %
% % % % % % % % % % % % % % % % % % % % % % % % % % % % % % % % %
% % % % % % % % % % % % % % % % % % % % % % % % % % % % % % % % %
% % % % % % % % % % % % % % % % % % % % % % % % % % % % % % % % %
% % % % % % % % % % % % % % % % % % % % % % % % % % % % % % % % %
% % % % % % % % % % % % % % % % % % % % % % % % % % % % % %

\chapter{La Nueva Carrera Armamentista y los Monos con Dinamita}

\lettrine[lines=2, loversize=0.3, lraise=0]{E}{l} eco de la explosión en Shenzhen reverberó por el mundo como un lamento primigenio, un grito que no era de hombre ni de máquina, sino del universo mismo al ser rasgado por una mota de polvo acelerada al filo de la luz. Aquella partícula, insignificante como un grano de arena, había desatado una furia cinética que redujo media ciudad a un cráter humeante, un altar de cenizas donde el sueño del FTL se convirtió en una pesadilla. En las semanas que siguieron, el mundo se sumió en una fiebre silenciosa, una carrera armamentista que no se libraba en campos de batalla, sino en laboratorios subterráneos donde la ciencia y la ambición se entrelazaban como amantes condenados.

Las superpotencias, armadas con los planos robados de Stellarius Lux y las migajas del artículo de Louis Martin en arXiv, se lanzaron a replicar \emph{El Fanal}. Tenían los emisores, las guías de onda, las ecuaciones garabateadas como hechizos en papel, pero les faltaba el alma del descubrimiento: la comprensión profunda de la geometría de Riemann y los haces fibrados que Louis había tejido en su mente, como un tapiz que solo él podía leer. Los resultados fueron un caos que parecía burlarse del orden humano.

En el desierto de Siberia, donde el viento ululaba como un coro de lobos, un equipo de científicos rusos intentó un experimento a pequeña escala. Encendieron un prototipo, y por un instante, el aire mismo pareció contener el aliento. Pero el haz de energía, caprichoso como un dios antiguo, se desvió, y un asteroide de hielo, un viajero errante del cosmos, fue arrancado de su órbita y lanzado a velocidades imposibles. Nadie lo notó, salvo los ojos fríos de un satélite ruso, que guardó el secreto como un monje en su celda.

En un búnker bajo las llanuras de Nevada, los ingenieros estadounidenses, con rostros pálidos y manos temblorosas, encendieron su propio prototipo FTL. No hubo explosión, sino algo más extraño: el espacio dentro del búnker se plegó sobre sí mismo, como una sábana doblada por manos invisibles. Las paredes crujieron, el metal gimió, y el prototipo desapareció en una burbuja de nada, un punto que ya no pertenecía a este universo. Los científicos huyeron, dejando tras de sí un silencio que pesaba como una maldición.

En China, la herida de Shenzhen aún sangraba. El experimento fallido, un intento de salto FTL, había convertido una mota de polvo en un proyectil de devastación. El gobierno, con la precisión de un cirujano que oculta su error, creó una zona de exclusión donde el polvo y el dolor se mezclaban en un paisaje de muerte. En laboratorios secretos, nuevos prototipos se encendían con la esperanza de domar la bestia, pero cada prueba era un recordatorio de que el FTL no era un juguete, sino un titán que exigía respeto.

En la oficina de Stellarius Lux en Luxemburgo, donde la luz del crepúsculo pintaba las paredes con tonos de ámbar y melancolía, Joris de Vries se paseaba como un alma en pena, su rostro surcado por el peso de un mundo que lo aplastaba. La oferta de Estados Unidos era una soga disfrazada de salvavidas; la de China, un veneno servido en una copa de cristal. Agentes de la CIA lo seguían como sombras pegajosas, mientras el MSS acechaba desde las esquinas, sus ojos invisibles en cada calle. Joris, con la voz quebrada como una rama seca, intentó una vez más persuadir a Louis.

\begin{quote}
\calli
\glqq Abandona esto, Louis. Es demasiado grande. Nos destruirá a todos.\grqq
\end{quote}

Louis, sentado frente a un escritorio cubierto de papeles que parecían mapas de constelaciones, alzó la vista con una calma que desafiaba la tormenta.

\begin{quote}
\calli
\glqq Joris, no lo entiendes,\grqq~ dijo, su voz suave como el susurro de un río. \glqq La física es la misma. Tienen los planos, pero no la partitura. Es como darle a un mono la batuta y esperar que dirija la Novena de Beethoven. Sin la música en el alma, solo hay caos.\grqq
\end{quote}

\section{Las Calles de Luxemburgo}

Fuera, en las calles de Luxemburgo, Louis caminaba ajeno al torbellino que su genio había desatado. Un equipo de la CIA, escondido tras las cortinas de un café, apuntó una cámara de alta resolución hacia él. Pero en ese instante, como si el destino tejiera su propio hechizo, un grupo de turistas, absortos en sus teléfonos como sacerdotes en sus oraciones, tropezó en una danza caótica. Sus brazos y dispositivos bloquearon el lente, y la única imagen capturada fue un dedo borroso, alzado al cielo como una burla. Una paloma blanca, posada en un farol cercano, ladeó la cabeza, sus ojos brillando como perlas que guardaban los secretos del universo.

En su laboratorio, Louis trabajaba solo, rodeado de pantallas que parpadeaban como estrellas atrapadas. Sus dedos trazaban ecuaciones que parecían danzar, y su mente, libre de las intrigas del mundo, navegaba por los confines del espacio-tiempo. No sabía que, mientras él soñaba con las estrellas, las superpotencias jugaban con su creación como monos con dinamita, encendiendo mechas que podían quemar el cielo. La paloma, ahora posada en el alféizar de su ventana, extendió sus alas, y por un instante, el laboratorio se llenó de una luz que no era de este mundo, como si el propio \emph{El Fanal} susurrara su aprobación.
