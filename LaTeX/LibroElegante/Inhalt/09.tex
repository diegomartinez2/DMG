

\chapter{La Conversación de los Ingenieros}

\lettrine[lines=2, loversize=0.3, lraise=0]{E}{n} un café de Luxemburgo, donde el aroma del café recién molido se entretejía con el susurro de las conversaciones olvidadas, Joris de Vries se hundió en una silla de terciopelo desgastado, su rostro oculto tras las páginas crujientes de un periódico que parecía un escudo contra el mundo. Su teléfono, posado sobre la mesa como un mensajero infatigable, vibraba con la noticia del día: el gobierno de Estados Unidos había sofocado la campaña de crowdfunding de Anonymous, un millón de donantes convertidos en activistas que ahora seguían un líder invisible, un espectro digital que reía en las sombras de la red. Pero Joris no ardía en furia; su mente era un remolino concentrado, un vórtice que devoraba las imágenes en su pantalla.

En un canal encriptado de YouTube, videos filtrados por Anonymous desfilaban como visiones de un apocalipsis mecánico. El primero mostraba a científicos rusos en el vasto desierto de Siberia, donde el viento aullaba como un lobo herido, intentando encender un prototipo del motor FTL. La cámara se saturó de energía, un resplandor blanco que cegaba como el sol del mediodía, y los hombres corrieron hacia la salida, sus siluetas borrosas como fantasmas en fuga. En el video, una burbuja de espacio y tiempo se hinchó como un sueño roto, y el prototipo desapareció, no en el lugar deseado, sino a miles de kilómetros, un error que parecía un susurro caprichoso del cosmos.

El segundo, grabado en un búnker subterráneo de Estados Unidos, donde el aire olía a metal y miedo contenido, capturaba a ingenieros huyendo mientras la pantalla se inundaba de luz. La burbuja se formó, pero se cerró con violencia, implosionando sobre sí misma en un micro-agujero negro que devoró el suelo como un hambre insaciable. El video se cortó abruptamente, dejando solo el eco de un desastre que Joris podía sentir en sus huesos.

Y el tercero, la herida abierta de China, era una sinfonía de caos: imágenes de satélites y teléfonos móviles entretejidas, mostrando una explosión que devoró media ciudad de Shenzhen, un hongo de fuego y polvo que ascendía como un lamento al cielo. Para el mundo, un accidente trágico; para Joris, la prueba irrefutable de que Pekín había intentado forjar un arma del FTL, un pecado contra la física que ahora sangraba en la tierra.

Mientras Joris devanaba los hilos de aquellos videos, como un cartógrafo trazando mapas de un territorio hostil, Louis Martin se sentó frente a él, con una taza de café humeante en la mano, su vapor ascendiendo como un velo de incienso. Joris le tendió el teléfono, un gesto silencioso que contenía el peso del mundo. Louis lo tomó, sus ojos absorbiendo las imágenes con la avidez de un profeta ante visiones divinas. Su rostro, usualmente un lago sereno sin ondas, se iluminó con un fulgor que parecía brotar de su interior, como si las estrellas mismas hubieran encendido una chispa en su alma.

\begin{quote}
\calli
\glqq Joris,\grqq~ murmuró Louis, su voz un río de emoción contenida, \glqq ¿te das cuenta de lo que significa esto?\grqq
\end{quote}

Joris, con el tono cansado de un hombre que carga el firmamento sobre los hombros, respondió: \glqq ¿Que los gobiernos juegan con una fuerza que no entienden, y que estamos atrapados en una guerra de locos, un torbellino que nos arrastrará a todos?\grqq

\begin{quote}
\calli
\glqq No, Joris. Es mucho más que sombras y amenazas. Es una confirmación, un eco del universo que responde a nuestra llamada. Todos esos fallos --los de Rusia, de Estados Unidos, de China-- nacen de un solo error: la imprecisión en el enfoque de la salida del agujero de gusano. Cada uno tropezó en el punto focal, pero de ese tropiezo surgieron prodigios inesperados.\grqq
\end{quote}

Louis extrajo una servilleta de papel, su superficie inmaculada como un lienzo virgen, y comenzó a dibujar con un lápiz que parecía una varita de hechicero. Sus trazos eran rápidos, fluidos, como ríos que se bifurcaban en constelaciones. \glqq En Rusia, el agujero de gusano se desenfocó apenas, un susurro de desviación, y la nave emergió en un lugar imprevisto, a miles de kilómetros. No fue un fracaso, Joris; fue un impulsor de haz, un puente que salta sin anclas.\grqq

Sonrió entonces, una sonrisa que Joris nunca había visto, un arco iris fugaz en un cielo tormentoso. \glqq En Estados Unidos, el agujero se cerró con furia, concentrando la materia en un punto infinitesimal, un micro-agujero negro que devoró el suelo como un dragón dormido. No un error, sino un manipulador de espacio, un titiritero que dobla la realidad a su antojo.\grqq

Y sonrió de nuevo, esa sonrisa que parecía iluminar el café entero, como si el sol hubiera decidido posarse en sus labios. \glqq Y en China, el agujero se selló sobre la mota de polvo, liberando la energía en una erupción explosiva, convirtiendo un grano insignificante en un proyectil que rasgó la ciudad. No un desastre, sino un proyector de energía, un sol en miniatura forjado en el caos.\grqq

Joris lo miró con incredulidad, sus ojos dos pozos de asombro. \glqq Louis, ¿me estás diciendo que los errores de los gobiernos han parido tres nuevas tecnologías, frutos amargos de su propia arrogancia?\grqq

\begin{quote}
\calli
\glqq Sí, Joris. \emph{El Fanal}, tal como lo concebimos, es un creador de agujeros de gusano, un portal que une los confines del vacío. Pero si jugamos con el enfoque, como un músico que varia la melodía, podemos invocarlo para otros fines. Y lo más crucial: podemos usarlo para el lanzamiento. Imagina, Joris: el motor impulsando la nave en la atmósfera, sin el rugido de cohetes ni el humo de la ambición ciega. Un impulsor no inercial, suave como el vuelo de un sueño. Construimos en Alemania, y desde el astillero, la elevamos al cielo como un pájaro que despierta.\grqq
\end{quote}

La cara de Joris se iluminó, un amanecer en su expresión agotada. La propuesta de Louis no era solo un escudo contra la guerra de superpotencias; era una llave para la eficiencia, un camino que reducía costes y tejía el proyecto con hilos de posibilidad. Pero Louis, con los ojos aún brillantes como estrellas fugaces, no había terminado su sinfonía.

\glqq Hay más, Joris. En la explosión de China, en ese fallo que devoró la ciudad, la energía no brotó del motor solo. Nació de la conversión de la materia misma, un alquimia que transmuta lo sólido en luz pura. Eso significa que nuestra tecnología no solo acelera la masa a través del vacío; puede convertir la materia en energía ilimitada. Imagínalo: el fin de la pobreza, como un río que inunda los desiertos; el fin de las guerras por recursos, como un viento que apaga las hogueras de la codicia. El fin de todo lo que nos ata a la tierra, Joris. Un nuevo amanecer para la humanidad.\grqq

Joris se quedó sin palabras, su aliento atrapado en el pecho como un pájaro enjaulado. Había llegado al café huyendo del caos, un peregrino en busca de refugio en las noticias amargas. Pero se marchó con una visión que trascendía sus miedos, un tapiz de futuro tejido en los fallos de los poderosos. El desorden de las superpotencias, como los errores en sus experimentos, no era un fin, sino un preludio. Y Joris, que se había sentido un peón en el tablero, ahora palpaba el poder de mover las piezas, de cambiar el mundo con un susurro de física.

Fuera del café, una paloma blanca, con alas que capturaban la luz del atardecer como un velo de plata, se posó en una mesa cercana. Ladeó la cabeza, como si escuchara los ecos de Miguel en las ecuaciones de Louis, y extendió sus alas, lista para llevar el secreto hacia los horizontes invisibles.
