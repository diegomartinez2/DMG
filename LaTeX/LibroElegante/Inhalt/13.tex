
% % % % % % % % % % % % % % % % % % % % % % % % % % % % % % % % %
% % % % % % % % % % % % % % % % % % % % % % % % % % % % % % % % %
% % % % % % % % % % % % % % % % % % % % % % % % % % % % % % % % %
% % % % % % % % % % % % % % % % % % % % % % % % % % % % % % % % %
% % % % % % % % % % % % % % % % % % % % % % % % % % % % % % % % %
% % % % % % % % % % % % % % % % % % % % % % % % % % % % % % % % %
% % % % % % % % % % % % % % % % % % % % % % % % % % % % % % % % %
% % % % % % % % % % % % % % % % % % % % % % % % % % % % % % % % %
% % % % % % % % % % % % % % % % % % % % % % % % % % % % % % % % %
% % % % % % % % % % % % % % % % % % % % % % % % % % % % % % % % %
% % % % % % % % % % % % % % % % % % % % % % % % % % % % % % % % %
% % % % % % % % % % % % % % % % % % % % % % % % % % % % % %

\chapter*{Der zehnte Schwangerschaftsmonat}

\lettrine[lines=2, loversize=0.3, lraise=0]{\initfamily U}{nsere}

\chapter[La Audacia China y el Fantasma del Pulso]{Capítulo 13: La Audacia China y el Fantasma del Pulso}
\lettrine[lines=2, loversize=0.3, lraise=0]{E}{n} el vasto teatro del cosmos, donde las estrellas observaban con indiferencia el frenesí de los mortales, el mundo había dejado atrás la lucha por el control de una tecnología para abrazar la supremacía en el firmamento. La democratización del FTL, un regalo inesperado de Louis Martin y el joven indio que había desentrañado sus secretos como un mago revela su truco, había allanado el camino, convirtiendo la carrera en un sprint donde el tiempo y la distancia se disolvían como niebla al sol. Alfa Centauri, el vecino estelar más cercano, se erigía como el premio, un viaje que antes devoraba milenios ahora reducido a una semana con cálculos que cabían en un ordenador doméstico. Pero la gloria verdadera, la inmortalidad grabada en el tapiz de la historia, aguardaba al primero que tocara sus orillas.
\section{Los Proyectos de las Potencias}
\subsection{Estados Unidos: El Híbrido}
El gobierno estadounidense, herido en su orgullo por la filtración que había desnudado sus secretos como un vendaval arrastra las hojas, reaccionó con la velocidad y arrogancia que lo definía, un toro enfurecido que carga contra la sombra de su propia humillación. Desplegaron un Super Heavy de SpaceX para lanzar una nave no tripulada, un coloso de metal que rugía hacia el cielo desde Cabo Cañaveral. No era exploración lo que buscaban, sino supremacía, una demostración de poder que eclipsara a los rivales. La tripulación —cuatro ingenieros y dos científicos, guardianes del conocimiento— viajaba en una cápsula Dragon adjunta, un arca de seguridad ante lo ignoto. Si el motor FTL fallaba, la cápsula se desprendería, regresando a la Tierra como un hijo pródigo; si triunfaba, se unirían al viaje. El plan era sencillo: llegar primero a Centauri, calcular el retorno con una supercomputadora que latía como un corazón mecánico, y regresar envueltos en la gloria de pioneros.
\subsection{China: El Fracaso del Dragón}
China, con la cicatriz de Shenzhen aún supurando como una herida abierta, se lanzó al abismo con la ferocidad de un dragón herido. Su nave, el \emph{Dragón Ascendente}, surcó el cielo impulsada por un cohete Larga Marcha, pero estalló en segundos, un fuego artificial trágico que iluminó su fracaso. Grabado por aficionados a los cohetes, el desastre se difundió como un susurro convertido en huracán, una humillación que Pekín intentó sepultar bajo capas de censura. El general que había apremiado el lanzamiento prematuro fue exiliado al olvido, reemplazado por otro que juró venganza estelar, prometiendo ser los primeros sin importar el costo en sangre o almas.
\subsection{Europa: El Sueño de Ícaro}
Europa, en su sutileza diplomática, tejía un tapiz distinto. \emph{El Sueño de Ícaro}, la nave de Stellarius Lux, ya medio forjada en el astillero de Hamburgo, se erguía como un fénix en gestación. Con el motor FTL de código abierto y los fondos de \emph{Eurociencia 2030} —un velo de subsidios encubiertos para “energía sostenible”— Joris y Louis completaron el sueño. La nave creció un 33\% más grande, con un anillo adicional para acunar a la tripulación y emisores duales que servían tanto para propulsión como para tejer agujeros de gusano. Un segundo condensador de Fluzo latía en su vientre, con un tercero de repuesto en la bodega, como reservas para un viaje eterno.

Pero mientras los ingenieros sudaban bajo el sol del Báltico, la política se colaba como una sombra. El Senado Europeo debatía con acaloradas pasiones quiénes serían los astronautas, un mosaico de nacionalidades que avivaba rivalidades. Cada senador reclamaba un hijo de su tierra para la gloria, un trozo del pastel estelar. Joris, ajeno al arte de la intriga, se sentía atrapado en el remolino, un ingeniero forzado a danzar en un ballet de egos.
\subsection{La Competencia Privada: El Futuro de los Negocios}
La carrera no se limitaba a naciones; las corporaciones, esos leviatanes del capital, se unían al frenesí con la astucia de mercaderes en un bazar cósmico. Google, Amazon y Nvidia forjaban sus propios FTL, vislumbrando mercados que eclipsaban imperios. Nvidia, en particular, soñaba con un vasto océano de ganancias: sus ingenieros diseñaban una tarjeta gráfica dedicada al FTL, capaz de calcular saltos de años luz en minutos. En su visión, pronto cada hogar tendría una, democratizando el espacio como un teléfono en cada bolsillo. El futuro del viaje interplanetario no yacía en manos de gobiernos, sino en las de corporaciones que tejían sueños con hilos de silicio.
\section{La Audacia China y el Fantasma del Pulso}
La nave estadounidense, \emph{Pionero}, se elevó con un rugido que estremeció la tierra desde Cabo Cañaveral, impulsada por la majestuosidad de un Super Heavy de SpaceX, un titán de fuego que devoraba el cielo. Sin tripulación, era un heraldo solitario; los astronautas —cuatro ingenieros y dos científicos— aguardaban en una cápsula Dragon adjunta, un capullo de seguridad ante el abismo. Su primer acto no era un salto FTL, sino una peregrinación de dos semanas por la trayectoria Hoffmann, alejándose de la garra gravitatoria de la Tierra. Solo en el vacío profundo, donde el espacio-tiempo de Riemann se extendía plano como un lienzo virgen, calcularían el salto único a Alfa Centauri.

Mientras \emph{Pionero} ascendía, en un búnker subterráneo del Tíbet, donde las montañas custodiaban secretos como guardianes eternos, el \emph{Dragón Ascendente} se preparaba para su desafío. La humillación del fracaso anterior ardía como brasas en el pecho del nuevo general, quien, en un acto de audacia desesperada, invocó una copia modificada del Motor Orión. Desoyeron advertencias y protocolos internacionales, como si el cosmos fuera un enemigo a doblegar. Su plan: detonaciones nucleares para catapultar la nave al punto de salto FTL en fracciones del tiempo americano, un atajo forjado en fuego atómico.

El lanzamiento fue un tapiz de terror y determinación: una serie de explosiones controladas, como relámpagos en cadena, impulsaron la nave hacia las estrellas. El cielo nocturno se tiñó de destellos antinaturales, una sinfonía de poder bruto y anhelo febril.

Pero el cosmos, caprichoso como un dios antiguo, no perdona la hybris. Lo que los chinos habían olvidado —o ignorado en su prisa— era una lección grabada en la memoria colectiva décadas atrás: el Test Starfish Prime. Las detonaciones nucleares en el espacio, eficaces para la propulsión, parían un hijo letal e invisible: un Pulso Electromagnético masivo, un fantasma que devoraba la luz de la civilización.

La primera explosión en la órbita baja de la Tierra liberó una onda invisible, un susurro mortal que se extendió como un velo de oscuridad. La costa este de China y vastas franjas del sudeste asiático se hundieron en el caos: redes eléctricas colapsaron como castillos de naipes, comunicaciones se evaporaron en silencio, y la infraestructura tecnológica —semáforos, ordenadores, corazones mecánicos de ciudades— se frió al instante. Las urbes se sumergieron en tinieblas, roto el velo solo por sirenas de emergencia y gritos de una población desorientada, como almas perdidas en un eclipse eterno.

A bordo del \emph{Dragón Ascendente}, la tripulación, acorazada contra la radiación como caballeros en su armadura, emergió ilesa. Pero en la sala de control de Pekín, los monitores se volvieron negros como la noche sin estrellas, la comunicación con la nave se disolvió en un vacío sepulcral, y el silencio respondió a los alaridos de alarma como un verdugo indiferente.

El caos en la Tierra fue un cataclismo instantáneo: mercados financieros se desplomaron como torres de Babel, bolsas asiáticas cerraron sus puertas en pánico, y el gobierno chino, enfrentado a una catástrofe en su propio seno, tejió narrativas desesperadas para contener el diluvio. Pero las noticias de la devastación y la interrupción inexplicable se propagaron como pólvora, avivadas por drones y cámaras de Anonymous, un coro digital que reía ante el tropiezo de los titanes.

En Washington D.C., el General Thompson, al enterarse del desastre, no mostró sorpresa, solo un leve movimiento de cabeza, como un sabio que anticipa la caída de un imperio. \begin{quote}
\calli
\glqq Monos,\grqq~ murmuró, su voz un eco de resignación.
\end{quote}
Pero la complejidad era mayor: el acto imprudente de China había desatado un conflicto no solo entre superpotencias, sino entre la humanidad y su propia hybris tecnológica. La carrera hacia las estrellas se había vuelto un sendero minado, donde cada paso podía abrir un abismo.

En Hamburgo, Joris de Vries recibió la noticia del EMP chino como un trueno lejano, mientras \emph{El Sueño de Ícaro}, con su segundo condensador de Fluzo ya latiendo en su vientre, se erguía majestuoso en el astillero, un coloso que desafiaba el horizonte. Joris, ajeno al pandemonio asiático, rumiaba el impulso no inercial, un sueño que parecía desvanecerse en la bruma de la política. Louis Martin, a su lado, garabateaba ecuaciones en el aire, sus dedos trazando arcos invisibles como un pintor del vacío. \begin{quote}
\calli
\glqq Joris,\grqq~ dijo, su voz un susurro que contenía el rumor de las estrellas, \glqq si usan un campo riemanniano irregular para esas explosiones, la energía se dispersa de forma anómala. Es el mismo error de enfoque, pero a escala titánica.\grqq
\end{quote}

Su rostro se iluminó con una idea nueva, como si el cosmos le hubiera confiado un secreto. \begin{quote}
\calli
\glqq Con precisión, Joris, podríamos domar ese pulso EMP: campos de energía para escudos de deflexión, o un amplificador para \emph{El Fanal} mismo. Una nueva aplicación, ¿no ves? El caos de ellos, nuestro puente.\grqq
\end{quote}
Joris lo miró, el mundo desmoronándose en llamas lejanas, y Louis solo veía ecuaciones danzando, un profeta que transformaba la ruina en renacimiento.
