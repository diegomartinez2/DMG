\chapter*{Der vierte Schwangerschaftsmonat}

\lettrine[lines=2, loversize=0.3, lraise=0]{\initfamily T}{he}

\chapter{Capítulo 7: La Entrevista y el Fantasma del Pasado}

En una sala de hotel en Los Ángeles, donde el sol del atardecer se filtraba a través de cortinas pesadas como velos de olvido, un hombre con un traje gris impecable y una mirada que parecía perforar las almas se inclinó hacia su interlocutor, su micrófono extendido como una lanza forjada en las fraguas de la intriga. No era un periodista de espectáculos, sino Julian Thorne, agente de la inteligencia británica, un cazador de sombras en un mundo de luces falsas. Su misión era desentrañar el enigma de Louis Martin, el fantasma que acechaba en los márgenes de la historia.

\begin{quote}
\calli
\glqq Gracias por tu tiempo, James,\grqq~ dijo Julian, con una sonrisa que se curvaba en sus labios pero no llegaba a sus ojos, como un río que fluye sin tocar la orilla. \glqq Tu papel en \emph{Odyssey Beyond} ha sido un triunfo, pero he oído que tu propia historia es tan cautivadora como el personaje que interpretas. ¿Es cierto que trabajaste en un instituto de ciencia en Europa?\grqq
\end{quote}

James, un actor de treinta y tantos años, con el cabello castaño rizado como ondas de un mar tormentoso y una sonrisa que iluminaba la habitación como un amanecer inesperado, soltó una risa que resonó como el eco de recuerdos lejanos. \glqq Sí, es verdad. Fue hace años, antes de que Hollywood me llamara con su sirena. Estudié ingeniería de telecomunicaciones en el Centro de Tecnología de Eindhoven. Compartía un despacho con dos ingenieros y dos físicos. Un verdadero crisol de mentes locas, donde las ideas chocaban como estrellas en colisión.\grqq

Julian se recostó en su silla, sus ojos brillando con un interés que parecía un fuego oculto bajo cenizas. \glqq Suena fascinante. ¿Quiénes eran esos compañeros tuyos?\grqq

\glqq Bueno, yo era el humilde ingeniero de comunicaciones, que apenas entendía la mitad de lo que discutían,\grqq~ rió James, su voz teñida de una nostalgia dulce como miel vieja. \glqq Y luego estaban los tres chiflados. El español, un físico llamado Miguel García Solís, era el mayor, un hombre tranquilo con una barba que parecía tejida de nubes y un acento que te hacía sentir en casa, como un abrazo de la tierra. Era la voz de la razón en nuestro caos. Luego estaba Joris de Vries,\grqq~ James se inclinó hacia Julian, bajando la voz como si compartiera un secreto ancestral, \glqq Joris era mi compañero, un tipo práctico, con los pies en la tierra. Un buen chico que ahora, ¿sabes?, es el CEO de una compañía. No me lo creo, pero me alegro por él. Y luego, estaba Louis.\grqq~ La voz de James se suavizó, como si tocara un recuerdo frágil. \glqq Louis. Era el genio, un tipo con ideas locas que parecían brotar de otro mundo. No socializaba mucho, hablaba con un acento raro, una mezcla de lenguas europeas que sonaba como un conjuro, y siempre garabateaba ecuaciones en el aire, como si invocara espíritus matemáticos. Era como un extraterrestre con un lápiz en la mano.\grqq

\glqq ¿Y cuál era la dinámica del grupo?\grqq~ preguntó Julian, su voz baja y uniforme, como el pulso de un reloj en la noche.

\glqq Era siempre la misma danza,\grqq~ respondió James, sus ojos perdidos en el pasado. \glqq Louis con su teoría, como si pudiera ver el universo en números y curvas invisibles. Joris, siempre práctico, intentaba forjar una máquina que capturara esa teoría, dándole forma tangible. Y Miguel, el viejo Miguel, se unía a la discusión con su experiencia, como un faro en la tormenta. Los dos físicos hablaban del futuro, de cómo la energía está en todas partes, solo hay que encontrar la manera de liberarla, como si desataran un río dormido en la tierra. Joris, en cambio, obsesionado con cómo esa energía podía impulsar una máquina, un sueño mecánico. Era un ballet intelectual que yo no entendía del todo. ¿Sabes lo que es no captar ni una palabra de lo que se discute en tu propio despacho? Es surrealista, como estar en un sueño ajeno.\grqq

\glqq ¿Y de qué hablaban, exactamente?\grqq~ insistió Julian, su tono un hilo de seda tensado.

\glqq De energía,\grqq~ dijo James, con un suspiro que parecía cargar el peso de aquellos días. \glqq El instituto se centraba en la energía. Querían revolucionar cómo el mundo la usa, liberarla de sus cadenas. Era una locura hermosa. \emph{`La energía está en todas partes, solo tenemos que encontrar la manera de liberarla,'} decían. No, no hablaban de naves espaciales; se enfocaban en la energía pura, el latido del cosmos que el instituto buscaba domesticar, ¿sabes?\grqq

Julian sonrió, un gesto que ocultaba el hambre de su curiosidad. \glqq Ya veo. ¿Y cómo se forjó esa amistad entre ustedes?\grqq

\glqq Joris y yo éramos los únicos `normales' en el despacho,\grqq~ rió James, su risa un bálsamo efímero. \glqq Nos unimos en la frustración, como náufragos en una isla de ecuaciones. Pero la verdadera amistad nació cuando nos sumergimos con los `chiflados'. Miguel era un padre para todos nosotros. Nos invitaba a su casa a cenar, nos daba consejos que sabían a sabiduría antigua, y siempre tenía la respuesta correcta, como un oráculo disfrazado de profesor. Nos cuidaba como un padre vela por sus hijos.\grqq

La sonrisa de James se desvaneció entonces, como una nube que cubre el sol. \glqq Fue hace cinco años, en Navidad. Miguel estaba de compras para su hija, que cumplía diecisiete. Cruzaba la calle, cargado de paquetes como un santo laico, cuando un coche lo atropelló. Murió al instante, en un instante que rompió algo en nosotros. Fue una tragedia que nos unió de una forma que nada más podría. Después de eso, Joris se fue, llevando consigo el fuego de nuestro trabajo. Y Louis\ldots~ Louis simplemente desapareció, como un fantasma que se disuelve en la niebla. Fue terrible, un vacío que aún duele.\grqq

Julian asintió, su rostro inexpresivo como una máscara de porcelana. \glqq Ya veo. ¿Y qué hizo Joris después?\grqq

\glqq Joris, el práctico, se marchó,\grqq~ dijo James, su voz teñida de admiración. \glqq Pero se llevó el legado: el genio de Louis, la experiencia de Miguel, y su propio pragmatismo para fundar Stellarius Lux, hace tres años. Un plan para construir un motor FTL, una locura que desafía el cielo. Pero Joris era el único que podía hacerla realidad, tejiendo sueños en metal.\grqq

Julian se levantó, su voz suave y uniforme como el rumor de un río subterráneo. \glqq Gracias, James. Ha sido muy informativo.\grqq~ Se despidió con un apretón de manos que parecía sellar un pacto invisible y se alejó por el pasillo del hotel, donde las sombras se alargaban como dedos acusadores. Al reunirse con un colega en el lobby, este le preguntó si había obtenido algo útil. Julian asintió, una sonrisa sutil curvando sus labios. \glqq Sí. El motor FTL no fue obra de un solo hombre. Fue un tapiz tejido por tres almas: dos ingenieros y dos físicos, una sinergia entre teoría loca, pragmatismo práctico y una voz de la razón. Y la clave está en los `chiflados' que nadie ve: el genio que se esconde tras el plan, un fantasma sin pasaporte.\grqq~ Julian sonrió más ampliamente. \glqq El fantasma. Ahora sabemos quién es.\grqq

Fuera del hotel, una paloma blanca posada en una rama cercana ladeó la cabeza, como si escuchara los secretos desenterrados, y extendió sus alas, lista para volar hacia horizontes invisibles.
