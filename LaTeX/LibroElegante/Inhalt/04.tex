\part*{Die Schwangerschaft}

% % % % % % % % % % % % % % % % % % % % % % % % % % % % % % % % %
% % % % % % % % % % % % % % % % % % % % % % % % % % % % % % % % %
% % % % % % % % % % % % % % % % % % % % % % % % % % % % % % % % %
% % % % % % % % % % % % % % % % % % % % % % % % % % % % % % % % %
% % % % % % % % % % % % % % % % % % % % % % % % % % % % % % % % %
% % % % % % % % % % % % % % % % % % % % % % % % % % % % % % % % %
% % % % % % % % % % % % % % % % % % % % % % % % % % % % % % % % %
% % % % % % % % % % % % % % % % % % % % % % % % % % % % % % % % %
% % % % % % % % % % % % % % % % % % % % % % % % % % % % % % % % %
% % % % % % % % % % % % % % % % % % % % % % % % % % % % % % % % %
% % % % % % % % % % % % % % % % % % % % % % % % % % % % % % % % %
% % % % % % % % % % % % % % % % % % % % % % % % % % % % % %

\chapter*{Der erste Schwangerschaftsmonat}

\lettrine[lines=2, loversize=0.3, lraise=0]{\initfamily A}{uch}

\chapter{La Diplomacia Secreta y la Sombra de Langley}

\lettrine[lines=2, loversize=0.3, lraise=0]{E}{n} las entrañas del Departamento de Justicia de Estados Unidos, donde los relojes marcaban el tiempo como verdugos inexorables, el bloqueo de los fondos de Stellarius Lux se confirmó en menos de veinticuatro horas, un susurro burocrático que se extendió como niebla sobre un lago. La justificación oficial era un velo tenue: \glqq cumplimiento de la Ley de Secreto Bancario y revisión de actividades de exportación.\grqq~ Pero en las alturas del poder, donde las decisiones se tejían con hilos de sombra, todos conocían la verdad. La cuenta de Joris de Vries era una carnada reluciente, lanzada al mar profundo para atrapar al pez chino, que ya había mordido con avidez.

En un salón privado de un club elitista en Washington D.C., donde el humo de cigarros antiguos flotaba como fantasmas de acuerdos olvidados, un hombre ataviado con un traje que parecía cosido con precisión quirúrgica ---el subdirector de la CIA--- se sentó frente a un empresario europeo. Su nombre en clave era \glqq Señor Smith\grqq, un alias tan anodino como el aire que respiraban.

\begin{quote}
\calligra
\glqq Señor Smith,\grqq~ inició el subdirector, con una sonrisa que se curvaba en sus labios pero moría antes de alcanzar sus ojos, fríos como el acero de una daga. \glqq Su empresa enfrenta un dilema. Y a mi gobierno no le agrada que compañías extranjeras utilicen nuestro sistema bancario para financiar proyectos que podrían amenazar nuestra seguridad nacional.\grqq
\end{quote}

El \glqq Señor Smith\grqq, emisario secreto de Joris de Vries, respondió con la firmeza de un roble centenario. \glqq Esos fondos nos pertenecen. Están en un banco estadounidense porque la compañía se forjó allí. El gobierno no puede arrebatar el dinero de una empresa sin una causa justificada.\grqq

El subdirector soltó una risa hueca, como el eco de un trueno lejano en una noche sin estrellas. \glqq Causa, la tenemos. Una de nuestras invenciones. La tecnología FTL es un quiebre en el tapiz del mundo: la próxima revolución en el transporte, la guerra, la economía. No podemos permitir que caiga en manos equivocadas. Por ejemplo, en las de una potencia extranjera que ya ha extendido sus tentáculos hacia ustedes, Señor Smith, ofreciendo `ayuda'.\grqq

El rostro del emisario palideció, como si la sangre se hubiera retirado a las profundidades de su ser. El gobierno estadounidense ya sabía del contacto con China. No había escapatoria; se movían en órbitas distintas, en un baile cósmico donde uno era el sol y el otro, un mero satélite.

\begin{quote}
\calligra
\glqq ¿Qué desean?\grqq~ preguntó Smith, su voz un hilo fino, tenso como una cuerda de violín a punto de romperse.
\end{quote}

\glqq Una asociación,\grqq~ replicó el subdirector, su tono suave como seda pero afilado como una hoja. \glqq Una verdadera. El gobierno de Estados Unidos descongelará los fondos y los multiplicará, proporcionará ingenieros, laboratorios, seguridad. Todo lo que necesiten, siempre y cuando el proyecto se traslade a nuestro suelo. Y, por supuesto, que Louis Martin se convierta en consultor exclusivo del gobierno.\grqq

El emisario sabía que Joris jamás aceptaría tal condición. Louis era un genio anárquico, libre como el viento que recorre los campos sin ataduras, un alma que aborrecía los protocolos y las cadenas de las reglas.

\begin{quote}
\calligra
\glqq Señor, dudo que el Señor Martin acceda,\grqq~ dijo Smith, su voz un murmullo de resignación.
\end{quote}

El subdirector se incorporó, su voz enfriándose como el aliento del invierno. \glqq No tiene elección. Dígale a Joris que, si rechazan, su gobierno lo considerará un activo enemigo. Invocaremos la Ley de Seguridad Nacional, que permite perseguir a cualquier ciudadano o empresa que trate con un país adversario. Dígale que podemos hacer su vida un infierno, tanto a él como a Louis, cerrando puertas en todos los bancos del mundo, vigilando cada paso como sombras inseparables.\grqq

Con esa amenaza, flotando en el aire como un presagio oscuro, el gobierno estadounidense lanzó su jugada: una oferta que Joris no podía rechazar, pero que tampoco podía abrazar sin perder su alma.

\section{La Sombra de Louis}

Mientras tanto, a miles de kilómetros de distancia, ajeno al torbellino de intrigas que giraba en torno a su nombre, Louis Martin se hallaba sentado en un parque de Luxemburgo, donde los árboles susurraban secretos al viento y las hojas caídas formaban tapices de oro y carmín. Estaba frustrado, su teléfono ---un relicto antiguo, tosco como una piedra tallada--- se negaba a conectar con el Wi-Fi público, un velo invisible que lo separaba de enviar sus cálculos a Joris.

Justo entonces, una joven cercana, absorta en la pantalla de su propio dispositivo como si contuviera los misterios del universo, tropezó y cayó al suelo con un sonido sordo, como el de una fruta madura desprendiéndose de la rama. Louis, impulsado por una bondad instintiva, corrió a ayudarla. Al agacharse, el teléfono de la chica, dañado por el impacto, se conectó milagrosamente a la red, como si el destino hubiera tejido un hilo invisible entre sus aparatos.

El brillo en los ojos de Louis, al ver la conexión establecida, podía confundirse con el de la joven que le sonreía en agradecimiento, un destello de luz en medio de la penumbra cotidiana. Al levantarse, una ráfaga de viento surgió de la nada, provocada por un camión de basura que pasaba con el rugido de un dragón adormecido, y esa ráfaga impidió que una cámara de vigilancia capturara el momento, como si el aire mismo conspirara para preservar su invisibilidad. Segundos después, el teléfono de la chica se desconectó nuevamente, dejando solo el eco de un milagro efímero.

Louis, con una sonrisa que parecía nacer de las profundidades de su ser, regresó a su banco, ignorante de que, en ese instante fugaz, había esquivado una vez más las redes del mundo. El parque, con su quietud eterna, parecía guardar su secreto, un refugio donde los genios como él podían soñar sin ser vistos.

\section{El Susurro del Astillero}

En el astillero del norte de Alemania, donde el viento del Báltico cantaba una elegía salada, las grúas se alzaban como gigantes de hierro que custodiaban un secreto más antiguo que el tiempo mismo. Bajo la luz gris del amanecer, los trabajadores, con rostros curtidos por el frío y el esfuerzo, soldaban placas de una aleación desconocida, un metal que parecía beber la luz en lugar de reflejarla. Nadie sabía para qué servía, pero en sus corazones, endurecidos por años de labor, sentían que estaban forjando algo que cambiaría el destino de los hombres.

El astillero, un laberinto de acero y sal, había sido sellado al mundo exterior. Inspectores europeos, con pretextos de contaminación marina, patrullaban los muelles, mientras agentes encubiertos de potencias lejanas merodeaban en las sombras, disfrazados de pescadores o turistas despistados. Pero el secreto de \emph{El Fanal} permanecía intacto, custodiado no por hombres armados, sino por el silencio de quienes trabajaban bajo su hechizo.

En una oficina improvisada, entre planos y tazas de café que olían a noches sin fin, Joris de Vries se paseaba como un lobo enjaulado. Frente a él, Louis Martin dibujaba en un cuaderno, su lápiz trazando líneas que parecían más conjuros que ecuaciones.

\begin{quote}
\calligra
\glqq Joris,\grqq~ dijo Louis, sin alzar la vista, su voz suave como el murmullo de un arroyo, \glqq he recalculado las tolerancias del campo de deformación. Si añadimos un estabilizador cuádruple, la nave podría soportar una transición más allá de Urano. Tal vez hasta Alfa Centauri.\grqq
\end{quote}

Joris se detuvo, su rostro pálido como la niebla que cubría el astillero. \glqq Louis, no estamos listos para Alfa Centauri. Apenas podemos mantener a los americanos y a los chinos fuera de aquí. Ayer interceptaron un dron sobre el Báltico. ¡Un dron, Louis! Sabemos que era chino, pero no podemos probarlo.\grqq

Louis alzó la mirada, sus ojos brillando con una calma que parecía desafiar las leyes del tiempo. \glqq Que busquen. No encontrarán nada. \emph{El Fanal} no es una máquina que se pueda robar. Es un sueño que aún no han aprendido a soñar.\grqq

En ese momento, una paloma blanca, idéntica a la que había cruzado los cielos de Berlín y Luxemburgo, se posó en el alféizar de la ventana. Sus alas, como lienzos de un pintor celestial, reflejaban el amanecer con un resplandor que parecía prometer milagros. Joris la miró, y por un instante, sintió que el universo entero conspiraba para protegerlos.

\section{La Danza de las Sombras}

Lejos del astillero, en un café de Copenhague, una agente de la CIA conocida solo como \glqq Clara\grqq~ sorbía un té con la precisión de un reloj. Frente a ella, un informante danés, un hombre de rostro arrugado como un mapa antiguo, deslizó un sobre bajo la mesa.

\begin{quote}
\calligra
\glqq Es todo lo que tengo,\grqq~ susurró, su voz temblando como las olas del Báltico. \glqq Planos parciales del astillero. Pero no hay nada sobre el motor. Solo\ldots rumores. Dicen que el hombre detrás de todo, ese Louis, no aparece en ninguna cámara. Es como un fantasma.\grqq
\end{quote}

Clara abrió el sobre, sus dedos fríos rozando el papel. Los planos mostraban una estructura colosal, una nave que no se parecía a nada que la NASA hubiera soñado. Pero los detalles del motor, del corazón de \emph{El Fanal}, eran un vacío, un espacio en blanco que parecía burlarse de ella. \glqq ¿Y este Louis? ¿Alguien lo ha visto?\grqq~ preguntó, su voz baja pero afilada como una navaja.

El informante negó con la cabeza. \glqq Nadie. Hay quien dice que no existe, que es un nombre inventado para despistar. Pero yo vi algo\ldots en el astillero. Un hombre, delgado, con el cabello desordenado. Iba a tomar una foto, pero una paloma, maldita sea, una paloma blanca voló justo frente a mi lente. Cuando miré de nuevo, él ya no estaba.\grqq

Clara frunció el ceño, su mente girando como las hélices de un barco perdido. La paloma. Siempre la paloma. En los informes de Berlín, de Luxemburgo, de todas partes, esa criatura alada aparecía como un guardián improbable, un emisario del destino que protegía al hombre sin rostro.

En Pekín, la frustración era un río que corría bajo la superficie. Un general del Ministerio de Seguridad del Estado, con el rostro endurecido por años de secretos, revisaba un video granulado. Mostraba el astillero desde un satélite, pero en el momento crucial, cuando Louis Martin salía de un hangar, una nube pasajera oscureció la imagen, como si el cielo mismo hubiera cerrado los ojos. \glqq Otra vez,\grqq~ gruñó el general. \glqq Siempre algo. Una paloma, una nube, un reflejo. ¡Encuentren a ese hombre!\grqq

\section{El Susurro del Metal}

De vuelta en el astillero, el trabajo continuaba bajo la vigilancia de las grúas, que se movían como titanes en un vals lento. Los trabajadores, sin saberlo, eran parte de un milagro. Cada soldadura, cada placa, era un paso hacia un futuro que ninguno podía imaginar. Pero en el corazón del astillero, en un hangar sellado donde la luz apenas se atrevía a entrar, \emph{El Fanal} comenzaba a tomar forma.

Louis, solo en el hangar, tocó el metal frío de la estructura, y por un instante, sintió que vibraba con una energía que no era de este mundo. \glqq Pronto,\grqq~ susurró, como si hablara con la nave misma. \glqq Pronto cruzarás las estrellas.\grqq

La paloma, posada ahora en una viga del hangar, ladeó la cabeza, sus ojos brillando como dos perlas negras. Y en el silencio del astillero, bajo el canto del viento y el rugido lejano del mar, el universo parecía contener el aliento, esperando el momento en que \emph{El Fanal} iluminaría los confines del cosmos.
