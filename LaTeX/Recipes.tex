
\documentclass[10pt,a4paper]{article}
\usepackage[T1]{fontenc}
\usepackage[sc,osf]{mathpazo}
\usepackage[english]{babel}
\usepackage[utf8]{inputenc}
\usepackage{amsmath}
\usepackage[margin=1in]{geometry}
\usepackage{graphicx}
\usepackage{float}
\usepackage[normalem]{ulem}
\usepackage{gb4e}\noautomath
\usepackage{setspace}
\usepackage{array,multirow,graphicx}
\usepackage[table]{xcolor}
\usepackage{tabularx}
\newcolumntype{L}[1]{>{\raggedright\let\newline\\\arraybackslash\hspace{0pt}}m{#1}}
\newcolumntype{C}[1]{>{\centering\let\newline\\\arraybackslash\hspace{0pt}}m{#1}}
\newcolumntype{R}[1]{>{\raggedleft\let\newline\\\arraybackslash\hspace{0pt}}m{#1}}
\usepackage{mdwlist}
\usepackage{csquotes}
\usepackage{tipa}
\usepackage{tree-dvips}
\usepackage{qtree}
\usepackage{tikz-qtree}
\usepackage{xfrac}
\usepackage{gensymb}
\usepackage{adjustbox}
\usepackage{enumerate}
\usepackage{unicode-math}
%\linespread{2}
%\usepackage{setspace}\onehalfspacing
\setlength{\parindent}{0in}

\title{Recipes}
%\author{Shayne Sloggett }
\date{Last Updated: \today}

\setcounter{secnumdepth}{0}

\begin{document}

\maketitle
\tableofcontents
\newpage
\section{China}

\subsection{Tomato \textit{\&} Egg Stir-fry}
\textit{Maggie Zhu, Omnivore's Cookbook}

\rule{\textwidth}{0.4pt}
\adjustbox{valign=t}{\begin{minipage}{0.3\linewidth}
\begin{enumerate}[]
    \item $1\sfrac{1}{2}$ tbsp oil, divided
    \item $4$ eggs, beaten
    \item $2$ tomatoes, roughly chopped
    \item $5$ cloves garlic, chopped
    \item $2$ tsp sugar
    \item $\sfrac{1}{2}$ tsp salt
    \item 1 green onion, sliced
\end{enumerate}
\end{minipage}}
\adjustbox{valign=t}{\begin{minipage}{0.03\linewidth}
\hfill
\end{minipage}}
\adjustbox{valign=t}{\begin{minipage}{0.66\linewidth}
    \begin{exe}
        \ex Heat $1$ tablespoon oil over medium high heat
        \ex Cook egg until side is done, but top still raw ($\sim$30 sec)
        \ex Using spatula, chop egg into bite sized pieces
        \ex Transfer egg to a plate and set aside
        \item Saute garlic over medium heat in $\sfrac{1}{2}$ tbsp oil until fragrant
        \item Add tomato and stir fry until slightly soft and charred
        \item Add egg back into skillet and season with salt and sugar
        \item Quickly mix together until evenly seasoned
    \end{exe}
\end{minipage}}
\vskip 0.2in
\rule{\textwidth}{0.4pt}
\begin{enumerate}[]
    \item \textbf{Notes:}
    \begin{enumerate}[-]
        \item In place of garlic, substitute green onion for a sweeter dish ($\sfrac{1}{2}$ cup)
        \item For soft egg, mix $\sfrac{1}{4}$ teaspoon of salt into beaten eggs; stir egg as soon as put into skillet, and transfer when half cooked; add egg back into the skillet after seasoning tomato
        \item Makes 2-4 servings
    \end{enumerate}
\end{enumerate}
\rule{\textwidth}{0.4pt}

\vfill

\subsection{Chinese Cucumber Salad}
\textit{Maggie Zhu, Omnivore's Cookbook}

\rule{\textwidth}{0.4pt}
\adjustbox{valign=t}{\begin{minipage}{0.3\linewidth}
\begin{enumerate}[]
    \item $1$ English cucumber
    \item $3$ cloves garlic
    \item $1$ tbsp chinkiang vinegar
    \item $1$ tbsp light soy sauce
    \item $1$ tsp sugar
    \item $\sfrac{1}{2}$ tsp salt
    \item $\sfrac{1}{2}$ tsp sesame oil
    \item $2$ tsp chili oil
    \item $1$ tbsp Lao Gan Ma
\end{enumerate}
\end{minipage}}
\adjustbox{valign=t}{\begin{minipage}{0.03\linewidth}
\hfill
\end{minipage}}
\adjustbox{valign=t}{\begin{minipage}{0.66\linewidth}
\setcounter{exx}{0}
\begin{exe}
    \ex Crush cucumber and cut into bite size pieces
    \ex Mince garlic and add to cucumber
    \ex Mix vinegar, soy sauce, sugar, salt, and sesame oil and toss with cucumber and garlic
    \ex Drizzle with chili oil and top with Lao Gan Ma
\end{exe}
\rule{\textwidth}{0.4pt}
\begin{enumerate}[]
    \item \textbf{Notes:}
    \begin{enumerate}[-]
        \item Dress the cucumber and garlic immediately before serving, or salad will be watery
        \item Makes 2-4 servings
    \end{enumerate}
\end{enumerate}
\end{minipage}}
\vskip 0.025in
\rule{\textwidth}{0.4pt}
\vfill
\newpage

\subsection{Sichuan Mapo Tofu}
\textit{Chris \& Steph, Chinese Cooking Demystified}

\rule{\textwidth}{0.4pt}
\adjustbox{valign=t}{\begin{minipage}{0.3\linewidth}
\begin{enumerate}[]
    \item $500$g soft tofu, cubed
    \item $80$g beef, minced
    \item $5$ tbsp peanut oil
    \item $1$ tbsp Sichuan peppercorn
    \item $3$ tbsp doubanjiang
    \item $2$ tbsp chilli flake
    \item $1$ tbsp douchi
    \item $4$ cloves garlic, minced
    \item $2$ inches ginger, minced
    \item $1$ cup stock (or water)
    \item $2$ scallions, roughly chopped
    \item \textbf{Seasoning:}
    \begin{enumerate}[]
        \item $1$ tbsp light soy sauce
        \item $1$ tbsp Shaoxing wine
        \item $\sfrac{1}{4}$ tsp MSG
        \item $\sfrac{1}{4}$ tsp white pepper powder
        \item $\sfrac{1}{2}$ tbsp sesame oil
    \end{enumerate}
\end{enumerate}
\end{minipage}}
\adjustbox{valign=t}{\begin{minipage}{0.03\linewidth}
\hfill
\end{minipage}}
\adjustbox{valign=t}{\begin{minipage}{0.66\linewidth}
\setcounter{exx}{0}
\begin{exe}
    \ex Simmer tofu in salted water for 3 minutes. Remove from heat, and let sit until ready for use.
    \ex In a dry wok, toast Sichuan peppercorn over low heat for 2 minutes. Transfer to a mortar and grind.
    \ex Heat wok on high. Add oil and stir fry beef over medium until crisp ($\sim$4 minutes).
    \ex Remove wok from heat and add doubanjiang. Cook over medium-low for 90 seconds.
    \ex Add douchi, ginger, garlic, and chili flake ($\sim$30 sec each), and fry until an even paste forms.
    \ex Add stock and drained tofu. Cook on medium-high until reduced by $\sfrac{1}{3}$ ($\sim$7 min).
    \ex Add seasoning and stir. Then add slurry (1 tbsp cornstarch/2 tbsp water) slowly, to prevent over thickening.
    \ex Add scallions and ground Sichuan peppercorn and serve.
\end{exe}
\rule{\textwidth}{0.4pt}
\begin{enumerate} []
    \item \textbf{Notes:}
        \begin{enumerate} [-]
        \item Any broth will work, and you can use water if necessary
        \item This dish is quite spicy: cut back on chilli flake as necessary
        \item Makes 2-4 servings
    \end{enumerate}
\end{enumerate}
\end{minipage}}
\vskip .025in
\rule{\textwidth}{0.4pt}
\vfill

\subsection{Sichuan Ganbian Sijidou (Dry-fried Stringbeans)}
\textit{Chris \& Steph, Chinese Cooking Demystified}

\rule{\textwidth}{0.4pt}
\adjustbox{valign=t}{\begin{minipage}{0.3\linewidth}
\begin{enumerate}[]
    \item $300$g string beans, trimmed
    \item $100$g pork belly, minced
    \item $10$g dried chilli, cut 1in pieces
    \item $40$g Sichuan yacai
    \item $1$ tbsp Sichuan peppercorn
    \item $1$ clove garlic, sliced
    \item $1$ inch ginger, sliced
    \item $2$ green onions, chopped
    \item $\sfrac{1}{2}$ tbsp Chinese vinegar
    \item \textbf{Seasoning:}
    \begin{enumerate}[]
        \item $\sfrac{1}{4}$ tsp salt
        \item $\sfrac{1}{4}$ tsp sugar
        \item $\sfrac{1}{8}$ tsp MSG
    \end{enumerate}
\end{enumerate}
\end{minipage}}
\adjustbox{valign=t}{\begin{minipage}{0.03\linewidth}
\hfill
\end{minipage}}
\adjustbox{valign=t}{\begin{minipage}{0.66\linewidth}
\setcounter{exx}{0}
\begin{exe}
    \ex Fry string beans in 3tbsp oil over medium low, until wrinkly/blistered ($\sim$8min). Transfer to a towel-lined plate.
    \ex Fry pork over medium heat until fat is rendered and pork is crispy. Remove pork and reserve.
    \ex Over low heat, fry Sichuan peppercorn, chillies, garlic/ginger, and yacai ($\sim$30sec each)
    \ex Increase heat to medium and add the string beans back
    \ex Add seasoning, pork, and green onion, mixing between
    \ex Finish with vinegar and cook until evaporated
\end{exe}
\rule{\textwidth}{0.4pt}
\begin{enumerate}[]
    \item \textbf{Notes:}
    \begin{enumerate}[-]
        \item Still figuring out the heat on this one. Varies widely by stovetop.
        \item If yacai is too salty, soak in water and pat dry before using
        \item Best served immediately
        \item Makes 2-4 servings
    \end{enumerate}
\end{enumerate}
\end{minipage}}
\vskip .025in
\rule{\textwidth}{0.4pt}
\vfill
\newpage

\vfill

\subsection{Hand-Cut Noodles}
\textit{Mandy, Souped up Recipes}

\rule{\textwidth}{0.4pt}
\adjustbox{valign=t}{\begin{minipage}{0.3\linewidth}
\begin{enumerate}[]
    \item $312$ g bread flour
    \item $156$ g eggs and water
    \item $\sfrac{1}{2}$ tsp salt
\end{enumerate}
\end{minipage}}
\adjustbox{valign=t}{\begin{minipage}{0.03\linewidth}
\hfill
\end{minipage}}
\adjustbox{valign=t}{\begin{minipage}{0.66\linewidth}
\setcounter{exx}{0}
\begin{exe}
    \ex Whisk together flour and salt. Beat eggs and water and stir into flour. Knead into a smooth dough and let rest 30 min.
    \ex Knead dough until smooth and shiny. Roll out 1.5mm thick, dust both sides with flour, and loosely concertina-fold.
    \ex Cut to desired thickness and toss to remove excess flour.
\end{exe}
\end{minipage}}
\vskip 0.025in
\rule{\textwidth}{0.4pt}
\begin{enumerate}[]
    \item \textbf{Notes:}
    \begin{enumerate}[-]
        \item Makes 2-4 servings
        \item Dough saves up to two days
        \item Cut noodles save up to 1 hour
    \end{enumerate}
\end{enumerate}
\vskip -0.15in
\rule{\textwidth}{0.4pt}

\vfill

\subsection{Biangbiang Noodles}
\textit{S.J.S.}

\rule{\textwidth}{0.4pt}
\adjustbox{valign=t}{\begin{minipage}{0.3\linewidth}
\begin{enumerate}[]
    \item $1$ tbsp chinkiang vinegar
    \item $1$ tsp light soy sauce
    \item $1$ tsp dark soy sauce
    \item $1$ tsp ginger, minced
    \item $1$ tsp garlic, minced
    \item $1$ tsp red chili flake
    \item $\sfrac{1}{4}$ tsp Sichuan peppercorn
    \item $2$ green onion, sliced
    \item $2$ tbsp cilantro, chopped
    \item $2\sfrac{1}{2}$ tbsp vegetable oil
    \item $1$ batch hand-cut noodles, sliced 1 inch wide
\end{enumerate}
\end{minipage}}
\adjustbox{valign=t}{\begin{minipage}{0.03\linewidth}
\hfill
\end{minipage}}
\adjustbox{valign=t}{\begin{minipage}{0.66\linewidth}
\setcounter{exx}{0}
\begin{exe}
    \ex Combine light soy sauce, dark soy sauce, and vinegar.
    \ex Cook noodles until just al dente ($\sim$2-3 minutes), and transfer to serving bowl.
    \ex Pour sauce over noodles and top with ginger, garlic, chili flake, Sichuan peppercorn, green onion, and cilantro.
    \ex Heat oil until just smoking and pour over aromatics. Mix noodles and eat immediately.
\end{exe}
\rule{\textwidth}{0.4pt}
\begin{enumerate}[]
    \item \textbf{Notes:}
    \begin{enumerate}[-]
        \item Makes 2-4 servings
        \item Best served immediately
        \item This is a modification of a recipe from Mandy (Souped up Recipes). Traditionally, noodles would be pulled, not cut.
    \end{enumerate}
\end{enumerate}
\end{minipage}}
\vskip .025in
\rule{\textwidth}{0.4pt}

\vfill

\newpage
\section{Korea}

\subsection{Crunchy Gochujang Fennel}
\textit{Sohui Kim, Bon Appétit (March 2019)}

\rule{\textwidth}{0.4pt}
\adjustbox{valign=t}{\begin{minipage}{0.3\linewidth}
\begin{enumerate}[]
    \item $1$ fennel bulb, stems removed
    \item $1\sfrac{1}{2}$ inch ginger, minced
    \item $1$ clove garlic, minced
    \item $2$ tbsp gochujang
    \item $1\sfrac{1}{2}$ tbsp rice vinegar
    \item $1$ tbsp honey
    \item $1$ tbsp sesame oil
    \item $1\sfrac{1}{2}$ tsp gochugaru
    \item $1\sfrac{1}{2}$ tsp toasted sesame seeds
\end{enumerate}
\end{minipage}}
\adjustbox{valign=t}{\begin{minipage}{0.03\linewidth}
\hfill
\end{minipage}}
\adjustbox{valign=t}{\begin{minipage}{0.66\linewidth}
\setcounter{exx}{0}
\begin{exe}
    \ex Slice fennel bulb in quarters through the core. Remove the core and slice lengthwise into $\sfrac{1}{4}$ inch strips
    \ex Boil fennel in salted water until slightly softened and translucent ($\sim$2 minutes). Chill in ice bath then pat dry.
    \ex Whisk together garlic, ginger, gochujang, vinegar, honey, oil, gochugaru, and sesame seeds in a medium bowl
    \item Add fennel and toss to coat. Season with salt, as necessary
\end{exe}
\rule{\textwidth}{0.4pt}
\begin{enumerate}[]
    \item \textbf{Notes:}
    \begin{enumerate}[-]
        \item Fennel can be made 4 days ahead. Cover and chill.
        \item Makes 4 servings
    \end{enumerate}
\end{enumerate}
\end{minipage}}
\vskip .025in
\rule{\textwidth}{0.4pt}

\subsection{Korean Fried Rice}
\textit{Maangchi}

\rule{\textwidth}{0.4pt}
\adjustbox{valign=t}{\begin{minipage}{0.3\linewidth}
\begin{enumerate}[]
    \item $3$ cups cooked rice
    \item $2$ cloves garlic
    \item $\sfrac{1}{4}$ cup onion, chopped
    \item $5$ oz beef, diced
    \item $\sfrac{1}{3}$ cup kimchi, chopped
    \item $3$ cups vegetables, diced
    \item $2$ hot green peppers, chopped
    \item $2$ green onions, chopped
    \item $2$ tsp sesame oil
    \item $2$ tbsp oyster sauce
    \item $\sfrac{1}{4}$ tsp black pepper
    \item $1$ egg/person, sunny side up
\end{enumerate}
\end{minipage}}
\adjustbox{valign=t}{\begin{minipage}{0.03\linewidth}
\hfill
\end{minipage}}
\adjustbox{valign=t}{\begin{minipage}{0.66\linewidth}
\setcounter{exx}{0}
\begin{exe}
    \ex Cook garlic and onion in 2tbsp vegetable oil over medium heat
    \ex Add beef and stir fry 2-3 minutes, until beef is browned
    \ex Add kimchi and stir fry 2 minut es
    \ex Add vegetables and stir fry  3 minutes
    \ex Add rice and cook, stirring constantly, 3-5 minutes
    \ex Add oyster sauce and cook, stirring constantly, 2 minutes
    \ex Remove from heat, add sesame oil, green onion, and pepper
    \ex Transfer to plate and top with fried egg
\end{exe}
\rule{\textwidth}{0.4pt}
\begin{enumerate}[]
    \item \textbf{Notes:}
    \begin{enumerate}[-]
        \item Optionally top with mild cheese (mozzarella, Oaxacan)
        \item Makes 4 servings
    \end{enumerate}
\end{enumerate}
\end{minipage}}
\vskip 0.025in
\rule{\textwidth}{0.4pt}

\subsection{Anchovy Stock}
\textit{Maangchi}

\rule{\textwidth}{0.4pt}
\adjustbox{valign=t}{\begin{minipage}{0.3\linewidth}
\begin{enumerate}[]
    \item $8$ dried anchovies, heads and guts removed
    \item $\sfrac{1}{3}$ cup daikon, thinly sliced
    \item $4\times 6$ inches dried kelp
    \item $3$ scallion roots
\end{enumerate}
\end{minipage}}
\adjustbox{valign=t}{\begin{minipage}{0.03\linewidth}
\hfill
\end{minipage}}
\adjustbox{valign=t}{\begin{minipage}{0.66\linewidth}
\setcounter{exx}{0}
\begin{exe}
    \ex boil ingredients in 4 cups water on medium-high for 20 min.
    \ex Reduce heat to low, simmer for another 5 minutes and strain
\end{exe}
\rule{\textwidth}{0.4pt}
\begin{enumerate}[]
    \item \textbf{Notes:}
    \begin{enumerate}[-]
        \item Makes $\sim2\sfrac{1}{2}$ cups of broth
    \end{enumerate}
\end{enumerate}
\end{minipage}}
\vskip 0.025 in
\rule{\textwidth}{0.4pt}
\newpage

\subsection{Kimchi Jjigae}
\textit{Maangchi}

\rule{\textwidth}{0.4pt}
\adjustbox{valign=t}{\begin{minipage}{0.3\linewidth}
\begin{enumerate}[]
    \item $1$ lb kimchi, roughly chopped
    \item $\sfrac{1}{4}$cup kimchi brine
    \item $\sfrac{1}{2}$ lb pork shoulder (or belly)
    \item $250$ g tofu, sliced ($\sfrac{1}{2}$ inch)
    \item $3$ scallions, chopped
    \item $1$ medium onion, sliced
    \item $1$ tsp salt
    \item $2$ tsp sugar
    \item $2$ tsp gochugaru
    \item $1$ tbsp gochujang
    \item $1$ tsp sesame oil
    \item $2$ cups anchovy stock
\end{enumerate}
\end{minipage}}
\adjustbox{valign=t}{\begin{minipage}{0.03\linewidth}
\hfill
\end{minipage}}
\adjustbox{valign=t}{\begin{minipage}{0.66\linewidth}
\setcounter{exx}{0}
\begin{exe}
    \ex In a shallow pot, combine kimchi, kimchi brine, pork, onion, and $\sfrac{2}{3}$ scallion
    \ex Add salt, sugar, gochugaru, and gochujang, and sesame oil, then add anchovy stock
    \ex Cover and cook over medium-high heat for 10 minutes.
    \ex Stir seasoning, and lay tofu over top. Cover and cook for another 10 minutes
    \ex Remove from heat and garnish with reserved green onion. Serve immediately
\end{exe}
\rule{\textwidth}{0.4pt}
\begin{enumerate}[]
    \item \textbf{Notes:}
    \begin{enumerate}[-]
        \item You can use your stock of choice, but anchovy is best
        \item Makes 2-4 servings
    \end{enumerate}
\end{enumerate}
\end{minipage}}
\vskip 0.025in
\rule{\textwidth}{0.4pt}
\vfill

\subsection{Napa Cabbage Kimchi}
\textit{Maangchi}

\rule{\textwidth}{0.4pt}
\adjustbox{valign=t}{\begin{minipage}{0.3\linewidth}
\begin{enumerate}[]
    \item $3$ lbs Nappa cabbage
    \item $\sfrac{1}{4}$ cup kosher salt
    \item \textbf{Porridge:}
    \begin{enumerate}[]
        \item $1$ cup water
        \item $1$ tbsp glutinous rice flour
        \item $1$ tbsp sugar
    \end{enumerate}
    \item \textbf{Vegetables:}
    \begin{enumerate}[]
        \item $1$ cup radish matchsticks
        \item $\sfrac{1}{2}$ cup carrot matchsticks
        \item $4$ scallions chopped
        \item $\sfrac{1}{2}$ cup buchu, chopped
        \item $\sfrac{1}{2}$ cup minari
    \end{enumerate}
    \item \textbf{Seasoning:}
    \begin{enumerate}[]
        \item $12$ cloves garlic
        \item $1$ tsp ginger, minced
        \item $\sfrac{1}{2}$ medium onion, minced
        \item $\sfrac{1}{4}$ cup fish sauce
        \item $\sfrac{1}{8}$ cup saeujeot, chopped
        \item $1$ cup gochugaru
    \end{enumerate}
\end{enumerate}
\end{minipage}}
\adjustbox{valign=t}{\begin{minipage}{0.03\linewidth}
\hfill
\end{minipage}}
\adjustbox{valign=t}{\begin{minipage}{0.66\linewidth}
\setcounter{exx}{0}
\begin{exe}
    \ex Cut a short slit in the base of each cabbage and gently pull the halves apart so the cabbage splits open.
    \ex Cut a slit through the core of each half, 2 inches above the stem. Leaves should be loose but still attached.
    \ex Rinse thoroughly and massage salt into leaves and stems.
    \item Let cabbage sit two hours, turning every $30$ minutes and ladling salty water over leaves. Then thoroughly rinse cabbage, splitting along the slits in the base.
    \item To make porridge: stir rice flour in water on medium heat until it starts to bubble ($\sim 10$ min). Stirr in sugar and cook 1 minute more. Remove from heat and cool.
    \item In a large bowl, combine cooled porridge and seasonings, mixing well, then add vegetables.
    \item Spread each cabbage leaf with some of the kimchi paste/vegetable mixture, bundling cabbage quarters and placing in storage container as you work.
    \item Leave in a dark place at room temperature to ferment, then store chilled to slow fermentation
\end{exe}
\rule{\textwidth}{0.4pt}
\begin{enumerate}[]
    \item \textbf{Notes:}
    \begin{enumerate}[-]
        \item More scallion may be substituted for buchu (asian chives)
        \item Minari (water dropwort) is optional
        \item Makes $\sim 4$ lbs kimchi
    \end{enumerate}
\end{enumerate}
\end{minipage}}
\vskip .025in
\rule{\textwidth}{0.4pt}
\vfill
\newpage

\subsection{Jjajangmyeon}
\textit{Maangchi}

\rule{\textwidth}{0.4pt}
\adjustbox{valign=t}{\begin{minipage}{0.3\linewidth}
\begin{enumerate}[]
    \item Jjajangmyeon noodles
    \item $\sfrac{1}{2}$lb Pork belly, diced
    \item $1$ cup daikon, diced
    \item $1$ cup zucchini, diced
    \item $1$ large potato, diced
    \item $2$ medium onion, diced
    \item $3$ tbsp vegetable oil
    \item $\sfrac{1}{4}$ cup black bean paste
    \item $1$ tsp sesame oil
    \item $\sfrac{1}{2}$ cucumber, julienned
    \item \textbf{Slurry:}
    \begin{enumerate}[]
        \item $2$ tbsp potato starch
        \item $\sfrac{1}{4}$ cup water
        \item $1$ tsp sugar
    \end{enumerate}
\end{enumerate}
\end{minipage}}
\adjustbox{valign=t}{\begin{minipage}{0.03\linewidth}
\hfill
\end{minipage}}
\adjustbox{valign=t}{\begin{minipage}{0.66\linewidth}
\setcounter{exx}{0}
\begin{exe}
    \item Heat 1 tbsp oil over medium heat and stir fry pork belly until fat is rendered and pork is crispy
    \item Add radish and stir fry one minute
    \item Add onion, potato, and zucchini and stir fry until potato is translucent ($\sim$3 min)
    \item Clear center of wok and add black bean paste. Cook 1 minute then mix.
    \item Add 2 cups water, bring to a simmer and cover for 10 minutes, or until potato is cooked through.
    \item Add in slurry (slowly to prevent over-thickening)
    \item Add in sesame oil and serve over noodles or rice
\end{exe}
\rule{\textwidth}{0.4pt}
\begin{enumerate}[]
    \item \textbf{Notes:}
    \begin{enumerate}[-]
        \item Makes 2-3 servings
    \end{enumerate}
\end{enumerate}
\end{minipage}}
\vskip .025in
\rule{\textwidth}{0.4pt}

\subsection{Spicy Garlic Fried Chicken}
\textit{Maangchi}

\rule{\textwidth}{0.4pt}
\adjustbox{valign=t}{\begin{minipage}{0.3\linewidth}
\begin{enumerate}[]
    \item $\sfrac{1}{2}$lb chicken, cubed
    \item $\sfrac{1}{2}$ tsp minced ginger
    \item $4$ tsp soy sauce, divided (1:3)
    \item $\sfrac{1}{2}$ cup plus $1$ tsp potato starch, divided
    \item $1$ egg white
    \item $1\sfrac{1}{4}$ cup oil, divided
    \item $\sfrac{1}{2}$ cup shredded leek
    \item $4$ cloves garlic, halved
    \item $1$ tbsp chili flake
    \item $2$ chillies, thinly sliced
    \item $1$ green onion, chopped
    \item $\sfrac{1}{2}$ medium onion, chopped
    \item $4$ dried chili pepper
    \item $2$ tbsp rice syrup (or sugar)
    \item $1$ tbsp white vinegar
\end{enumerate}
\end{minipage}}
\adjustbox{valign=t}{\begin{minipage}{0.03\linewidth}
\hfill
\end{minipage}}
\adjustbox{valign=t}{\begin{minipage}{0.66\linewidth}
\setcounter{exx}{0}
\begin{exe}
    \item Combine chicken, ginger, 1 tsp soy sauce, and $\sfrac{1}{4}$ tsp black pepper in bowl and let sit 10-20 minutes; then add $\sfrac{1}{2}$ cup potato starch and egg white and mix well
    \item Combine 1 tbsp soy sauce, 2 tbsp water, rice syrup, vinegar, and 1 tsp potato starch to make sauce
    \item Heat small pan with $\sfrac{1}{4}$ cup oil. Reduce heat to medium and add garlic and leek. Stir until golden brown and crispy. Transfer garlic and leek to plate.
    \item Stir chilli flake into reserved oil. Let sit 3 minutes and strain.
    \item Heat $1$ cup oil on high, then reduce to medium and add chicken. Fry until chicken is crispy on both sides breaking apart stuck-together pieces. Strain chicken and turn off heat.
    \item Re-fry chicken until golden brown ($\sim$5 min), then strain.
    \item Cook onion, green onion, and chillies in infused oil over medium-high heat until onion is crispy (3 min). Add sauce and stir 20-30 seconds, then remove from heat.
    \item Reheat sauce until sizzling, then add chicken, leek, garlic and $1$ tsp sesame oil, toss and serve.
\end{exe}
\rule{\textwidth}{0.4pt}
\begin{enumerate}[]
    \item \textbf{Notes:}
    \begin{enumerate}[-]
        \item Makes 2 servings (use whole egg when doubling)
    \end{enumerate}
\end{enumerate}
\end{minipage}}
\vskip 0.025in
\rule{\textwidth}{0.4pt}
\newpage

\section{Thailand}
\subsection{Pad Kee Mao (Drunken Noodles)}
\textit{Maggie Zhu, Omnivore's Cookbook}

\rule{\textwidth}{0.4pt}
\adjustbox{valign=t}{\begin{minipage}{0.3\linewidth}
\begin{enumerate}[]
    \item \textbf{Sauce:}
    \begin{enumerate}[]
        \item $2$ tbsp oyster sauce
        \item $2$ tsp light soy sauce
        \item $2$ tsp dark soy sauce
        \item $\sfrac{1}{2}$ tsp fish sauce
        \item $1$ tbsp brown sugar
    \end{enumerate}
    \item \textbf{Protein:}
    \begin{enumerate}[]
        \item $170g$ chicken, thinly sliced
        \item $\sfrac{1}{4}$ tsp soy sauce
    \end{enumerate}
    \item \textbf{Stir Fry:}
    \begin{enumerate}[]
        \item $1$ tbsp vegetable oil
        \item $4$ cloves garlic
        \item $3$-$5$ fresh Thai chilis
        \item $4$ stalks Chinese broccoli (or broccolini), thinly sliced
        \item $\sfrac{1}{2}$ cup baby corn, sliced
        \item $340$ g fresh rice noodles
        \item $\sfrac{1}{2}$ cup basil
    \end{enumerate}
\end{enumerate}
\end{minipage}}
\adjustbox{valign=t}{\begin{minipage}{0.03\linewidth}
\hfill
\end{minipage}}
\adjustbox{valign=t}{\begin{minipage}{0.66\linewidth}
\setcounter{exx}{0}
\begin{exe}
    \ex Combine sauce ingredients in a bowl, stir well, and set aside.
    \ex Combine protein of choice with soy sauce and set aside.
    \ex Pulverise garlic and chilis in a mortar and pestle.
    \ex Heat oil on medium high. Add garlic-chili paste ($\sim$30 sec).
    \ex Brown one side of protein, then flip and stir.
    \ex Add vegetables. Cook until protein is nearly done.
    \ex Add noodles and sauce. Cook, tossing constantly, until noodles absorb sauce and start to crackle.
    \ex Remove from heat and add basil. Toss until basil is wilted, then serve.
\end{exe}
\rule{\textwidth}{0.4pt}
\begin{enumerate}[]
    \item \textbf{Notes:}
    \begin{enumerate}[-]
        \item Any protein may be used. Chicken thighs are optimal, but pork, chicken breast, beef, and tofu all work.
        \item Any crisp, dry vegetable (e.g. water chestnuts) can be substituted for baby corn
        \item 200 g dried rice noodles may be used instead of fresh. Re-hydrate according to package instructions.
        \item Makes 2-4 servings
    \end{enumerate}
\end{enumerate}
\end{minipage}}
\vskip .025in
\rule{\textwidth}{0.4pt}
\vfill

\newpage
\section{Asian Fusion}
\subsection{Kimchi Udon}
\textit{Andy Baraghani, Bon Appétit (October 2016)}

\rule{\textwidth}{0.4pt}
\adjustbox{valign=t}{\begin{minipage}{0.3\linewidth}
\begin{enumerate}[]
    \item $5$ tbsp butter, divided
    \item $1$ cup finely chopped kimchi
    \item $\sfrac{1}{3}$ cup kimchi brine
    \item $2$ tbsp gochujang
    \item $\sfrac{1}{2}$ cup chicken broth
    \item $1$ lb udon noodles (fresh or frozen)
    \item $4$ egg yolks, room temperature
    \item $3$ scallions, sliced diagonally
    \item $1$ tbsp toasted sesame seeds
\end{enumerate}
\end{minipage}}
\adjustbox{valign=t}{\begin{minipage}{0.03\linewidth}
\hfill
\end{minipage}}
\adjustbox{valign=t}{\begin{minipage}{0.66\linewidth}
\setcounter{exx}{0}
\begin{exe}
    \item Heat 2 tbsp butter in large skillet over medium-high heat
    \item Add kimchi and gochujang and cook until kimchi is softened and lightly caramelized (4 minutes)
    \item Add broth and kimchi brine and bring to a simmer
    \item Cook until sauce is slightly reduced (3 minutes)
    \item Cook noodles according to package directions
    \item Transfer noodles to skillet and add remaining 3 tbsp butter
    \item Cook, tossing noodles to coat, about 2 minutes
    \item Serve topped with egg yolks, scallion, and sesame seeds
\end{exe}
\rule{\textwidth}{0.4pt}
\begin{enumerate}[]
    \item \textbf{Notes:}
    \begin{enumerate}[-]
        \item Makes 4 servings
    \end{enumerate}
\end{enumerate}
\end{minipage}}
\vskip 0.025 in
\rule{\textwidth}{0.4pt}
\vfill

\subsection{Spicy-Sweet Sambal Pork Noodles}
\textit{Chris Morocco, Bon Appétit (February 2019)}

\rule{\textwidth}{0.4pt}
\adjustbox{valign=t}{\begin{minipage}{0.3\linewidth}
\begin{enumerate}[]
    \item $2$ tbsp olive oil
    \item $2$ lbs ground pork, divided
    \item $2$ inches ginger, julienned
    \item $8$ cloves garlic, thinly sliced
    \item $2$ tbsp sugar
    \item $2$ tbsp tomato paste
    \item $2$ sprigs basil
    \item $\sfrac{1}{3}$ cup sambal oelek
    \item $\sfrac{1}{4}$ cup soy sauce
    \item $\sfrac{1}{4}$ cup rice vinegar
    \item $1\sfrac{1}{2}$ lbs fresh ramen noodles
    \item $2$ tbsp butter
\end{enumerate}
\end{minipage}}
\adjustbox{valign=t}{\begin{minipage}{0.03\linewidth}
\hfill
\end{minipage}}
\adjustbox{valign=t}{\begin{minipage}{0.66\linewidth}
\setcounter{exx}{0}
\begin{exe}
    \item Heat oil over medium high. Add half of the pork and break into 6-8 large chunks ($\sim$ 10 min)
    \item Add ginger, garlic, sugar, and remaining pork and cook until meat is mostly cooked through ($\sim$5 min)
    \item Add tomato paste and basil. Cook until paste darkens
    \item Add chili paste, soy sauce, vinegar, and 2 cups water. Bring to a simmer, then reduce heat to low and cook, uncovered, until sauce is slightly thickened ($\sim$30-45 min)
    \item Cook noodles in salted water 1 min short of al dente. Transfer to pot with sauce and add butter and a splash of pasta water. Simmer $\sim$ 1 min.
    \item Remove basil stems and serve.
\end{exe}
\rule{\textwidth}{0.4pt}
\begin{enumerate}[]
    \item \textbf{Notes:}
    \begin{enumerate}[-]
        \item Makes 6-8 servings
    \end{enumerate}
\end{enumerate}
\end{minipage}}
\vskip 0.025 in
\rule{\textwidth}{0.4pt}
\vfill

\newpage
\section{Mexico}
\subsection{Pozole Verde con Hongos}
\textit{Pati Jinich, Epicurious (January 2021)}

\rule{\textwidth}{0.4pt}
\adjustbox{valign=t}{\begin{minipage}{0.3\linewidth}
\begin{enumerate}[]
    \item $1\sfrac{1}{4}$ lb tomatillo, husked and rinsed
    \item $2$ cloves garlic
    \item $3$ poblano chiles, seeded and coarsely chopped
    \item $1$ serrano chile, coarsely chopped
    \item $1$ cup salted, roasted pumpkin seeds (pepitas)
    \item $1$ cup chopped cilantro
    \item $1$ cup chopped parsley
    \item $3$ tbsp chopped white onion
    \item $6$ cups vegetable broth
    \item $1\sfrac{1}{2}$ tsp salt
    \item $2$ tbsp vegetable oil
    \item $2$ lbs mixed mushroom, sliced
    \item $\sfrac{1}{2}$ tsp black pepper
    \item $30$ oz canned hominy, drained
    \item $3$ radishes, sliced crosswise
    \item lime wedges
\end{enumerate}
\end{minipage}}
\adjustbox{valign=t}{\begin{minipage}{0.03\linewidth}
\hfill
\end{minipage}}
\adjustbox{valign=t}{\begin{minipage}{0.66\linewidth}
\setcounter{exx}{0}
\begin{exe}
    \ex Combine tomatillos and garlic in a large pot. Cover with water, bring to a boil, and simmer until tomatillos are soft ($\sim$10 min). Drain, reserving 1 cup cooking liquid.
    \ex Combine reserved cooking liquid, $1$ cup broth, tomatillos, garlic, poblano, serrano, pepitas, cilantro, parsely, and onion. Blend until smooth.
    \ex Heat oil in another pot over medium-high. Add mushrooms, season with salt and pepper, and cook until liquid evaporates and mushrooms are slightly browned ($\sim$10 min).
    \ex Add tomatillo sauce to mushrooms. Cook until thickened and slightly darkened ($\sim$10 min).
    \ex Add hominy and remaining $5$ cups broth. Cook until flavors meld ($\sim$15 min).
    \ex Serve with additional cilantro, parsley, onion, radishes and lime as (optional) toppings
\end{exe}
\rule{\textwidth}{0.4pt}
\begin{enumerate}[]
    \item \textbf{Notes:}
    \begin{enumerate}[-]
        \item 4-6 jalapeños may be substituted in place of poblano
        \item Freshly fried corn toritalla strips make a good topping
        \item Crema or sour cream may be added to moderate spice
        \item Thoroughly brown some mushrooms and reserve for topping
        \item Makes 4-6 servings
    \end{enumerate}
\end{enumerate}
\end{minipage}}
\vskip 0.025in
\rule{\textwidth}{0.4pt}
%\vfill

\newpage
\section{Italy}
\subsection{Pasta Puttanesca}
\textit{Ian Knauer, Gourmet (June 2018)}

\rule{\textwidth}{0.4pt}
\adjustbox{valign=t}{\begin{minipage}{0.3\linewidth}
\begin{enumerate}[]
    \item $1$ lb spaghetti
    \item $\sfrac{1}{3}$ cup olive oil
    \item $5$ cloves garlic, minced
    \item $5$ anchovy fillets, minced
    \item $1$ tsp red pepper flakes
    \item $1$ tsp salt
    \item $\sfrac{1}{2}$ teaspoon pepper
    \item $28$ oz canned whole tomatoes
\item $\sfrac{1}{2}$ cup Kalamata olive, sliced
    \item $2$ tbsp drained capers
\end{enumerate}
\end{minipage}}
\adjustbox{valign=t}{\begin{minipage}{0.03\linewidth}
\hfill
\end{minipage}}
\adjustbox{valign=t}{\begin{minipage}{0.66\linewidth}
\setcounter{exx}{0}
\begin{exe}
    \item Cook garlic, anchovy, red-pepper, salt, and pepper on medium-high until pale golden ($2$ minutes)
    \item Add tomato, olives, and capers and simmer, breaking up tomatoes with the back of a spoon
    \item Cook pasta until barely al dente, then drain, reserving water
    \item Add pasta to sauce and simmer until coated and al dente, add pasta water if dry
\end{exe}
\rule{\textwidth}{0.4pt}
\begin{enumerate}[]
    \item \textbf{Notes:}
    \begin{enumerate}[-]
        \item Add olive/caper brine to increase brightness
        \item $2$ tsp anchovy paste may be substituted for minced fillets
        \item Makes 4-6 servings
    \end{enumerate}
\end{enumerate}
\end{minipage}}
\vskip 0.025in
\rule{\textwidth}{0.4pt}
%\vfill

\subsection{Pasta Carbonara}
\textit{Claire Saffitz, Bon Appétit (February 2018)}

\rule{\textwidth}{0.4pt}
\adjustbox{valign=t}{\begin{minipage}{0.3\linewidth}
\begin{enumerate}[]
    \item $1$ lb spaghetti/bucatini
    \item $4$ oz guanciale/pancetta
    \item $2$ oz parmesan, grated
    \item $4$ egg yolks
    \item $2$ whole eggs
    \item freshly ground black pepper
\end{enumerate}
\end{minipage}}
\adjustbox{valign=t}{\begin{minipage}{0.03\linewidth}
\hfill
\end{minipage}}
\adjustbox{valign=t}{\begin{minipage}{0.66\linewidth}
\setcounter{exx}{0}
\begin{exe}
    \item Whisk together eggs, yolks, parmesan, and black pepper.
    \item Cook pancetta to render. Transfer to bowl, reserving fat.
    \item Cook pasta $\sim 4$-$6$ min. Reserve $1\sfrac{3}{4}$ cup water and drain.
    \item Add $1$ cup pasta water to fat and bring to a boil. Add pasta, cook til al dente ($\sim 2$ min), and remove from heat.
    \item Whisk $\sfrac{1}{4}$ cup pasta water into egg/parmesan mixture, then slowly drizzle into pasta, mixing constantly.
    \item Mix in cooked pancetta, and serve.
\end{exe}
\end{minipage}}
\vskip 0.025in
\rule{\textwidth}{0.4pt}
\begin{enumerate}[]
    \item \textbf{Notes:}
    \begin{enumerate}[-]
        \item Thin sauce with remaining $\sfrac{1}{2}$ cup pasta water as necessary
        \item Makes 3-4 servings
    \end{enumerate}
\end{enumerate}
\vskip -0.15in
\rule{\textwidth}{0.4pt}

\subsection{Bucatini All'Amatriciana}
\textit{Mailino, Epicurious (December 2015)}

\rule{\textwidth}{0.4pt}
\adjustbox{valign=t}{\begin{minipage}{0.3\linewidth}
\begin{enumerate}[]
    \item $8$ oz guanciale, diced
    \item $2$ tsp red pepper flake
    \item $28$ oz canned whole tomatoes
    \item $1\sfrac{1}{2}$ lbs bucatini or spaghetti
    \item $\sfrac{1}{4}$ cup olive oil
    \item $1\sfrac{1}{4}$ cup pecorino, grated
\end{enumerate}
\end{minipage}}
\adjustbox{valign=t}{\begin{minipage}{0.03\linewidth}
\hfill
\end{minipage}}
\adjustbox{valign=t}{\begin{minipage}{0.66\linewidth}
\setcounter{exx}{0}
\begin{exe}
    \item Cook guanciale to render fat ($\sim 10$ min). Stir in pepper and cook 1 minute. Add tomatoes and simmer until thickened.
    \item Cook pasta until just al dente. Drain, reserving $1$ cup water.
    \item Finish pasta in sauce with reserved water over high heat.
    \item Stir in oil and cheese and serve
\end{exe}
\rule{\textwidth}{0.4pt}
\begin{enumerate}[]
    \item \textbf{Notes:}
    \begin{enumerate}[-]
    \item Makes 6-8 servings
    \end{enumerate}
\end{enumerate}
\end{minipage}}
\vskip 0.025in
\rule{\textwidth}{0.4pt}
\vfill
%\newpage

\subsection{Ricotta Gnudi}
\textit{Sara Jenkins, Bon Appétit (February 2013)}

\rule{\textwidth}{0.4pt}
\adjustbox{valign=t}{\begin{minipage}{0.3\linewidth}
\begin{enumerate}[]
    \item $16$ oz ricotta
    \item $1$ egg, beaten
    \item $1$ egg yolk, beaten
    \item $\sfrac{1}{2}$ tsp black pepper
    \item $\sfrac{1}{2}$ tsp salt
    \item $\sfrac{1}{2}$ cup parmesan, grated
    \item $\sfrac{1}{2}$ cup flour
\end{enumerate}
\end{minipage}}
\adjustbox{valign=t}{\begin{minipage}{0.03\linewidth}
\hfill
\end{minipage}}
\adjustbox{valign=t}{\begin{minipage}{0.66\linewidth}
\setcounter{exx}{0}
\begin{exe}
    \item Combine ricotta, egg, egg yolk, pepper, salt, and parmesan in a large bowl
    \item Add flour and mix to incorporate (dough will be very moist)
    \item Using two large spoons, shape dough into footballs, transferring to floured baking sheet
    \item Cook gnudi in large pot of boiling, salted water until cooked through and tender (5-6 minutes)
\end{exe}
\end{minipage}}
\vskip 0.025in
\rule{\textwidth}{0.4pt}
\begin{enumerate}[]
    \item \textbf{Notes:}
    \begin{enumerate}[-]
    \item Gnudi can be made up to 2 hours in advance; cover and chill on baking sheet in refrigerator
    \item Alternative shaping method:
    \begin{enumerate}[$\ast$]
        \item Divide dough into three portions
        \item Roll each ortion into long, $1$in thick strands
        \item Cut strands at $1\sfrac{1}{2}$ inch intervals
    \end{enumerate}
    \item Makes 30 gnudi (6 servings)
    \end{enumerate}
\end{enumerate}
\rule{\textwidth}{0.4pt}
\vfill

\subsection{Ricotta}
\textit{Gourmet (April 2006)}

\rule{\textwidth}{0.4pt}
\adjustbox{valign=t}{\begin{minipage}{0.3\linewidth}
\begin{enumerate}[]
    \item $2$ qts whole milk
    \item $1$ cup heavy cream
    \item $1$ tsp salt
    \item $3$ tbsp lemon juice
\end{enumerate}
\end{minipage}}
\adjustbox{valign=t}{\begin{minipage}{0.03\linewidth}
\hfill
\end{minipage}}
\adjustbox{valign=t}{\begin{minipage}{0.66\linewidth}
\setcounter{exx}{0}
\begin{exe}
    \item Slowly bring milk, cream, and salt to boil, stirring occasionally to prevent scorching
    \item Add lemon juice and reduce heat to low
    \item Simmer, stirring constantly, until mixture curdles (2 min)
    \item Strain mixture through fine mesh sieve lined with heavy-duty cheesecloth and let drain (1 hour)
\end{exe}
\end{minipage}}
\vskip 0.025in
\rule{\textwidth}{0.4pt}
\begin{enumerate}[]
    \item \textbf{Notes:}
    \begin{enumerate}[-]
    \item Cook with lemon juice longer for drier ricotta
    \item Drain less time for wetter ricotta
    \item Can be stored in airtight container, chilled, for two days
    \item Makes about 16 oz
    \end{enumerate}
\end{enumerate}
\rule{\textwidth}{0.4pt}
\vfill

\newpage

\section{European Fusion}
\subsection{Pork Marbela}
\textit{Adam Rapoport, Bon Appétit (February 2020)}

\rule{\textwidth}{0.4pt}
\adjustbox{valign=t}{\begin{minipage}{0.3\linewidth}
\begin{enumerate}[]
    \item $2$ $1$-lb pork tenderloins
    \item $2$ tsp kosher salt
    \item $1$ cup dry white wine
    \item $\sfrac{1}{2}$ cup pitted prunes, torn in half
    \item $\sfrac{1}{3}$ cup light brown sugar
    \item $\sfrac{1}{4}$ cup capers, plus $1$ tbsp brine
    \item $\sfrac{1}{2}$ cup pitted, spanish green olives
    \item $\sfrac{1}{4}$ cup red wine vinegar
    \item $4$ garlic cloves, peeled and smashed
    \item $2$ bay leaves
    \item $1$ tbsp dried oregano
    \item $\sfrac{1}{4}$ cup plus $1$ tbsp olive oil
    \item $1$ tbsp unsalted butter
    \item $\sfrac{1}{4}$ cup chopped parsley
\end{enumerate}
\end{minipage}}
\adjustbox{valign=t}{\begin{minipage}{0.03\linewidth}
\hfill
\end{minipage}}
\adjustbox{valign=t}{\begin{minipage}{0.66\linewidth}
\setcounter{exx}{0}
\begin{exe}
    \item Season pork with salt and let sit 15 minutes while you prepare marinade
    \item Combine wine, prunes, brown sugar, capers, brine, olives, vinegar, garlic, bay leaves, oregano and $\sfrac{1}{4}$ cup olive oil in large resealable plastic bag
    \item Add pork and seal tightly, removing as much air as possible. Chill at least 2 hours or overnight.
    \item Preheat oven to 325$\degree$, remove pork from marinade and pat dry, reserving marinade
    \item Heat oil in oven-proof pan on medium high heat. Cook pork, turning until browned (6-8 minutes)
    \item Remove from heat and let cool 1 minute. Pour marinade over pork and transfer to oven.
    \item Bake until center of pork is 145$\degree$ (22-26 minutes), turning pork occasionally
    \item Transfer pork to cutting board and let rest, 10 minutes
    \item Add butter to pan and cook on medium high until reduced (3-5 minutes), then stir in parsley
    \item Slice pork crosswise and top with pan juice
\end{exe}
\rule{\textwidth}{0.4pt}
\begin{enumerate}[]
    \item \textbf{Notes:}
    \begin{enumerate}[-]
        \item Makes 4-6 servings
    \end{enumerate}
\end{enumerate}
\end{minipage}}
\vskip 0.025in
\rule{\textwidth}{0.4pt}
\vfill

\subsection{Fennel Tumeric Rice Pilaf}
\textit{Yours Truly}

\rule{\textwidth}{0.4pt}
\adjustbox{valign=t}{\begin{minipage}{0.3\linewidth}
\begin{enumerate}[]
    \item $\sfrac{1}{2}$ fennel bulb, diced, fronds reserved
    \item $1$ small sweet onion, diced
    \item $1$ tsp tumeric
    \item $1$ cup basmatic rice
    \item $1\sfrac{1}{2}$ cups chicken stock
    \item $\sfrac{1}{4}$ cup toasted almonds, chopped
    \item $\sfrac{1}{4}$ cup cilantro, chopped
\end{enumerate}
\end{minipage}}
\adjustbox{valign=t}{\begin{minipage}{0.03\linewidth}
\hfill
\end{minipage}}
\adjustbox{valign=t}{\begin{minipage}{0.66\linewidth}
\setcounter{exx}{0}
\begin{exe}
    \item Cook fennel and onion in 1tbs butter until translucent
    \item Add rice and cook until toasted and grains are translucent
    \item Add tumeric and stock and bring to a boil
    \item Reduce heat to low and cover. Cook until rice is barely dry.
    \item Remove from heat and let sit, covered, at least 10 minutes
    \item Fluff rice and top with fennel frond, almond, and cilantro
\end{exe}
\rule{\textwidth}{0.4pt}
\begin{enumerate}[]
    \item \textbf{Notes:}
    \begin{enumerate}[-]
        \item Makes 4-6 servings
    \end{enumerate}
\end{enumerate}
\end{minipage}}
\vskip0.025 in
\rule{\textwidth}{0.4pt}
\vfill
\newpage

\subsection{Glazed Duck Confit with Olive Relish \textit{\&} Sauce Verte}
\textit{Naomi Pomeroy, Bon Appétit (September 2014)}

\rule{\textwidth}{0.4pt}
\adjustbox{valign=t}{\begin{minipage}{0.3\linewidth}
\begin{enumerate}[]
    \item \textbf{Duck:}
    \begin{enumerate}[]
        \item $5$ juniper berries
        \item $1$ tsp coriander seed
        \item $1$ bay leaf, crumbled
        \item $\sfrac{3}{4}$ tsp black peppercorn
        \item $\sfrac{3}{4}$ tsp ground allspice
        \item $\sfrac{1}{4}$ tsp ground cinnamon
        \item $\sfrac{1}{4}$ tsp ground cloves
        \item $3$ tbsp kosher salt
        \item $8$ duck legs
        \item $1$ bunch thyme
        \item $1$ head garlic, quartered
    \end{enumerate}
    \item \textbf{Relish:}
    \begin{enumerate}[]
        \item $\sfrac{1}{2}$ cup pitted prunes
        \item $2$ wide strips lemon zest
        \item $\sfrac{1}{2}$ cup Armagnac
        \item $\sfrac{1}{2}$ cup sugar
        \item $6$ tbsp olive oil
        \item $1$ small shallot, minced
        \item $2$ cloves garlic, minced
        \item $\sfrac{1}{4}$ tsp red pepper flake
        \item $\sfrac{1}{4}$ cup green olive, chopped
        \item $1$ tbsp sherry vinegar
    \end{enumerate}
    \textbf{Parsnip Puree:}
    \begin{enumerate}[]
        \item $1$ lb parsnip, thinly sliced
        \item $2$ cloves garlic, thinly sliced
        \item $1$ cup heavy cream
        \item $2$ tbsp butter
    \end{enumerate}
    \item \textbf{Sauce Verte:}
    \begin{enumerate}[]
        \item $1$ small shallot, minced
        \item $3$ tbsp white wine vinegar
        \item $\sfrac{1}{2}$ cup olive oil
        \item $3$ tbsp parsely, minced
        \item $1$ tbsp chive, minced
        \item $1$ tbsp mint, minced
    \end{enumerate}
    \item \textbf{Glaze:}
    \begin{enumerate}[]
        \item $\sfrac{1}{2}$ cup dark brown sugar
        \item $\sfrac{1}{2}$ cup sherry vinegar
    \end{enumerate}
\end{enumerate}
\end{minipage}}
\adjustbox{valign=t}{\begin{minipage}{0.03\linewidth}
\hfill
\end{minipage}}
\adjustbox{valign=t}{\begin{minipage}{0.66\linewidth}
\setcounter{exx}{0}
\textbf{Duck:}
\begin{exe}
    \ex Toast juniper and coriander until fragrant ($\sim$2 min). Cool and grind with bay leaf, peppercorn, and allspice. Mix with cinnamon, cloves, and salt.
    \ex Score duck and rub in spices. Chill at least 8 hours.
    \ex Heat oven to 250$\degree$F. Rinse duck legs, pat dry, and arrange skin-side down in a large roasting pan. Add thyme, garlic, and $\sfrac{1}{2}$ cup water. Cover with foil and cook until fat submerges duck ($\sim$2 hours). Turn duck skin-side up, and continue to bake until meat is very tender ($\sim$2-2.5 hours).
\end{exe}
\textbf{Relish:}
\begin{exe}
    \ex Soften prunes in 1 cup boiling water ($\sim$20 min).
    \ex Simmer lemon zest, Armagnac, and sugar until alcohol cooks off ($\sim$15 min). Add drained prunes and simmer until plumped ($\sim$12 min). When cool, drain and quarter prunes.
    \ex Heat 3 tbsp oil on medium-high. Cook garlic, shallot, and red pepper until shallot is translucent ($\sim$2 min). Combine with prunes, olives, vinegar, and 3 tbsp oil.
\end{exe}
\textbf{Parsnip Puree:}
\begin{exe}
    \ex Bring parsnip, garlic, cream, and butter to a boil; cover, and simmer until tender ($\sim$15 min). Uncover, and reduce liquid by half ($\sim$ 5 min). Season with salt, and puree until smooth.
\end{exe}
\textbf{Sauce Verte:}
\begin{exe}
    \ex Combine shallot, white wine vinegar, and a pinch of salt in a small bowl. Let sit five minutes, then add oil and herbs.
\end{exe}
\textbf{Glaze \textit{\&} Assembly:}
\begin{exe}
    \ex Simmer brown sugar, sherry vinegar, and a pinch of salt until slightly reduced ($\sim$3 min).
    \ex Remove duck from fat and arrange on a rimmed baking sheet. Roast at 400$\degree$F until skin is browned and crisp ($\sim$20 min). Brush duck with vinegar mixture and roast, brushing again halfway through, until glaze is bubbling ($\sim$8 min).
    \ex Serve over parsnip puree. Top with relish and sauce verte.
\end{exe}

\rule{\textwidth}{0.4pt}
\begin{enumerate}[]
    \item \textbf{Notes:}
    \begin{enumerate}[-]
        \item Makes 6-8 servings
        \item Duck confit can be made 5 days ahead (store in fat)
        \item Relish can be made 2 days ahead
        \item Parsnip puree can be made 1 day ahead, reheat on low
        \item Sauce verte can be made 1 hour ahead
        \item \textbf{For duck breast:} sear duck, unrinsed, skin-side down on high heat until fat is rendered and skin is browned. Roast skin-side up at 425$\degree$F for 8-10 minutes. Let rest ($\sim$10 min), slice, and serve.
    \end{enumerate}
\end{enumerate}
\end{minipage}}
\vskip0.025 in
\rule{\textwidth}{0.4pt}
\vfill

\newpage

\subsection{Oven Polenta with Roasted Mushrooms \textit{\&} Thyme}
\textit{Claire Saffitz, Bon Appétit (September 2018)}

\rule{\textwidth}{0.4pt}
\adjustbox{valign=t}{\begin{minipage}{0.3\linewidth}
\begin{enumerate}[]
    \item $1\sfrac{1}{2}$ lb mixed mushrooms
    \item $4$ sprigs thyme
    \item $6$ cloves garlic, smashed
    \item $\sfrac{1}{4}$ olive oil
    \item $2$ tbsp butter
    \item $1$ cup polenta
    \item $4$ oz parmesan, grated
    \item $1$ tbsp red wine vinegar
    \item Salt and pepper for seasoning
\end{enumerate}
\end{minipage}}
\adjustbox{valign=t}{\begin{minipage}{0.03\linewidth}
\hfill
\end{minipage}}
\adjustbox{valign=t}{\begin{minipage}{0.66\linewidth}
\setcounter{exx}{0}
\begin{exe}
    \item Combine mushrooms, thyme, and garlic on a baking sheet. Season and toss with oil. Roast at 325$\degree$ in upper part of oven.
    \item Bring $4\sfrac{1}{2}$ cup water to simmer, then add butter and salt. Gradually add polenta, whisking constantly. Return mixture to a boil and cover, then move to lower rack in oven. Bake until tender ($\sim$25-30 min).
    \item Remove polenta from oven. Maximise oven temp and roast mushrooms until edges are crispy ($\sim$5-10 min).
    \item Whisk polenta vigorously, then gradually add parmesan. Season with salt and pepper, then cover to keep warm.
    \item Drizzle mushrooms with vineger, toss, and cool slightly.
    \item Serve topped with mushrooms, thyme, and more parmesan.
\end{exe}
\end{minipage}}
\vskip 0.025in
\rule{\textwidth}{0.4pt}
\begin{enumerate}[]
    \item \textbf{Notes:}
    \begin{enumerate}[-]
        \item Don't overcrowd the baking pan when roasting the mushrooms
        \item For added decadence, top polenta with a bit of cream mixed with grated garlic
        \item Makes 4-6 servings
    \end{enumerate}
\end{enumerate}
\rule{\textwidth}{0.4pt}
\vfill

\subsection{Pecorino \textit{\&} Leek Risotto with Lemony Asparagus}
\textit{Sonja Overhiser \textit{\&} Alex Overhiser, ACoupleCooks.com}

\rule{\textwidth}{0.4pt}
\adjustbox{valign=t}{\begin{minipage}{0.3\linewidth}
\begin{enumerate}[]
    \item \textbf{Risotto:}
    \begin{enumerate}[]
        \item $2$ cups vegetable broth
        \item $2$ cups water
        \item $1$ tbsp olive oil
        \item $1$ tbsp butter
        \item $\sfrac{1}{2}$ large leek, chopped
        \item $1$ cup arborio rice
        \item $\sfrac{3}{4}$ cup dry white wine
        \item $\sfrac{1}{2}$ tsp salt
        \item $85$ g pecorino, grated
    \end{enumerate}
    \item \textbf{Asparagus:}
    \begin{enumerate}[]
        \item $8$ oz asparagus, trimmed
        \item $1$ lemon
        \item olive oil
    \end{enumerate}
\end{enumerate}
\end{minipage}}
\adjustbox{valign=t}{\begin{minipage}{0.03\linewidth}
\hfill
\end{minipage}}
\adjustbox{valign=t}{\begin{minipage}{0.66\linewidth}
\setcounter{exx}{0}
\begin{exe}
    \item Bring broth and water to simmer over medium-low heat.
    \item Drizzle asparagus with oil and season with salt and pepper. Roast at 425$\degree$ until tender ($\sim$12 min).
    \item Heat butter and oil on medium. Add leek and cook to soften.
    \item Add rice, and cook until just barely browning.
    \item Add broth, one ladle-full at a time, allowing rice to absorb broth between additions.
    \item When grains are al dente, add one more ladle of broth, cheese, and season with salt/pepper, stirring vigorously.
    \item Serve risotto topped with asparagus, spritzed with lemon juice and topped with more cheese.
\end{exe}
\rule{\textwidth}{0.4pt}
\begin{enumerate}[]
    \item \textbf{Notes:}
    \begin{enumerate}[-]
        \item Makes 2-3 servings
    \end{enumerate}
\end{enumerate}
\end{minipage}}
\vskip 0.025in
\rule{\textwidth}{0.4pt}
\vfill

\newpage

\subsection{Clams with Spicy Tomato Broth \textit{\&} Garlic Mayo}
\textit{Andy Baraghani, Bon Appétit (June 2019)}

\rule{\textwidth}{0.4pt}
\adjustbox{valign=t}{\begin{minipage}{0.3\linewidth}
\begin{enumerate}[]
    \item $\sfrac{1}{2}$ lemon
    \item $4$ cloves garlic sliced, $1$ whole
    \item $\sfrac{1}{2}$ cup mayonnaise
    \item $7$ tbsp olive oil, divided
    \item $2$ large shallots, thinly sliced
    \item $1$ Fresno chile, thinly sliced
    \item $2$ tbsp tomato paste
    \item $2$ cups cherry tomatoes
    \item $1$ cup dry white wine
    \item $36$ littleneck clams, scrubbed
    \item $6$ tbsp butter, cubed
    \item $6$ tbsp chives, chopped
    \item $4$ slices country bread
\end{enumerate}
\end{minipage}}
\adjustbox{valign=t}{\begin{minipage}{0.03\linewidth}
\hfill
\end{minipage}}
\adjustbox{valign=t}{\begin{minipage}{0.66\linewidth}
\setcounter{exx}{0}
\begin{exe}
    \ex Combine zest from lemon, juice, 1 clove grated garlic, and mayo in small bowl; season with salt/pepper
    \ex Over medium, heat $\sfrac{1}{4}$ cup oil in skillet. Add sliced garlic, shallots, and chile and cook until just softened ($\sim2$ min).
    \item Add tomato paste and cook until paste darkens ($\sim 1$ min). Add tomatoes and a pinch of salt and cook until tomatoes soften and release their juices ($\sim 4$ min).
    \item Add wine and cook until almost reduced by half and no longer smells boozy ($\sim 3$ min).
    \item Add clams and butter and cover. Cook until clams open ($\sim 6$–$10$ min). Remove from heat and sprinkle with chives.
    \item Drizzle bread with remaining $3$ tbsp oil and season lightly with salt. Grill until golden brown and crisp.
    \item Serve clams with toasted bread and reserved garlic mayo.
\end{exe}
\rule{\textwidth}{0.4pt}
\begin{enumerate}[]
    \item \textbf{Notes:}
    \begin{enumerate}[-]
    \item May substitute $\sfrac{1}{2}$ tsp red pepper flakes for Fresno chile
    \item Makes 4 servings
    \end{enumerate}
\end{enumerate}
\end{minipage}}
\vskip 0.025in
\rule{\textwidth}{0.4pt}
\vfill

\subsection{Skillet Stuffing with Italian Sausage \textit{\&} Wild Mushrooms}
\textit{Anna Stockwell, Epicurious (November 2017)}

\rule{\textwidth}{0.4pt}
\adjustbox{valign=t}{\begin{minipage}{0.3\linewidth}
\begin{enumerate}[]
    \item $10\sfrac{1}{2}$ oz crusty white bread, cut into $\sfrac{3}{4}$ inch cubes
    \item $2$ tbsp olive oil
    \item $\sfrac{1}{2}$ lb hot Italian sausage
    \item $1$ medium onion, chopped
    \item $8$ oz mushrooms, torn
    \item $2$ tbsp sage, chopped
    \item $1$ cup dry white wine
    \item $2$ large eggs
    \item $1\sfrac{1}{4}$ cup chicken broth
    \item $1\sfrac{1}{2}$ oz parmesan, grated
    \item $1$ tbsp Dijon mustard
    \item $1\sfrac{1}{4}$ tsp salt
    \item $1$ tsp black pepper
\end{enumerate}
\end{minipage}}
\adjustbox{valign=t}{\begin{minipage}{0.03\linewidth}
\hfill
\end{minipage}}
\adjustbox{valign=t}{\begin{minipage}{0.66\linewidth}
\setcounter{exx}{0}
\begin{exe}
    \ex Toast bread at $425^\degree$F until dry ($\sim$10 min).
    \item Cook sausage in $1$ tbsp oil over medium high, breaking into pieces, until browned ($\sim$7 min). Add onion, mushroom, and sage, cooking to brown ($\sim$8-$10$ min).
    \item Add wine and cook until reduced by half. Allow to cool.
    \item Whisk eggs, broth, cheese, mustard, salt, and pepper in a large bowl. Add sausage mixture and bread and toss. Let sit $5$ minutes, toss again, then pack back into skillet.
    \item Bake at $375^\degree$F til browned and cooked through ($\sim 20$ min).
\end{exe}
\rule{\textwidth}{0.4pt}
\begin{enumerate}[]
    \item \textbf{Notes:}
    \begin{enumerate}[-]
    \item Demi-baguettes are preferred for their greater crust/bread ratio
    \item Can use sweet sausage instead of hot sausage
    \item Substitute more mushroom for sausage (and use vegetable broth) to make vegetarian
    \item Use cast iron skillet to cook sausage, and re-use to bake stuffing
    \item Makes 8 servings
    \end{enumerate}
\end{enumerate}
\end{minipage}}
\vskip 0.025in
\rule{\textwidth}{0.4pt}
\vfill

\newpage

\subsection{Beef \textit{\&} Bacon Stew}
\textit{Molly Baz, Bon Appétit (December 2018)}

\rule{\textwidth}{0.4pt}
\adjustbox{valign=t}{\begin{minipage}{0.3\linewidth}
\begin{enumerate}[]
    \item $1$ bottle dry red wine
    \item $2\sfrac{1}{2}$ lb boneless beef, cut into 2 inch cubes
    \item $1$ tsp ground black pepper
    \item $1$ tbsp kosher salt
    \item $8$ oz bacon, cut in 1in strips
    \item $3$ medium red onion, quartered lengthwise
    \item $3$ carrots, halved diagonally
    \item $8$ cloves garlic ($7$ smashed, with carrots, $1$ reserved)
    \item $4$ sprigs thyme (with carrots)
    \item $2$ tbsp white miso (w/carrots)
    \item $2$ tbsp flour
    \item $1$ cup parsley (minced)
\end{enumerate}
\end{minipage}}
\adjustbox{valign=t}{\begin{minipage}{0.03\linewidth}
\hfill
\end{minipage}}
\adjustbox{valign=t}{\begin{minipage}{0.66\linewidth}
\setcounter{exx}{0}
\begin{exe}
    \item In a medium bowl, season beef with salt and pepper
    \item Cook bacon in dutch oven over medium heat until crisp ($\sim$11 min), then transfer to bowl.
    \item On medium-high, brown beef ($\sim 5$min/side). Transfer to bacon bowl.
    \item Over medium, brown onion until brown ($\sim$2-3 min/side).
    \item Add carrots, garlic, thyme, and miso. Season with salt. Cook until garlic is just brown ($\sim$2 min).
    \item Add flour to vegetables and stir to coat ($\sim$30 seconds).
    \item Add wine, scraping to deglaze, and simmer until alcohol has burned off ($\sim$2 min)
    \item Add beef, bacon, and water to submerge ($\sim$2 cups). Return to simmer, then cover and bake in oven at 300$^\circ$ for 1.5 hours
    \item Return to stove and cook uncovered over medium-low until beef is tender (30-60 min)
    \item For gremolata, combine parsley and reserved garlic (grated)
\end{exe}
\rule{\textwidth}{0.4pt}
\begin{enumerate}[]
    \item \textbf{Notes:}
    \begin{enumerate}[-]
    \item Serve topped with gremolata alongside crusty bread
    \item Makes 4 servings
    \end{enumerate}
\end{enumerate}
\end{minipage}}
\vskip 0.025in
\rule{\textwidth}{0.4pt}
\vfill

\subsection{Creamy Chickpeas with Eggs and Prosciutto}
\textit{Andy Baraghani, Bon Appétit (March 2018)}

\rule{\textwidth}{0.4pt}
\adjustbox{valign=t}{\begin{minipage}{0.3\linewidth}
\begin{enumerate}[]
    \item $2$ tbsp olive oil
    \item $2$ cloves garlic, thinly sliced
    \item $1$ sprig rosemary (optional)
    \item $15$ oz chickpeas, rinsed
    \item $1$ tbsp tomato paste
    \item $1$ tsp sweet smoked paprika
    \item $1$ cup heavy cream
    \item $2$ eggs
    \item $4$ slices prosciutto
    \item flatbread (for serving)
\end{enumerate}
\end{minipage}}
\adjustbox{valign=t}{\begin{minipage}{0.03\linewidth}
\hfill
\end{minipage}}
\adjustbox{valign=t}{\begin{minipage}{0.66\linewidth}
\setcounter{exx}{0}
\begin{exe}
    \item Heat oil in medium skillet over medium heat
    \item Cook garlic and rosemary, tossing until rosemary crisps (30 sec) and garlic is golden (2 min); reserve rosemary
    \item Stir in tomato paste, paprika, and chickpeas and cook to coat
    \item Stir in cream, season with salt and pepper and simmer
    \item Crack eggs into chickpeas and season with salt
    \item Cover and cook till whites are set but yolk is runny (2 min)
    \item Remove from heat and drape prosciutto around eggs
    \item Garnish with reserved rosemary, serve with flatbread
\end{exe}
\rule{\textwidth}{0.4pt}
\begin{enumerate}[]
    \item \textbf{Notes:}
    \begin{enumerate}[-]
        \item For individual portions, spoon chickpeas into ramekins, then add egg. Cook at 350$\degree$F in the oven, covered with a baking sheet
        \item Makes 2 servings
    \end{enumerate}
\end{enumerate}
\end{minipage}}
\vskip 0.025in
\rule{\textwidth}{0.4pt}
\vfill
\newpage

\subsection{Cacio e Pepe Rösti}
\textit{Christina Chaey, Bon Appétit (October 2021}

\rule{\textwidth}{0.4pt}
\adjustbox{valign=t}{\begin{minipage}{0.3\linewidth}
\begin{enumerate}[]
    \item $3$ lbs russett potatoes, peeled
    \item $1\sfrac{1}{2}$ tsp kosher salt
    \item $3$ oz parmesan, grated
    \item $2$ tsp freshly ground black pepper
    \item $6$ tbsp olive oil
\end{enumerate}
\end{minipage}}
\adjustbox{valign=t}{\begin{minipage}{0.03\linewidth}
\hfill
\end{minipage}}
\adjustbox{valign=t}{\begin{minipage}{0.66\linewidth}
\setcounter{exx}{0}
\begin{exe}
    \item Starting from cold, bring potatoes to boil in salted water. Simmer until tender on the outside, but very firm in the middle ($\sim$8-10 min). Drain and fully cool. Coarsely grate into a large bowl and toss with parmesan, salt, and pepper.
    \item Heat 4 tbsp. oil over medium. Add potatoes and spread evenly. Cook, pressing down on potatoes and tucking in edges until underside is golden brown ($\sim$ 7–9 min).
    \item Invert rösti onto a plate. Set pan over medium heat, add remaining 2 tbsp oil, and slide in rösti, crispy side up. Cook until second side is golden brown, ($\sim$5–7 min). Slide rösti onto plate and top with more parmesan and pepper.
\end{exe}
\end{minipage}}
\rule{\textwidth}{0.4pt}
\begin{enumerate}[]
    \item \textbf{Notes:}
    \begin{enumerate}[-]
        \item Potatoes can be boiled 1 day ahead. Let cool; cover and chill.
        \item Reheat from room temperature at 300$\degree$ for 10 minutes
        \item Makes 6-8 servings
    \end{enumerate}
\end{enumerate}
\vskip 0.025in
\rule{\textwidth}{0.4pt}

\vfill

\subsection{Extremely Cheesy Potato Gratin}
\textit{Carla Lalli Music, Bon Appétit (December 2020}

\rule{\textwidth}{0.4pt}
\adjustbox{valign=t}{\begin{minipage}{0.3\linewidth}
\begin{enumerate}[]
    \item $3$ oz sharp white cheddar (not aged), grated
    \item $3$ oz gruyère, grated
    \item $3$ lbs russet potatoes, peeled and sliced ($\sfrac{1}{4}\dprime$)
    \item $\sfrac{1}{2}$ large yellow onion, sliced
    \item $2$ tsp freshly salt
    \item $\sfrac{1}{2}$ tsp cayenne
    \item $1\sfrac{1}{4}$ cup heavy cream
    \item Freshly ground black pepper
    \end{enumerate}
\end{minipage}}
\adjustbox{valign=t}{\begin{minipage}{0.03\linewidth}
\hfill
\end{minipage}}
\adjustbox{valign=t}{\begin{minipage}{0.66\linewidth}
\setcounter{exx}{0}
\begin{exe}
    \ex Butter one side of a large square of foil. Line a large rimmed baking sheet with another sheet of foil. Place a rack in upper third of oven; preheat to 400$\degree$.
    \ex Toss cheddar and gruyère to combine
    \ex Toss potatoes, onion, salt, pepper, and cayenne to combine
    \item Arrange potatoes vertically in baking dish, tucking onions into crevices. Place on prepared baking sheet and pour in cream. Top with half of the cheese mixture, cover with foil, and bake until potatoes are tender ($\sim$60 min).
    \item Remove foil, top potatoes with remaining cheese, and broil until browned and bubbling. Let sit 10 minutes before serving.
\end{exe}
\end{minipage}}
\rule{\textwidth}{0.4pt}
\begin{enumerate}[]
    \item \textbf{Notes:}
    \begin{enumerate}[-]
        \item Makes 6 servings
    \end{enumerate}
\end{enumerate}
\vskip 0.025in
\rule{\textwidth}{0.4pt}

\vfill

\subsection{Steak au Poivre}
\textit{Molly Baz, Bon Appétit (May 2021}

\rule{\textwidth}{0.4pt}
\adjustbox{valign=t}{\begin{minipage}{0.3\linewidth}
\begin{enumerate}[]
    \item $2$ $1\sfrac{1}{2}\dprime$ New York strip steak
    \item $1$ tbsp whole black peppercorn
    \item $2$ tbsp vegetable oil
    \item $4$ cloves garlic, $2$ smashed, $2$ sliced
    \item $3$ sprigs thyme
    \item $3$ tbsp butter
    \item $1$ shallot, chopped
    \item $\sfrac{1}{3}$ cup cognac, dry sherry, or brandy
    \item $\sfrac{1}{2}$ cup heavy cream
    \end{enumerate}
\end{minipage}}
\adjustbox{valign=t}{\begin{minipage}{0.03\linewidth}
\hfill
\end{minipage}}
\adjustbox{valign=t}{\begin{minipage}{0.66\linewidth}
\setcounter{exx}{0}
\begin{exe}
    \ex Pat steaks dry and generously season with salt and pepper. Let sit 15 minutes.
    \ex Coarsely crush peppercorns in a mortar and pestle.
    \ex Heat oil over medium high. Sear steaks until a crust forms ($\sim$3 minutes per side).
    \ex Reduce heat and add 1 tbsp butter, smashed garlic, and thyme. Baste steak until cooked to desired finish ($\sim$2 minutes for medium rare). Remove steak from heat and let rest 10 minutes.
    \ex Combine remaining butter, garlic, and shallot in skillet. Cook until slightly softened, but not browned ($\sim$5 minutes). Add cognac (off heat) and cook until mostly evaporated ($\sim$2 minutes). Add cream and bring to a simmer. Cook until slightly thickened ($sim$1 minute).
    \ex Slice steak against the grain. Serve topped with sauce.
\end{exe}
\end{minipage}}
\rule{\textwidth}{0.4pt}
\begin{enumerate}[]
    \item \textbf{Notes:}
    \begin{enumerate}[-]
        \item Makes 4 servings
    \end{enumerate}
\end{enumerate}
\vskip 0.025in
\rule{\textwidth}{0.4pt}

\vfill

\subsection{Citrus \& Peppercorn Dry Brine}
\textit{Alison Roman, Bon Appétit (November 2013)}

\rule{\textwidth}{0.4pt}
\adjustbox{valign=t}{\begin{minipage}{0.3\linewidth}
\begin{enumerate}[]
    \item $2$ tbsp black peppercorn
    \item $1$ tbsp pink peppercorn
    \item $2$ tsp white peppercorn
    \item $2$ tsp coriander seed
    \item $6$ bay leaves
    \item $2$ tbsp light brown sugar
    \item $4$ tbsp lemon zest
    \item $2$ tbsp orange zest
    \item $\sfrac{1}{2}$ cup kosher salt
\end{enumerate}
\end{minipage}}
\adjustbox{valign=t}{\begin{minipage}{0.03\linewidth}
\hfill
\end{minipage}}
\adjustbox{valign=t}{\begin{minipage}{0.66\linewidth}
\setcounter{exx}{0}
\begin{exe}
    \item Toast peppercorns, coriander, and bay leaves until fragrant. Cool, and roughly crush
    \item Mix crushed spices with sugar, citrus zest, and sugar
    \item Apply brine to poultry and let sit, uncovered, in refrigerator over night
    \item Rinse poultry and pat dry before roasting
\end{exe}
\rule{\textwidth}{0.4pt}
\begin{enumerate}[]
    \item \textbf{Notes:}
    \begin{enumerate}[-]
        \item For best results, massage brine under skin
        \item Brine may be made 3 days ahead and chilled
        \item Makes enough for a 12-14 lb turkey
    \end{enumerate}
\end{enumerate}
\end{minipage}}
\vskip 0.025in
\rule{\textwidth}{0.4pt}

\vfill

\subsection{Spicy Fennel Tomato Sauce}
\textit{Alison Roman, Bon Appétit (November 2013)}

\rule{\textwidth}{0.4pt}
\adjustbox{valign=t}{\begin{minipage}{0.3\linewidth}
\begin{enumerate}[]
    \item $\sfrac{1}{4}$ cup olive oil
    \item $4$ cloves garlic, minced
    \item $\sfrac{1}{2}$ yellow onion, diced
    \item $1\sfrac{1}{2}$ tsp fennel seed
    \item $1\sfrac{1}{2}$ tsp red pepper flake
    \item $2$ tsp kosher salt
    \item $8$ medium tomatoes, diced
\end{enumerate}
\end{minipage}}
\adjustbox{valign=t}{\begin{minipage}{0.03\linewidth}
\hfill
\end{minipage}}
\adjustbox{valign=t}{\begin{minipage}{0.66\linewidth}
\setcounter{exx}{0}
\begin{exe}
    \ex Heat oil on medium heat. Add garlic and onion and reduce to low heat. Cook until just caramelised ($\sim$ 15 min).
    \ex Add fennel, pepper, and salt; cook until fragrant ($\sim$5 min).
    \ex Increase heat and add tomato. Bring to a boil, then reduce to lowest setting and cover. Cook until tomatoes are completely broken down and flavors have melded ($\sim$4-6 hours).
    \ex Uncover and blend thoroughly. Return to medium-low and reduce until slightly thickened. Serve over pasta of choice.
\end{exe}
\rule{\textwidth}{0.4pt}
\begin{enumerate}[]
    \item \textbf{Notes:}
    \begin{enumerate}[-]
        \item makes enough for 2-4 servings of pasta
    \end{enumerate}
\end{enumerate}
\end{minipage}}
\vskip 0.025in
\rule{\textwidth}{0.4pt}

\vfill

\newpage

\section{Starters}
\subsection{Smoked Trout Rillettes}
\textit{Molly Stevens, Epicurious (December 2007)}

\rule{\textwidth}{0.4pt}
\adjustbox{valign=t}{\begin{minipage}{0.3\linewidth}
\begin{enumerate}[]
    \item $\sfrac{3}{4}$ cup mascarpone
    \item $\sfrac{1}{4}$ cup green onion, finely chopped
    \item $2$ tbsp fresh dill, chopped
    \item $1$\sfrac{1}{2} tbsp lemon juice
    \item $7$ boneless, skinless smoked trout, flaked
\end{enumerate}
\end{minipage}}
\adjustbox{valign=t}{\begin{minipage}{0.03\linewidth}
\hfill
\end{minipage}}
\adjustbox{valign=t}{\begin{minipage}{0.66\linewidth}
\setcounter{exx}{0}
\begin{exe}
    \ex Combine mascarpone, green onion, dill, and lemon juice
    \ex Gently fold in smoked trout
\end{exe}
\rule{\textwidth}{0.4pt}
\begin{enumerate}[]
    \item \textbf{Notes:}
    \begin{enumerate}[-]
        \item Rillettes can be made 3 days in advance
        \item Serve with rye crackers and radishes
        \item Makes 10 servings
    \end{enumerate}
\end{enumerate}
\end{minipage}}
\vskip 0.025in
\rule{\textwidth}{0.4pt}

\vfill

\subsection{Smoked Trout Caraway Rillettes}
\textit{Kay Chun, Food \textit{\&} Wine (November 2015)}

\rule{\textwidth}{0.4pt}
\adjustbox{valign=t}{\begin{minipage}{0.3\linewidth}
\begin{enumerate}[]
    \item $1\sfrac{1}{2}$ cup lebneh
    \item $2$ tbsp minced shallot
    \item $3$ tbsp olive oil
    \item $2$ tsp caraway seeds
    \item $3$ scallions, thinly sliced
    \item $12$ oz smoked trout, meat flaked into large pieces
\end{enumerate}
\end{minipage}}
\adjustbox{valign=t}{\begin{minipage}{0.03\linewidth}
\hfill
\end{minipage}}
\adjustbox{valign=t}{\begin{minipage}{0.66\linewidth}
\setcounter{exx}{0}
\begin{exe}
    \ex Combine lebneh, shallot, olive oil, caraway, and scallion. Season with salt and pepper and fold in flaked trout. Serve at room temperature with pickled radishes and rye crackers.
\end{exe}
\rule{\textwidth}{0.4pt}
\begin{enumerate}[]
    \item \textbf{Notes:}
    \begin{enumerate}[-]
        \item Makes $2\sfrac{1}{2}$ cups
        \item Rillettes can be made 3 days in advance
        \item Crème fraîche or mascarpone may be substituted for lebneh
    \end{enumerate}
\end{enumerate}
\end{minipage}}
\vskip 0.025in
\rule{\textwidth}{0.4pt}

\vfill

\subsection{Spicy Quick-Pickled Radishes}
\textit{Kay Chun, Food \textit{\&} Wine (November 2015)}

\rule{\textwidth}{0.4pt}
\adjustbox{valign=t}{\begin{minipage}{0.3\linewidth}
\begin{enumerate}[]
    \item $1$ lb radishes, with greens, halved lengthwise
    \item $12$ sprigs dill
    \item $1$\sfrac{1}{2} cups white vinegar
    \item $5$ cloves garlic
    \item $5$ chiles de árbol, halved
    \item $3$ tbsp kosher salt
    \item $1$ tbsp sugar
\end{enumerate}
\end{minipage}}
\adjustbox{valign=t}{\begin{minipage}{0.03\linewidth}
\hfill
\end{minipage}}
\adjustbox{valign=t}{\begin{minipage}{0.66\linewidth}
\setcounter{exx}{0}
\begin{exe}
    \ex Pack the radishes and dill into a 1 quart jar.
    \ex Bring the vinegar, garlic, chiles, salt, sugar, and $\sfrac{1}{2}$ cup water to a boil. Pour over radishes and let cool.
\end{exe}
\rule{\textwidth}{0.4pt}
\begin{enumerate}[]
    \item \textbf{Notes:}
    \begin{enumerate}[-]
        \item Makes 1 quart
        \item Radishes can be refrigerated in brine for 3 days
    \end{enumerate}
\end{enumerate}
\end{minipage}}
\vskip 0.025in
\rule{\textwidth}{0.4pt}

\vfill

\newpage

\subsection{Curried Egg Tartines}
\textit{Zaynab Issa, Bon Appétit (November 2021)}

\rule{\textwidth}{0.4pt}
\adjustbox{valign=t}{\begin{minipage}{0.3\linewidth}
\begin{enumerate}[]
    \item $8$ large eggs
    \item $3$ radishes, thinly sliced
    \item $1$ cup parsley/chives/dill coarsely chopped
    \item $1$ lemon, juiced
    \item $5$ scallions
    \item $3$ tbsp olive oil
    \item $4$ slices country bread
    \item $1\sfrac{1}{2}$ tsp curry powder
    \item $\sfrac{1}{2}$ cup Kewpie mayonnaise
    \item $1$ tbsp dijon mustard
    \item Aleppo-style pepper
\end{enumerate}
\end{minipage}}
\adjustbox{valign=t}{\begin{minipage}{0.03\linewidth}
\hfill
\end{minipage}}
\adjustbox{valign=t}{\begin{minipage}{0.66\linewidth}
\setcounter{exx}{0}
\begin{exe}
    \ex Hard boil eggs ($\sim$9 min). Chill in ice bath until cool.
    \ex Meanwhile, combine radishes, herbs, and $\sfrac{1}{2}$ tsp salt in a medium bowl. Drizzle lemon juice over and toss to combine.
    \ex Thinly slice scallions, separating greens from whites.
    \ex Peel and coarsely chop eggs, keeping pieces large.
    \ex Heat 1 tbsp oil and cook curry powder and reserved scallion white, stirring occasionally, until slightly softened and fragrant ($\sim$3 min). Transfer to a large bowl, add mayonnaise, mustard, and remaining $\sfrac{1}{2}$ tsp salt, mixing to combine. Add reserved scallion greens and fold in eggs.
    \ex Heat 1 tbsp oil over medium. Add 2 slices of bread and cook until golden and crisp underneath, about 4 minutes. Transfer to plates and arrange fried side up. Repeat with another 1 Tbsp. oil and remaining slices of bread.
    \ex To serve, top fried bread with egg salad and herb salad. Sprinkle with sea salt and Aleppo-style pepper.
\end{exe}
\end{minipage}}
\vskip 0.025in
\rule{\textwidth}{0.4pt}
\begin{enumerate}[]
    \item \textbf{Notes:}
    \begin{enumerate}[-]
        \item Egg salad can be made 3 days ahead. Cover and chill.
        \item Makes 4 servings.

    \end{enumerate}
\end{enumerate}
\rule{\textwidth}{0.4pt}

\vfill

\subsection{Peppery Antipasto Skewers}
\textit{Zaynab Issa, Bon Appétit (November 2021)}

\rule{\textwidth}{0.4pt}
\adjustbox{valign=t}{\begin{minipage}{0.3\linewidth}
\begin{enumerate}[]
    \item $1\sfrac{1}{2}$ lbs boneless rib-eye, room temperature
    \item $3$ tbsp vegetable oil
    \item $8$ oz semifirm cheese (e.g. comté), cubed ($
    \sfrac{3}{4}\dprime$)
    \item $4$ oz pickled cherry peppers or peperoncini, cubed ($
    \sfrac{3}{4}\dprime$)
    \item Salt
    \item Pepper
    \item Olive oil
\end{enumerate}
\end{minipage}}
\adjustbox{valign=t}{\begin{minipage}{0.03\linewidth}
\hfill
\end{minipage}}
\adjustbox{valign=t}{\begin{minipage}{0.66\linewidth}
\setcounter{exx}{0}
\begin{exe}
    \ex Pat steak dry and generously season with salt and pepper. Let sit 15 minutes.
    \ex Heat oil in a medium skillet over medium-high. Sear steak until crus forms ($\sim$3 minutes per side). Let rest 15 minutes and slice against the grain $\sfrac{1}{8}\dprime - \sfrac{1}{4}\dprime$ thick.
    \ex Wrap one cheese and one pepper cube in a slice in steak. Thread onto a skewer and place on a platter. Drizzle with olive oil; garnish with flaky salt and fresh-cracked pepper.
\end{exe}
\rule{\textwidth}{0.4pt}
\begin{enumerate}[]
    \item \textbf{Notes:}
    \begin{enumerate}[-]
        \item Makes 4-6 servings
    \end{enumerate}
\end{enumerate}
\end{minipage}}
\vskip 0.025in
\rule{\textwidth}{0.4pt}

\vfill

\newpage

\section{Salads}
\subsection{Orzo Salad with Tomatoes, Feta, \textit{\&} Green Onion}
\textit{Giada de Laurentiis, Bon Appétit (April 2006)}

\rule{\textwidth}{0.4pt}
\adjustbox{valign=t}{\begin{minipage}{0.3\linewidth}
\begin{enumerate}[]
    \item \textbf{Dressing:}
    \begin{enumerate}[]
    \item $\sfrac{1}{4}$ cup red wine vinegar
    \item $2$ tbsp fresh lemon juice
    \item $1$ tsp honey
    \item $\sfrac{1}{2}$ cup olive oil
    \end{enumerate}
\end{enumerate}
\begin{enumerate}[]
    \item \textbf{Salad:}
    \begin{enumerate}[]
    \item 1 lb orzo
    \item 6 cups stock
    \item 2 cup yellow and red grape tomatoes, halved
    \item 7 oz feta, diced or crumbled
    \item 1 cup basil, chopped
    \item 1 cup sliced scallions
    \item $\sfrac{1}{2}$ cup pine nuts, toasted
    \end{enumerate}
\end{enumerate}
\end{minipage}}
\adjustbox{valign=t}{\begin{minipage}{0.03\linewidth}
\hfill
\end{minipage}}
\adjustbox{valign=t}{\begin{minipage}{0.66\linewidth}
\setcounter{exx}{0}
\begin{exe}
    \ex Combine vinegar, lemon juice, and honey, then whisk in olive oil. Season with salt and pepper.
    \ex Bring stock to boil, then add orzo. Cook until just barely al dente.
    \ex Drain orzo and transfer to bowl. Let cool, stirring frequently.
    \ex Once orzo is cooled, toss with tomatoes, feta, basil, scallions, and pine nuts
    \item Add dressing and toss
\end{exe}
\rule{\textwidth}{0.4pt}
\begin{enumerate}[]
    \item \textbf{Notes:}
    \begin{enumerate}[-]
    \item Dressing can be made two days ahead; keep chilled.
    \item Salad keeps up to two hours at room temperature. Pine nuts may be omitted.
    \item Makes 8 servings (generous)
    \end{enumerate}
\end{enumerate}
\end{minipage}}
\vskip 0.025in
\rule{\textwidth}{0.4pt}

\vfill

\subsection{Warm Onion Vinaigrette}
\textit{A.J.R.}

\rule{\textwidth}{0.4pt}
\adjustbox{valign=t}{\begin{minipage}{0.3\linewidth}
\begin{enumerate}[]
    \item $1$ red onion, halved and sliced lengthwise ($\sfrac{1}{2}$ in)
    \item $5$ cloves garlic, pressed
    \item $2$ tbsp olive oil
    \item $3$ tbsp white balsamic vinegar
    \item $1\sfrac{1}{2}$ tbsp soy sauce
\end{enumerate}
\end{minipage}}
\adjustbox{valign=t}{\begin{minipage}{0.03\linewidth}
\hfill
\end{minipage}}
\adjustbox{valign=t}{\begin{minipage}{0.66\linewidth}
\setcounter{exx}{0}
\begin{exe}
    \item Heat oil and sweat the onion until soft, but not carmalized
    \item Add vinegar, soy sauce, and garlic and cook until reduced
    \item Cover and let marinate for at least five minutes
\end{exe}
\rule{\textwidth}{0.4pt}
\begin{enumerate}[]
    \item \textbf{Notes:}
    \begin{enumerate}[-]
    \item Dressing pairs well with arugula and other astringent greens
    \item Makes 4-6 servings
    \end{enumerate}
\end{enumerate}
\end{minipage}}
\vskip 0.025in
\rule{\textwidth}{0.4pt}
\vfill
\newpage

\subsection{Arrugula Salad with Lemon Parmesan Dressing}
\textit{Tori Ritchie, Bon Appétit (\textit{April 2009)}}

\rule{\textwidth}{0.4pt}
\adjustbox{valign=t}{\begin{minipage}{0.3\linewidth}
\begin{enumerate}[]
    \item \textbf{Dressing:}
    \begin{enumerate}[]
    \item $\sfrac{1}{3}$ cup grated parmesan
    \item $5$ tbsp olive oil
    \item $2$ tbsp lemon juice
    \item $1$ tsp lemon zest
    \item $2$ cloves garlic
    \end{enumerate}
    \item \textbf{Salad:}
    \begin{enumerate}[]
    \item $4$ cups arugula
    \item $1$ cup cherry tomatoes
    \end{enumerate}
\end{enumerate}
\end{minipage}}
\adjustbox{valign=t}{\begin{minipage}{0.03\linewidth}
\hfill
\end{minipage}}
\adjustbox{valign=t}{\begin{minipage}{0.66\linewidth}
\setcounter{exx}{0}
\begin{exe}
    \item Blend dressing ingredients in a food processor, season with salt and pepper
    \item Cut tomatoes in half lengthwise and combine with arugula, toss with dressing to coat
\end{exe}
\rule{\textwidth}{0.4pt}
\begin{enumerate}[]
    \item \textbf{Notes:}
    \begin{enumerate}[-]
        \item Dressing can be kept refrigerated for up to three days
        \item Makes 4 servings
    \end{enumerate}
\end{enumerate}
\end{minipage}}
\vskip 0.025in
\rule{\textwidth}{0.4pt}
\vfill

\subsection{Fennel-Celery Salad with Blue Cheese and Walnuts}
\textit{Claire Saffitz, Bon Appétit (\textit{November 2016)}}

\rule{\textwidth}{0.4pt}
\adjustbox{valign=t}{\begin{minipage}{0.3\linewidth}
\begin{enumerate}[]
    \item $\sfrac{1}{2}$ cup chopped walnuts
    \item $1$ shallot, halved lengthwise, divided
    \item $2$ tbsp wholegrain mustard
    \item $\sfrac{1}{2}$ cup sherry vinegar
    \item $\sfrac{1}{3}$ cup olive oil
    \item kosher salt
    \item black pepper
    \item $6$ dried Turkish figs, coarsely chopped
    \item $1$ fennel bulb, cored, thinly sliced
    \item $6$-$8$ celery stalks, thinly sliced
    \item $4$ oz blue cheese
\end{enumerate}
\end{minipage}}
\adjustbox{valign=t}{\begin{minipage}{0.03\linewidth}
\hfill
\end{minipage}}
\adjustbox{valign=t}{\begin{minipage}{0.66\linewidth}
\setcounter{exx}{0}
\begin{exe}
    \item Toast walnuts at $350^\degree$F until golden ($\sim 5$-$8$ min)
    \item Thinly slice half shallot. Combine with figs and $\sfrac{1}{4}$ cup vinegar. Let stand until softened ($\sim 30$ min)
    \item Finely chop remaining half shallot. Combine with mustard, sugar, and remaining $\sfrac{1}{4}$ cup vinegar in resealable jar. Add oil and season with salt and pepper. Shake to emulsify.
    \item Immediately prior to serving, toss fennel, celery, blue cheese, and walnuts in large bowl. Drizzle with vinaigrette and toss to coat. Season with salt and pepper.
\end{exe}
\rule{\textwidth}{0.4pt}
\begin{enumerate}[]
    \item \textbf{Notes:}
    \begin{enumerate}[-]
        \item Vinaigrette can be made 1 day ahead, cover and chill.
        \item Figs and shallot can be soaked 4 hours ahead, store covered at room temperature.
        \item Fennel and celery can be sliced 4 hours ahead, cover with damp towel and chill.
        \item Makes 8 servings
    \end{enumerate}
\end{enumerate}
\end{minipage}}
\vskip 0.025in
\rule{\textwidth}{0.4pt}
\vfill

\subsection{Herby Barley Salad with Butter-basted Mushrooms}
\textit{Chris Morocco, Bon Appétit (\textit{November 2015)}}

\rule{\textwidth}{0.4pt}
\adjustbox{valign=t}{\begin{minipage}{0.3\linewidth}
\begin{enumerate}[]
    \item $1$ cup hulled/hull-less/pearl barley
    \item kosher salt
    \item black pepper
    \item $2$ shallots, thinly sliced into rings
    \item $\sfrac{1}{3}$ cup vegetable oil
    \item $2$ tbsp olive oil
    \item $8$ oz mixed mushrooms, torn into large pieces
    \item $2$ sprigs thyme
    \item $3$ tbsp unsalted butter
    \item $1$ cup cilantro, chopped
    \item $1$ cup parsley, chopped
    \item $4$ oz blue cheese
    \item $2$ tbsp lemon juice
    \item $1\sfrac{1}{2}$ oz parmesan, shaved, plus more for serving
\end{enumerate}
\end{minipage}}
\adjustbox{valign=t}{\begin{minipage}{0.03\linewidth}
\hfill
\end{minipage}}
\adjustbox{valign=t}{\begin{minipage}{0.66\linewidth}
\setcounter{exx}{0}
\begin{exe}
    \item Boil barley in a medium pot of salted water until tender, 50–60 minutes for hulled or hull-less, 20–30 minutes for pearl. Drain; spread out on a baking sheet and let cool.
    \item Cook shallots in vegetable oil in over medium-high heat, until golden brown ($\sim 5$–$7$ min). Transfer shallots to paper towels and season with salt. Reserve cooking oil.
    \item Heat olive oil over medium-high until smoking. Cook mushrooms in single layer, undisturbed, until undersides are golden brown, ($\sim 3$ min). Season with salt/pepper, toss, and continue cooking until golden brown all over ($\sim 5$ min).
    \item Reduce heat to medium and add thyme, garlic, and butter. Tip skillet use spoon to baste mushrooms with foaming butter; cook until butter smells nutty. Transfer mushrooms to a small bowl, leaving thyme and garlic behind.
    \item Toss barley, cilantro, parsley, lemon juice, parmesan, and 2 tbsp shallot oil; season with salt/pepper. Add mushrooms.
    \item Serve topped with fried shallots and more shaved parmesan.
\end{exe}
\rule{\textwidth}{0.4pt}
\begin{enumerate}[]
    \item \textbf{Notes:}
    \begin{enumerate}[-]
        \item Barley can be cooked 1 day ahead and chilled.
        \item Dish can be made 3 hours ahead; store at room temperature.
        \item Makes 8 servings
    \end{enumerate}
\end{enumerate}
\end{minipage}}
\vskip 0.025in
\rule{\textwidth}{0.4pt}
\vfill
\newpage

\section{Baking}
\subsection{Rough Puff Pastry}
\textit{Gourmet (October 2004)}

\rule{\textwidth}{0.4pt}
\adjustbox{valign=t}{\begin{minipage}{0.3\linewidth}
\begin{enumerate}[]
    \item $1\sfrac{1}{4}$ cup flour
    \item $\sfrac{1}{4}$ tsp salt
    \item $13$ tbsp butter, frozen
    \item $5$ tbsp ice water
\end{enumerate}
\end{minipage}}
\adjustbox{valign=t}{\begin{minipage}{0.03\linewidth}
\hfill
\end{minipage}}
\adjustbox{valign=t}{\begin{minipage}{0.66\linewidth}
\setcounter{exx}{0}
\begin{exe}
    \item Sift flour and salt into a chilled bowl
    \item Coarsely grate butter into flour mixture, lifting flour to coat
    \item Drizzle water over mixture and stir with fork to incorporate
    \item Test by squeezing handful (should hold without crumbling)
    \item If necessary, add another tablespoon of water
    \item Gather mixture together into a 5 inch cube, then chill in plastic wrap  until firm (30 minutes)
    \item Roll chilled dough into $15\times 8$ inch rectangle, then letter fold
    \item Chill layered dough in plastic wrap 30 minutes
    \item Repeat process two more times
    \item Let final dough chill 1 hour before use
\end{exe}
\end{minipage}}
\vskip 0.025in
\rule{\textwidth}{0.4pt}
\begin{enumerate}[]
    \item \textbf{Notes:}
    \begin{enumerate}[-]
    \item Do not over work dough, nor add too much water (pastry will be tough)
    \item Makes about 1 lb
    \end{enumerate}
\end{enumerate}
\rule{\textwidth}{0.4pt}
\vfill

\subsection{Pastry Dough}
\textit{Gourmet (October 2004)}

\rule{\textwidth}{0.4pt}
\adjustbox{valign=t}{\begin{minipage}{0.3\linewidth}
\begin{enumerate}[]
    \item $1\sfrac{1}{4}$ cup flour
    \item $\sfrac{1}{4}$ tsp salt
    \item $6$ tbsp cold butter, cubed
    \item $2$ tbsp cold shortening
    \item $3$-$4$ tbsp ice water
\end{enumerate}
\end{minipage}}
\adjustbox{valign=t}{\begin{minipage}{0.03\linewidth}
\hfill
\end{minipage}}
\adjustbox{valign=t}{\begin{minipage}{0.66\linewidth}
\setcounter{exx}{0}
\begin{exe}
    \item Sift flour and salt in a large bowl
    \item Blend in butter and shortening until mixture resembles coarse meal with small (pea sized) butter lumps.
    \item Drizzle $3$ tbsp water over mixture and stir to incorporate
    \item Test by squeezing handful (should hold without crumbling).
    \item Turn out mixture and divide in $4$. With the heel of your hand, smear portions in a forward motion to distribute fat.
    \item Gather dough, press into ball, and flatten into a 5-inch disk.
    \item Chill dough in plastic wrap until firm, at least 1 hour.
\end{exe}
\end{minipage}}
\vskip 0.025in
\rule{\textwidth}{0.4pt}
\begin{enumerate}[]
    \item \textbf{Notes:}
    \begin{enumerate}[-]
    \item Use your fingers, a pastry cutter, or a food processor to blend fat into flour
    \item Do not over work dough, nor add too much water (pastry will be tough)
    \item Makes enough dough for a $9$-$10$ in round shell
    \end{enumerate}
\end{enumerate}
\rule{\textwidth}{0.4pt}
\vfill
\newpage

\subsection{Tomato Lemon Tart}
\textit{Josef Centeno, Bon appétit (September 2017)}

\rule{\textwidth}{0.4pt}
\adjustbox{valign=t}{\begin{minipage}{0.3\linewidth}
\begin{enumerate}[]
    \item $14$ oz puff pastry
    \item $1$ clove garlic, grated
    \item $1$ tbsp olive oil
    \item $1$ lemon, thinly sliced
    \item $1$ cup basil, torn
    \item $1$ lb tomato, thinly sliced
\end{enumerate}
\end{minipage}}
\adjustbox{valign=t}{\begin{minipage}{0.03\linewidth}
\hfill
\end{minipage}}
\adjustbox{valign=t}{\begin{minipage}{0.66\linewidth}
\setcounter{exx}{0}
\begin{exe}
    \item On parchment, roll pastry out into a $15\times 10$ inch rectangle
    \item Prick pastry with fork, leaving $1$ inch border around edges
    \item Transfer pastry on parchment paper to baking sheet
    \item Whisk garlic and oil and brush over pastry, staying in border
    \item Arrange lemon slices in single layer on pastry; top with basil
    \item Arrange tomatoes on top of lemons, drizzle with olive oil, and season with salt and pepper
    \item Bake at 375$\degree$ until edges are puffed, and browned, ($\sim$35 min)
    \item Allow tart to rest 10 minutes, then slice
    \item Drizzle with olive olive oil and serve with crème fraîche
\end{exe}
\end{minipage}}
\vskip 0.025in
\rule{\textwidth}{0.4pt}
\begin{enumerate}[]
    \item \textbf{Notes:}
    \begin{enumerate}[-]
        \item Use a mandolin to get even $\sfrac{1}{4}$in slices on the lemons and tomato
        \item Pat lemon and tomato slices dry before adding to tart
        \item Makes 8 servings
    \end{enumerate}
\end{enumerate}
\rule{\textwidth}{0.4pt}
\vfill

\subsection{Shallot Tarte Tatin}
\textit{Bon Appétit (December 2016)}

\rule{\textwidth}{0.4pt}
\adjustbox{valign=t}{\begin{minipage}{0.3\linewidth}
\begin{enumerate}[]
    \item $14$ oz puff pastry
    \item $1$ tbsp pine nuts
    \item $6$ large shallots, halved lengthwise
    \item $2$ tsp vegetable oil
    \item kosher salt, black pepper
    \item $\sfrac{1}{4}$ balsamic vinegar
    \item $1$ tsp sugar
    \item $3$ tbsp butter, divided
    \item $8$ oz mixed mushrooms, torn
    \item $1$ garlic clove, smashed
    \item $3$ oz burrata, torn, or ricotta
    \item $1$ cup baby arrugula
    \item $\sfrac{1}{2}$ oz parmesan, shaved
    \item olive oil, lemon wedges
\end{enumerate}
\end{minipage}}
\adjustbox{valign=t}{\begin{minipage}{0.03\linewidth}
\hfill
\end{minipage}}
\adjustbox{valign=t}{\begin{minipage}{0.66\linewidth}
\setcounter{exx}{0}
\begin{exe}
    \item Toast nuts at $400^\degree$F until browned ($\sim4$ min), then reserve.
    \item Toss shallots and vegetable oil on baking sheet, season with salt and pepper. Roast until tender and browned ($\sim20$ min). Let cool and remove leathery outer layers.
    \item Bring vinegar and sugar to a simmer in a small skillet over medium-low, swirling until syrupy ($\sim5$ min). Stir in $1$ tbsp. butter; remove from heat. Arrange shallots, cut side up, in skillet, overlapping slightly.
    \item Roll out pastry and cut a circle $1$" larger than the skillet; prick with a fork. Drape over shallots, tucking edges inside skillet. Bake at $400^\degree$F until golden and puffed ($\sim30$ min).
    \item Cook mushrooms and garlic in $2$ tbsp butter on medium-high til browned, ($\sim5$-$8$ min). Season with salt and pepper.
    \item Cool slightly, invert, and top with burrata, mushrooms, arugula, parmesan, and nuts. Drizzle olive oil and lemon.
\end{exe}
\rule{\textwidth}{0.4pt}
\begin{enumerate}[]
    \item \textbf{Notes:}
    \begin{enumerate}[-]
        \item To make ahead: reheat skillet to soften glaze, then invert
        \item Makes 4 servings
    \end{enumerate}
\end{enumerate}
\end{minipage}}
\vskip 0.025in
\rule{\textwidth}{0.4pt}
\vfill
\newpage

\subsection{Wild Mushroom Tart}
\textit{Bon Appétit (December 2016)}

\rule{\textwidth}{0.4pt}
\adjustbox{valign=t}{\begin{minipage}{0.3\linewidth}
\begin{enumerate}[]
    \item pastry dough
    \item $1$ tbsp butter
    \item $1$ tsp vegetable oil
    \item $12$ oz mixed mushrooms, torn
    \item $2$ tbsp minced shallot
    \item $1$ tsp fresh thyme, chopped
    \item $\sfrac{3}{4}$ tsp salt
    \item $\sfrac{3}{8}$ tsp black pepper
    \item $\sfrac{1}{2}$ cup crème fraîche
    \item $\sfrac{1}{2}$ cup heavy cream
    \item $1$ large egg
    \item $1$ large egg yolk
    \item \textbf{Equipment:}
    \begin{enumerate}[]
        \item Pie weights (or raw rice)
        \item $9\times 1$ inch fluted round tart pan with removable bottom
    \end{enumerate}
\end{enumerate}
\end{minipage}}
\adjustbox{valign=t}{\begin{minipage}{0.03\linewidth}
\hfill
\end{minipage}}
\adjustbox{valign=t}{\begin{minipage}{0.66\linewidth}
\setcounter{exx}{0}
\begin{exe}
    \item Roll out dough into an 11 inch round and fit into tart pan, trimming excess. Chill until firm ($\sim30$ min).
    \item Prick pastry with fork, line with foil, and fill with pie weights. Bake at $375^\degree$F until edge is golden ($\sim18$ min).
    \item Remove foil and pie weights and bake until bottom is golden ($\sim10$ min). Transfer to rack and cool completely.
    \item Heat butter and oil over medium-high heat until foam subsides. Sauté mushrooms, shallot, thyme, $\sfrac{1}{2}$ tsp salt, and \sfrac{1}{4} tsp pepper, until mushrooms are tender and liquid is evaporated ($\sim8$ min). Transfer to bowl and cool completely.
    \item Whisk crème fraîche, heavy cream, whole egg, yolk, $\sfrac{1}{4}$ tsp salt and $\sfrac{1}{8}$ tsp pepper until combined.
    \item Scatter mushrooms in tart shell and pour custard over. Bake at $325^\degree$F on baking sheet until custard is just set and slightly puffed ($\sim35$ min).
    \item Cool tart in pan on rack at least 20 minutes. Serve tart warm or at room temperature.
\end{exe}
\rule{\textwidth}{0.4pt}
\begin{enumerate}[]
    \item \textbf{Notes:}
    \begin{enumerate}[-]
        \item Can definitely handle more mushrooms
        \item Makes 6-8 servings
    \end{enumerate}
\end{enumerate}
\end{minipage}}
\vskip 0.025in
\rule{\textwidth}{0.4pt}
\vfill

\subsection{Celariac and Sweet Garlic Pie}
\textit{Anna Jones, The Guardian (November 2021)}

\rule{\textwidth}{0.4pt}
\adjustbox{valign=t}{\begin{minipage}{0.3\linewidth}
\begin{enumerate}[]
    \item \textbf{Pastry:}
    \begin{enumerate}[]
        \item $250$ g plain spelt flour
        \item $125$ g butter, chilled and cubed
        \item $\sfrac{1}{2}$ tsp salt
        \item $2$ sprigs each rosemary, thyme, and sage, chopped
        \item $1$ lemon, zest of
        \item $25$ g cheddar, grated
        \item $1$ egg yolk
        \item $50-70$ g water, chilled
    \end{enumerate}
    \item \textbf{Filling and Assembly:}
    \begin{enumerate}[]
        \item $3$ heads garlic, peeled
        \item olive oil
        \item $1$ tsp balsamic vinegar
        \item $1$ tbsp honey
        \item $2$ sprigs each rosemary, thyme, and sage, chopped
        \item $1$ kg celariac, peeled and quartered
        \item $220$ g Lancashire cheese, crumbled
        \item $150$ g creme fraiche
        \item $\sfrac{1}{2}$ lemon, juice of
        \item $1$ tbsp wholegrain mustard
        \item $1$ bunch parsley, chopped
        \item $2$ eggs, beaten
        \item Worcestershire sauce
    \end{enumerate}
\end{enumerate}
\end{minipage}}
\adjustbox{valign=t}{\begin{minipage}{0.03\linewidth}
\hfill
\end{minipage}}
\adjustbox{valign=t}{\begin{minipage}{0.66\linewidth}
\setcounter{exx}{0}
\begin{exe}
    \ex Whisk together flour and salt, then cut in butter. Stir in the herbs, lemon zest and grated cheddar.
    \ex Beat the egg yolk with 1 tablespoon of cold water. Add to the flour and mix until it forms a dough. Add more water, a teaspoon at a time, until it comes together into a smooth dough. Wrap in clingfilm and chill.
    \ex Put the cloves of garlic into a saucepan, cover with cold water and bring to a gentle simmer. Cook for 2–3 minutes, then drain.
    \ex Wipe the saucepan dry. Add the garlic and 1 tablespoon of olive oil and fry on high for 2 minutes. Add the vinegar and 100ml water, bring to the boil and simmer gently for 10 minutes.
    \ex Add the honey, most of the rosemary and thyme (reserving the rest, with the sage) and a good pinch of salt. Continue cooking until most of the liquid has evaporated and the garlic cloves are coated in a dark syrup ($\sim$ 5 minutes).
    \ex Reserve 300g celariac and slice remainder into 2cm-thick pieces. Put them into a saucepan, cover with hot water and boil for 7–10 minutes, until soft and slightly translucent.
    \ex Combine drained celariac with the cheese, creme fraiche, lemon juice, mustard, parsley, a splash of Worcestershire and eggs. Season with salt and pepper and fold in garlic.
    \ex Preheat the oven to 180$\degree$C. Roll out the pastry to 3–4mm thick and line a 20cm-diameter cake tin, ensuring a little spills over the edges. Pour the filling into the pastry case.
    \ex Coarsely grate the reserved celeriac and pile it on top. Finish with the reserved rosemary and thyme, all the sage, and a drizzle of olive oil.
    \ex Bake for 45 minutes or until the tart filling has set and the top is golden brown.
    \ex Let cool slightly, then take remove from tin and garnish with a few herbs on fresh herbs to serve.
\end{exe}
\end{minipage}}
\vskip 0.025in
\rule{\textwidth}{0.4pt}
\begin{enumerate}[]
    \item \textbf{Notes:}
    \begin{enumerate}[-]
        \item Miso makes a good substitute for Worcestershire
        \item Good, crumbly cheddar may be substituted for Lancashire
        \item Pastry dough may be made a day in advance
        \item Makes 8-10 servings
    \end{enumerate}
\end{enumerate}
\vskip 0.025in
\rule{\textwidth}{0.4pt}
\vfill

\subsection{Samosa Pie}
\textit{Nadiya Hussain, New York Times (Adapted by Rachel Wharton)}

\rule{\textwidth}{0.4pt}
\adjustbox{valign=t}{\begin{minipage}{0.3\linewidth}
\begin{enumerate}[]
    \item \textbf{Filling:}
    \begin{enumerate}[]
        \item $5$ tbsp olive oil
        \item $1$ small white onion, chopped
        \item Kosher salt
        \item $1$ tsp ginger
        \item $1$ tbsp garlic granules
        \item $1$ tbsp cumin seeds
        \item $2$ tsp red pepper flakes
        \item $1$ lb ground lamb
        \item $1$ lb potatoes, cubed ($\sfrac{1}{2}\dprime$)
        \item $1$ cup peas (fresh or frozen)
        \item $\sfrac{3}{4}$ cilantro, chopped
        \item $2$ tbsp cornstarch whisked into $\sfrac{1}{4}$ cup water
    \end{enumerate}
    \item \textbf{Pastry:}
    \begin{enumerate}[]
        \item $265$ g all purpose flour
        \item $55$ g bread flour
        \item $1$ tsp kosher salt
        \item $1$ tsp turmeric
        \item $65$ g shortening
        \item $165$ g water
        \item $1$ egg, beaten
    \end{enumerate}
\end{enumerate}
\end{minipage}}
\adjustbox{valign=t}{\begin{minipage}{0.03\linewidth}
\hfill
\end{minipage}}
\adjustbox{valign=t}{\begin{minipage}{0.66\linewidth}
\setcounter{exx}{0}
\begin{exe}
    \ex Heat over medium heat; add onion and $1\sfrac{1}{2}$ tsp salt and cook, until soft and translucent ($\sim5$ min).
    \item Stir in ginger, garlic, cumin and red pepper. Cook until fragrant ($\sim2$ min).
    \item Add lamb and cook, breaking into small pieces, until just cooked through ($\sim7$ min).
    \item Stir in potatoes, cover, reduce heat to medium-low and let everything steam, stirring halfway through, until the potatoes are just soft ($\sim20$ min).
    \ex Uncover, remove from heat and stir in peas, cilantro, and cornstarch slurry. Cool completely.
    \ex While filling is cooling, preheat oven to 400$^\degree$F
    \ex To make pastry, whisk together flour, salt and turmeric in a large bowl
    \ex Bring shortening and water to boil until shortening melts, then pour into dry ingredients and mix quickly to combine
    \ex Once cool enough to handle, knead dough until smooth
    \ex Reserve $\sfrac{1}{3}$ of dough, and roll out remainder into $11\dprime$ circle ($\sim\sfrac{1}{8}\dprime$ thick). Transfer to an $8\dprime$ springform pan, pressing into the bottom and against the sides.
    \ex Add cooled filling to pastry shell, pressing down to smooth and compact.
    \ex Roll reserved dough into a $9\dprime$ circle ($\sim\sfrac{1}{8}\dprime$ thick) and place atop filling. Trim and crimp edges, and cut a steam vent. Brush top with beaten egg.
    \ex Bake until deep golden brown on top ($1-\sim1\sfrac{1}{4}$ hour). Allow to rest at least $1\sfrac{1}{2}$ hours before unmolding.
\end{exe}
\end{minipage}}
\vskip 0.025in
\rule{\textwidth}{0.4pt}
\begin{enumerate}[]
    \item \textbf{Notes:}
    \begin{enumerate}[-]
        \item Can replace bread flour with AP flour
        \item Can use butter or ghee in place of shortening
        \item Can substitute chicken, beef, or cauliflower for lamb
    \end{enumerate}
\end{enumerate}
\vskip 0.025in
\rule{\textwidth}{0.4pt}
\vfill

\subsection{Currant Scones}
\textit{Margaret Moon}

\rule{\textwidth}{0.4pt}
\adjustbox{valign=t}{\begin{minipage}{0.3\linewidth}
\begin{enumerate}[]
    \item $2$ cups flour
    \item $1$ tbsp baking powder
    \item $\sfrac{1}{2}$ tsp salt
    \item $\sfrac{1}{4}$ cup sugar
    \item $\sfrac{1}{2}$ cup currants
    \item $\sfrac{1}{3}$ cup butter
    \item $\sfrac{1}{4}$ cup shortening
    \item $\sfrac{1}{2}$ cup buttermilk
    \item $1$ egg
\end{enumerate}
\end{minipage}}
\adjustbox{valign=t}{\begin{minipage}{0.03\linewidth}
\hfill
\end{minipage}}
\adjustbox{valign=t}{\begin{minipage}{0.66\linewidth}
\setcounter{exx}{0}
\begin{exe}
    \item Mix together dry ingredients and cut in butter
    \item Add currents and mix to combine
    \item Beat buttermilk and egg and mix into dry ingredients
    \item Knead on floured board until dough comes together
    \item Roll 1 inch thick and cut into desired shape
    \item Bake at 375$\degree$  for 15-20 minutes
\end{exe}
\rule{\textwidth}{0.4pt}
\begin{enumerate}[]
    \item \textbf{Notes:}
    \begin{enumerate}[-]
        \item Butter can be substituted for shortening
        \item Makes approximately 12 2in diameter rounds
    \end{enumerate}
\end{enumerate}
\end{minipage}}
\vskip 0.025in
\rule{\textwidth}{0.4pt}

\vfill

\subsection{Date and Gorgonzola Scones}
\textit{Zaynab Issa, Bon Appétit (November 2021)}

\rule{\textwidth}{0.4pt}
\adjustbox{valign=t}{\begin{minipage}{0.3\linewidth}
\begin{enumerate}[]
    \item $1$ tbsp baking powder
    \item $1$ tbsp sugar
    \item $\sfrac{1}{2}$ tsp kosher salt
    \item $\sfrac{1}{4}$ tsp ground nutmeg
    \item $3$ cups ($375$g) flour
    \item $\sfrac{1}{2}$ cup chilled butter, cubed
    \item $1$ egg
    \item $1$ tsp vanilla extract
    \item $\sfrac{3}{4}$ heavy cream, chilled
    \item $10$ medjool dates, chopped
    \item $4$ oz gorgonzola, crumbled
    \item Flaky sea salt
    \item Fresh cracked pepper
\end{enumerate}
\end{minipage}}
\adjustbox{valign=t}{\begin{minipage}{0.03\linewidth}
\hfill
\end{minipage}}
\adjustbox{valign=t}{\begin{minipage}{0.66\linewidth}
\setcounter{exx}{0}
\begin{exe}
    \item Whisk baking powder, sugar, salt, nutmeg, and 3 cups flour in a large bowl. Cut in butter.
    \item Whisk together egg, vanilla, and cream. Add dates and Gorgonzola and mix gently to coat. Add to dry ingredients and mix until dough just begins to come together. Gently knead dough in the bowl until it comes together.
    \item Transfer dough to a lightly floured surface. Flatten to $\sfrac{3}{4}$'' thick. Cut into quarters, stack pieces on top of each other, and flatten dough into a $\sfrac{3}{4}$'' $8$'' square. Shape as desired and chill, uncovered, at least 15 minutes.
    \item Brush scones with cream and sprinkle with sea salt and fresh cracked pepper. Bake until tops are golden brown, 18–20 minutes. Transfer scones to a wire rack and let cool.
\end{exe}
\rule{\textwidth}{0.4pt}
\begin{enumerate}[]
    \item \textbf{Notes:}
    \begin{enumerate}[-]
        \item Makes 12 triangular scones
        \item Serve with mascarpone and honey
        \item Scones can be shaped 1 day ahead. Cover and chill.
    \end{enumerate}
\end{enumerate}

\end{minipage}}
%\vskip 0.025in
\rule{\textwidth}{0.4pt}

\vfill

\subsection{Sweet Cream Scones}
\textit{Claire Ptak, Saveur (February 2016)}

\rule{\textwidth}{0.4pt}
\adjustbox{valign=t}{\begin{minipage}{0.3\linewidth}
\begin{enumerate}[]
    \item $5$ cups flour
    \item $\sfrac{1}{2}$ cup sugar
    \item $2$ tbsp baking powder
    \item $\sfrac{1}{2}$ tsp salt
    \item $14$ tbsp chille dbutter, cubed
    \item $2\sfrac{1}{2}$ cups heavy cream
    \item $1$ egg, beaten
\end{enumerate}
\end{minipage}}
\adjustbox{valign=t}{\begin{minipage}{0.03\linewidth}
\hfill
\end{minipage}}
\adjustbox{valign=t}{\begin{minipage}{0.66\linewidth}
\setcounter{exx}{0}
\begin{exe}
    \item Sift flour, sugar, baking powder and salt in large bowl
    \item Using fingers, rub in butter until pea-sized crumbles form
    \item Add cream and stir until dough just comes together
    \item Turn dough out and form into a block, let rest 5 min
    \item Roll out and book-fold the dough and rest 5 minutes more
    \item Roll dough 1 inch thick and cut into rounds
    \item Place rounds on baking sheet in freezer for 20 minutes
    \item Egg-wash chilled scones and bake at 400$\degree$ for 20 minutes
\end{exe}
\end{minipage}}
\vskip 0.025in
\rule{\textwidth}{0.4pt}
\begin{enumerate}[]
    \item \textbf{Notes:}
    \begin{enumerate}[-]
        \item Makes two dozen 2 inch rounds
    \end{enumerate}
\end{enumerate}
\rule{\textwidth}{0.4pt}

\vfill

\subsection{Rhubarb and Angelica Jam}
\textit{Claire Ptak, Saveur (February 2016)}

\rule{\textwidth}{0.4pt}
\adjustbox{valign=t}{\begin{minipage}{0.3\linewidth}
\begin{enumerate}[]
    \item $1\sfrac{1}{4}$ lb rhubarb, chopped
    \item $1\sfrac{3}{4}$ cups sugar
    \item $2$ small angelica stalks
    \item Lemon juice ($1$ lemon)
    \item $1$ tsp chartreuse
\end{enumerate}
\vskip 0.025in
\end{minipage}}
\adjustbox{valign=t}{\begin{minipage}{0.03\linewidth}
\hfill
\end{minipage}}
\adjustbox{valign=t}{\begin{minipage}{0.66\linewidth}
\setcounter{exx}{0}
\begin{exe}
    \item In saucepan, toss rhubarb and half sugar. Let stand 1 hour.
    \item Add remaining sugar, angelica, and lemon over medium.
    \item Bring to boil, stirring to dissolve sugar, and cook 15 minutes.
    \item Stir in chartreuse and cook until jam thickens (3-5 minutes).
    \item Transfer jam to bowl/jar to cool completely
\end{exe}
\end{minipage}}
\rule{\textwidth}{0.4pt}
\begin{enumerate}[]
    \item \textbf{Notes:}
    \begin{enumerate}[-]
    \item Makes 2 cups
    \end{enumerate}
\end{enumerate}
\rule{\textwidth}{0.4pt}

\vfill
\newpage

\subsection{Holiday Breakfast Wreath}
\textit{Sarah Kate Gillingham, Kitchn (May 2019)}

\rule{\textwidth}{0.4pt}
\adjustbox{valign=t}{\begin{minipage}{0.3\linewidth}
\begin{enumerate}[]
    \item \textbf{Dough:}
    \begin{enumerate}[]
        \item $\sfrac{1}{4}$ oz active dry yeast
        \item $\sfrac{1}{4}$ cup warm water
        \item $\sfrac{1}{2}$ cup warm milk
        \item $\sfrac{1}{4}$ cup butter, softened
        \item $3$ tbsp sugar
        \item $1\sfrac{1}{2}$ tsp salt
        \item $1$ tsp ground cardamom
        \item $2$ eggs
        \item $2$ tsp lemon zest
        \item $3\sfrac{1}{2}$ cup flour
    \end{enumerate}
    \item \textbf{Filling:}
    \begin{enumerate}[]
        \item $\sfrac{3}{4}$ dried cranberries
        \item $\sfrac{1}{2}$ cup brandy
        \item $6$ tbsp butter, softened
        \item $\sfrac{1}{3}$ cup flour
        \item $\sfrac{3}{4}$ cup almonds, chopped
        \item $3$ tbsp sugar
        \item $1$ tsp lemon zest
        \item $1$ tbsp almond extract
    \end{enumerate}
    \item \textbf{Glaze:}
    \begin{enumerate}[]
        \item $1$ cup powdered sugar
        \item $2$ tbsp water
        \item $1$ tbsp lemon juice
        \item $1$ tsp ground cardamom
    \end{enumerate}
\end{enumerate}
\end{minipage}}
\adjustbox{valign=t}{\begin{minipage}{0.03\linewidth}
\hfill
\end{minipage}}
\adjustbox{valign=t}{\begin{minipage}{0.66\linewidth}
\setcounter{exx}{0}
\begin{exe}
    \item Dissolve the yeast in the water and let it foam for a minute or two. Blend in the milk, sugar, butter, salt, cardamom, eggs and lemon zest. Stir in two cups of the flour, one cup at a time. Beat for 2 minutes. Add remaining flour 1/2 cup at a time until you have a soft, workable dough.
    \item Knead dough until smooth ($sim 5- 10$ min). Add flour as needed to prevent sticking. Place in an oiled mixing bowl, cover, and let rise until doubled in size ($sim 1\sfrac{1}{2}$ hours).
    \item Soak cranberries in brandy until plump ($sim 30$ min). Drain fruit and combine with remaining filling ingredients.
    \item Punch down dough and turn it out onto a lightly floured board, kneading to release air bubbles. Roll out into a $9\times 30$-inch rectangle. Crumble the filling over the dough to within 1 inch of the edges. Tightly roll the dough long-wise, pinching edge to seal.
    \item With a sharp knife, cut roll in half lengthwise. Carefully turn the halves so the cut sides are facing up, and then loosely twist the halves around each other, keeping cut sides up.
    \item Line a baking sheet with parchment paper. Transfer the rope to the baking sheet and shape into a wreath, pinching the ends together to seal. Let it rise, uncovered, in a warm place, until puffy ($\sim 45$ min.).
    \item Preheat the oven to 350°F. Bake the wreath at 350$\degree$ until lightly browned ($\sim 25$ min). Meanwhile, stir together the ingredients for the glaze.
    \item Transfer wreath to a cooling rack and let cool. Drizzle the glaze over the warm wreath.
\end{exe}
\rule{\textwidth}{0.4pt}
\begin{enumerate}[]
    \item \textbf{Notes:}
    \begin{enumerate}[-]
       \item Wreath can be baked up to 2 days ahead. Tightly wrap and store at room temperature. Re-heat at $350\degree$ for 10 to 15 minutes, and drizzle with glaze just before serving.
       \item Makes 12 servings
    \end{enumerate}
\end{enumerate}
\end{minipage}}
\vskip 0.025in
\rule{\textwidth}{0.4pt}
\vfill

\subsection{Cardamom Knots}
\textit{Martha Collinson, Crave (September 2018)}

\rule{\textwidth}{0.4pt}
\adjustbox{valign=t}{\begin{minipage}{0.3\linewidth}
\begin{enumerate}[]
    \item \textbf{Dough:}
    \begin{enumerate}[]
        \item $50$ g butter, cubed
        \item $250$ g whole milk
        \item $500$ g white bread flour
        \item $7$ g fast-acting dried yeast
        \item $1$ tsp fine salt
        \item $50$ g sugar
        \item $1$ tsp ground cardamom
    \end{enumerate}
    \item \textbf{Filling:}
    \begin{enumerate}[]
        \item $150$ g butter, softened
        \item $100$ g dark brown sugar
        \item $2$ tsp ground cardamom
    \end{enumerate}
    \item \textbf{Syrup:}
    \begin{enumerate}[]
        \item $50$ g sugar
        \item $\sfrac{1}{2}$ tsp ground cardamom
        \item $50$ g water
    \end{enumerate}
\end{enumerate}
\end{minipage}}
\adjustbox{valign=t}{\begin{minipage}{0.03\linewidth}
\hfill
\end{minipage}}
\adjustbox{valign=t}{\begin{minipage}{0.66\linewidth}
\setcounter{exx}{0}
\begin{exe}
    \ex Heat butter in milk to melt; let cool slightly
    \item Combine flour, yeast, salt sugar, and cardamom
    \item Incorporate milk/butter mixture, stirring constantly
    \item Turn dough out and knead until smooth (5-10 min)
    \item Place in greased bowl and let rise 1-3 hours
    \item To make filling, beat butter, brown sugar, and cardamom
    \item Roll dough into $30\times50$cm rectangle and spread with filling
    \item Letter fold dough from short end, then cut into $12$ even strips
    \item Lightly stretch each strip, then tie into loose knot

    \item Transfer buns to lined baking sheet and  let prove $45$ minutes
    \item Bake knots at $200\degree$c for 15-20 minutes (or till golden brown)
    \item To make syrup, bring sugar, cardamom, and water to boil, stirring to dissolve
    \item Brush knots with syrup ASAP after removal from oven
\end{exe}
\rule{\textwidth}{0.4pt}
\begin{enumerate}[]
    \item \textbf{Notes:}
    \begin{enumerate}[-]
        \item Dough is quite dry, use less flour or more milk
        \item Chill after second prove to prevent butter leaking (n.b. butter \textit{will} leak, no matter what).
    \end{enumerate}
\end{enumerate}
\end{minipage}}
\vskip 0.025in
\rule{\textwidth}{0.4pt}
\vfill

\subsection{Rhubarb Custard Cake}
\textit{Claire Saffitz, Bon Appétit (May 2018)}

\rule{\textwidth}{0.4pt}
\adjustbox{valign=t}{\begin{minipage}{0.3\linewidth}
\begin{enumerate}[]
    \item $4$ tbsp unsalted butter, melted and cooled
    \item $1$ cup flour
    \item $\sfrac{3}{4}$ tsp baking powder
    \item $\sfrac{1}{2}$ tsp salt
    \item $2$ eggs
    \item $1$ egg yolk
    \item $1\sfrac{1}{2}$ cups sugar
    \item $\sfrac{1}{4}$ cup sour cream
    \item $2$ tbsp dark rum
    \item $2$ tsp lemon zest
    \item $13$ oz rhubarb
\end{enumerate}
\end{minipage}}
\adjustbox{valign=t}{\begin{minipage}{0.03\linewidth}
\hfill
\end{minipage}}
\adjustbox{valign=t}{\begin{minipage}{0.66\linewidth}
\setcounter{exx}{0}
\begin{exe}
    \item Whisk baking powder, salt, and 1 cup flour in a bowl
    \item Whisk eggs, yolk, and sugar until pale and thick (1 min)
    \item Whisk in butter, sour cream, rum, and lemon zest
    \item Fold in dry ingredients
    \item Scrape into buttered and floured springform (9 inch) pan
    \item Arrange rhubarb in tight rows over batter, trimming to fit
    \item Sprinkle with sugar and bake at 350$\degree$ until golden on top and browned on sides (45-55 minutes)
    \item Rest on wire rack 10 minutes before unmolding
\end{exe}
\rule{\textwidth}{0.4pt}
\begin{enumerate}[]
    \item \textbf{Notes:}
    \begin{enumerate}[-]
    \item Hand whisk rather than machine (prevents fruit from sinking)
    \item Can be made 1 day ahead; store wrapped at room temperature
    \item Makes 8 servings
    \end{enumerate}
\end{enumerate}
\end{minipage}}
\vskip 0.025in
\rule{\textwidth}{0.4pt}
\vfill
\newpage

\subsection{Double Ginger Sticky Toffee Pudding}
\textit{Claire Saffitz, Bon Appétit (October 2015)}

\rule{\textwidth}{0.4pt}
\adjustbox{valign=t}{\begin{minipage}{0.3\linewidth}
\begin{enumerate}[]
    \item \textbf{Cake:}
    \begin{enumerate}[]
        \item $\sfrac{1}{2}$ cup butter, softened
        \item $2$ cups all purpose flour
        \item $10$ oz dates, chopped
        \item $1$ tsp baking soda
        \item $2$ tsp baking powder
        \item $1\sfrac{1}{2}$ tsp kosher salt
        \item $1$ cup dark brown sugar
        \item $2$ tsp grated peeled ginger
        \item $3$ large eggs
        \item $\sfrac{1}{2}$ cup crystallised ginger, chopped
    \end{enumerate}
    \item \textbf{Toffee Sauce:}
    \begin{enumerate}[]
        \item $1$ cup dark brown sugar
        \item $1$ cup heavy cream
        \item $\sfrac{3}{4}$ cup butter
        \item $1$ tsp kosher salt
        \item Demerara sugar
    \end{enumerate}
    \item \textbf{Equipment:}
    \begin{enumerate}[]
        \item $9$ cup Bundt pan, buttered and dusted with flour
    \end{enumerate}
\end{enumerate}
\end{minipage}}
\adjustbox{valign=t}{\begin{minipage}{0.03\linewidth}
\hfill
\end{minipage}}
\adjustbox{valign=t}{\begin{minipage}{0.66\linewidth}
\setcounter{exx}{0}
\begin{exe}
    \item Toss dates and baking soda in small bowl, add $1$ cup boiling water. When dates soften ($\sim15$ min) mash with fork.
    \item Whisk baking powder, salt, and flour in medium bowl.
    \item Using an electric mixer on high, beat brown sugar, ginger, and butter in a large bowl until light and fluffy ($\sim 4$ min).
    \item Add eggs one at a time, beating to blend between additions.
    \item Add dry ingredients and date mixture in 2 additions each, starting with dry ingredients and ending with date mixture.
    \item Fold in crystallised ginger and scrape into prepared pan.
    \item Bake cake at $350^\degree$F until top is firm and springs back when pressed and a tester comes out clean, 35–45 minutes.
    \item Cool in pan $10$ min, then turn out and cool 20 min.
    \item To make sauce: bring brown sugar, cream, butter, salt, and 2 tbsp water to a boil over medium-low heat. Cook until mixture is thick enough to coat a spoon ($\sim5$-$8$ min).
    \item Poke holes all over cake with toothpick. Pour $\sfrac{1}{3}$ sauce into bundt pan and invert cake into it. Poke holes in bottom of cake and pour sauce over; let sit until absorbed ($\sim20$ min).
    \item Turn out cake (it may stick, but will eventually release) and sprinkle with demerara sugar. Serve with remaining sauce.
\end{exe}
\rule{\textwidth}{0.4pt}
\begin{enumerate}[]
    \item \textbf{Notes:}
    \begin{enumerate}[-]
        \item Can be made $1$ day ahead, store wrapped at room temperature
        \item Makes 8 servings
    \end{enumerate}
\end{enumerate}
\end{minipage}}
\vskip 0.025in
\rule{\textwidth}{0.4pt}
\vfill

\subsection{Russian Honey Cake}
\textit{Michelle Pozine, NYT Cooking (Adapted by Samin Nosrat)}

\rule{\textwidth}{0.4pt}
\adjustbox{valign=t}{\begin{minipage}{0.3\linewidth}
\begin{enumerate}[]
    \item \textbf{Cake:}
    \begin{enumerate}[]
        \item $18$ oz wildflower honey
        \item $2$ oz water
        \item $8$ oz sugar
        \item $7$ oz butter, cubed
        \item $6$ eggs
        \item $2\sfrac{1}{2}$ tsp baking soda
        \item $2\sfrac{1}{2}$ tsp kosher salt
        \item $1$ tsp ground cinnamon
        \item $16$ oz all-purpose flour
        \item $13.4$ oz dulce de leche (1 can)
        \item $4\sfrac{3}{4}$ cup heavy cream, chilled
    \end{enumerate}
\end{enumerate}
\rule{\textwidth}{0.4pt}
\begin{enumerate}[]
    \item \textbf{Notes:}
    \begin{enumerate}[-]
        \item Batter is easier to spread when warm. Keep in a warm spot (e.g. over a water bath) to make step (9) easier.
        \item Once cooled, you can trim the layer edges to achieve more uniform circles.
        \item Assembly can be messy. Don't be afraid to manhandle the cake to get it symmetric, and don't worry about rough edges poking through your outer frosting layer!
        \item Finished cake can be made up to $2$ days ahead, and leftovers can be refigerated upt to $3$ days.
        \item Makes 8-12 servings (or more)
    \end{enumerate}
\end{enumerate}
\end{minipage}}
\adjustbox{valign=t}{\begin{minipage}{0.03\linewidth}
\hfill
\end{minipage}}
\adjustbox{valign=t}{\begin{minipage}{0.66\linewidth}
\setcounter{exx}{0}
\begin{exe}
    \item Trace twelve 9-inch circles onto squares of parchment paper.
    %\item Fill a small saucepan with water and set over medium heat.
    \item Bring $\sfrac{3}{4}$ cup honey to simmer over high heat, then reduce heat to medium. Cook until honey foams ($\sim3$ min), then stir until it begins to smoke. Remove from heat and add water (honey will sputter). Whisk to combine, then transfer to spouted measuring cup and keep warm in a water bath.
    \item Combine $\sfrac{1}{4}$ cup burnt honey, $\sfrac{3}{4}$ cup honey, sugar, and butter in a large metal bowl and place over a second water bath.
    \item Crack eggs into a small bowl, whisk briefly, and set aside.
    \item Separately combine baking soda, $1\sfrac{1}{2}$ tsp salt and cinnamon.
    \item After the butter melts, whisk honey mixture to combine. Once warm, whisk in the eggs. When returned to temperature, add cinnamon mixture and whisk $\sim30$ seconds, until the batter foams and emits a curious odor. Remove from heat.
    \item Sift flour into batter in three additions, whisking to incorporate between. The batter should be completely smooth.
    \item Spoon heaping $\sfrac{1}{3}$ cup batter onto a prepared piece of parchment paper, then spread batter to evenly fill in the traced circle. Repeat for remaining batter/parchment squares.
    \item Transfer parchment papers to baking trays and bake as many at a time as you can, $\sim6$ min per batch. Cake layers will be deep caramel and springy when finished.
    \item Transfer finished layers to a wire rack until just cool, and carefully peel away the parchment paper. Then place the cooled layer \textit{back} on the parchment paper and store on a flat surface until final assembly.
    \item Reduce oven temperature to 250$\degree$. Return least attractive layer to oven and toast until deep red-brown and dry ($\sim15$ min). Cool, then pulse to crumbs in a food processor.
    \item Whisk together $\sfrac{1}{2}$ cup burnt honey, dulce de leche, and $1$ tsp salt. Add $\sfrac{3}{4}$ cup cream and mix until homogeneous. Chill completely ($\sim30$ min).
    \item Whip remaining $4$ cups cream to soft peaks. Then add honey mixture and continue whipping to medium-stiff peaks.
    \item To assemble, place a cake layer on a serving plate and spoon $\sim1$ cup frosting on top, spreading to achieve an even distribution and leaving a $\sfrac{1}{4}"$ border of the cake base un-frosted. Top with another cake layer and repeat. Following the final cake layer, using remaining frosting to coat the cake top and sides, then sprinkle with the cake crumb from step (12).
    \item Refrigerate the cake overnight, and serve chilled.
\end{exe}
\end{minipage}}
\vskip 0.025in
\rule{\textwidth}{0.4pt}
\vfill

\subsection{Salted-Caramel Chocolate Tart}
\textit{Claire Saffitz, Bon Appétit (February 2018)}

\rule{\textwidth}{0.4pt}
\adjustbox{valign=t}{\begin{minipage}{0.3\linewidth}
\begin{enumerate}[]
    \item \textbf{Pastry:}
    \begin{enumerate}[]
        \item $\sfrac{1}{3}$ cup cocoa powder
        \item $2$ tbsp sugar
        \item $\sfrac{1}{2}$ tsp kosher salt
        \item $1\sfrac{2}{3}$ cups all-purpose flour
        \item $\sfrac{3}{4}$ cup butter, cubed
        \item $1$ large egg yolk
        \item $3$ tbsp milk or water, chilled
    \end{enumerate}
    \item \textbf{Caramel Filling:}
    \begin{enumerate}[]
        \item $1\sfrac{1}{2}$ cup sugar
        \item $\sfrac{1}{8}$ tsp cream of tartar
        \item $6$ tbsp butter, cubed
        \item $\sfrac{1}{3}$ cup heavy cream
        \item $1$ tsp kosher salt
    \end{enumerate}
    \item \textbf{Chocolate Ganache:}
    \begin{enumerate}[]
        \item $4$ oz semisweet chocolate ($\leq70\%$ cacao), chopped
        \item $\sfrac{1}{2}$ cup heavy cream
        \item $2$ tbsp butter, cubed
        \item Flaky sea salt
    \end{enumerate}
\end{enumerate}
\end{minipage}}
\adjustbox{valign=t}{\begin{minipage}{0.03\linewidth}
\hfill
\end{minipage}}
\adjustbox{valign=t}{\begin{minipage}{0.66\linewidth}
\setcounter{exx}{0}
\begin{exe}
    \item Whisk together cocoa, sugar, salt, and flour. Add butter and toss to coat. Work in butter until mixture holds together.
    \item Incorporate yolk and milk to form a shaggy dough. Knead until smooth. Flatten into a $\sfrac{3}{4}$ inch disk and chill.
    \item Roll dough to a $14$ inch round $\sim\sfrac{1}{8}$ inch thick. Transfer to tart pan, pressing into edges and trimming excess. Prick bottom, and line with parchment paper and pie-weights.
    \item Blind-bake pastry $\sim12$ min at 350$\degree$. Patch cracks, and bake until firm and dry ($\sim20$ min). Cool completely.
    \item Bring sugar, cream of tartar, and $\sfrac{1}{3}$ cup water to boil, stirring to dissolve. Cook, swirling often, until amber and smoking. Stir in butter piecemeal, then whisk in cream.
    \item Pour caramel into tart shell. Let chill until set ($\geq 1$ hour)
    \item Combine chocolate, cream, and butter over a bain-marie. Stir until the ganache is smooth. Let cool until thickened.
    \item Gently work ganache over caramel, creating decorative swirls. Sprinkle with sea salt; let sit 10–15 minutes and serve.

\end{exe}
\rule{\textwidth}{0.4pt}
\begin{enumerate}[]
    \item \textbf{Notes:}
    \begin{enumerate}[-]
        \item Dough can be made 2 days ahead and chilled. Pastry can be baked 1 day ahead and kept wrapped at room temperature.
        \item Caramel can be stored 3 days. Chill, and reheat until pourable.
        \item Tart can be assembled 1 day ahead. Chill and loosely cover.
        \item Makes 8 servings
    \end{enumerate}
\end{enumerate}
\end{minipage}}
\vskip 0.025in
\rule{\textwidth}{0.4pt}

\vfill

\subsection{Black Sesame Mochi Cake with Black Sesame Caramel}
\textit{Sarah Jampel, Bon Appétit (December 2019)}

\rule{\textwidth}{0.4pt}
\adjustbox{valign=t}{\begin{minipage}{0.3\linewidth}
\begin{enumerate}[]
    \item $\sfrac{1}{4}$ cup butter, melted
    \item $3.6$ oz black sesame seeds
    \item $1\sfrac{1}{4}$ cups sugar
    \item $254$ g sweet glutinous rice flour (e.g. Koda Farms Mochiko)
    \item $1$ tsp baking powder
    \item $1$ tsp kosher salt
    \item $2$ large eggs
    \item $1$ cup whole milk
    \item $1$ tsp vanilla extract
    \item $\sfrac{1}{2}$ cup plus $5$ tbsp heavy cream
    \item $\sfrac{1}{8}$ tsp cream of tartar
    \item Toasted white sesame seeds
\end{enumerate}
\end{minipage}}
\adjustbox{valign=t}{\begin{minipage}{0.03\linewidth}
\hfill
\end{minipage}}
\adjustbox{valign=t}{\begin{minipage}{0.66\linewidth}
\setcounter{exx}{0}
\begin{exe}
    \ex Preheat oven to 350$\degree$. Generously grease an 8$\dprime$ cake pan with butter. Line with baking paper and re-grease bottom.
    \ex Toast black sesame seeds over medium until fragrant and just starting to crackle ($\sim$2 minutes). Transfer to a blender and pulse until thoroughly chopped, but not yet a paste. Reserve $\sfrac{1}{4}$ cup ground sesame.
    \ex Add $\sfrac{3}{4}$ cup sugar to remaining sesame and pulse until thoroughly combined. Transfer to a large bowl.
    \ex Use some of sesame sugar mix to thoroughly coat the prepared cake pan. Tap excess back into bowl.
    \ex Add flour, baking powder, and $\sfrac{1}{2}$ tsp salt to sesame sugar, whisking to combine.
    \ex Whisk eggs and melted butter to emulsify. Whisk in milk, vanilla, and $\sfrac{1}{2}$ cup cream. Add to the dry ingredients and stir to combine.
    \ex Transfer batter to the cake pan and bake until the cake is risen and springy with a golden brown crust ($45-55$ minutes).
    \ex Run a kinfe around the edge of the pan. Let cool 10 minutes, then double-invert onto a baking rack and let cool completely.
    \ex Whisk reserved ground sesame with remaining $5$ tbsp cream and $\sfrac{1}{2}$ tsp salt until smooth.
    \ex Bring cream of tartar, $\sfrac{1}{2}$ cup sugar and $3$ tbsp water to a boil over medium-high, stirring to dissolve sugar. Cook, stirring frequently, until mixture is the colour of light honey ($\sim$4 minutes).
    \ex Immediately remove caramel from heat and whisk in sesame cream mixture. Pour onto cooled cake and sprinkle border with white sesame seeds.
\end{exe}
\end{minipage}}
\vskip 0.025in
\rule{\textwidth}{0.4pt}
\begin{enumerate}[]
    \item \textbf{Notes:}
    \begin{enumerate}[-]
        \item Makes one 8 or 9$\dprime$ cake
    \end{enumerate}
\end{enumerate}
\vskip 0.025in
\rule{\textwidth}{0.4pt}

\vfill
\subsection{Chocolate-Almond Pear Tart}
\textit{Kristen Hall, Bon Appétit (October 2021)}

\rule{\textwidth}{0.4pt}
\adjustbox{valign=t}{\begin{minipage}{0.3\linewidth}
\begin{enumerate}[]
    \item \textbf{Tart Shell:}
    \begin{enumerate}[]
        \item $1$ tbsp Dutch-process cocoa powder
        \item $\sfrac{1}{4}$ tsp kosher salt
        \item $167$ g all-purpose flour
        \item $5$ tbsp butter
        \item $67$ g sugar
        \item $1$ large egg
    \end{enumerate}
    \item \textbf{Filling and Assembly:}
    \begin{enumerate}[]
        \item $121$ g almond flour
        \item $23$ g all-purpose flour
        \item $12$ g Dutch-process cocoa powder
        \item $\sfrac{1}{2}$ cup butter
        \item $100$ g sugar
        \item $3$ large eggs
        \item $3$ oz bittersweet chocoloate, melted and cooled
        \item $1$ tsp vanilla
        \item $\sfrac{1}{2}$ tsp almond extract
        \item $\sfrac{3}{4}$ tsp kosher salt
        \item $2$ firm pears (bartlett, anjou, or packham), thinly sliced
        \item $\sfrac{1}{4}$ cup sliced almonds
        \item powdered sugar
    \end{enumerate}
\end{enumerate}
\end{minipage}}
\adjustbox{valign=t}{\begin{minipage}{0.03\linewidth}
\hfill
\end{minipage}}
\adjustbox{valign=t}{\begin{minipage}{0.66\linewidth}
\setcounter{exx}{0}
\begin{exe}
    \ex Whisk cocoa powder, salt, and 167 grams flour in a medium bowl. Beat butter and sugar in the bowl of a stand mixer fitted with the paddle attachment on medium-low speed until combined, about 2 minutes. Beat in egg. Reduce speed to low; with motor running, add dry ingredients. Mix just until a soft dough forms. Fold in any remaining dry streaks of flour.
    \ex Gather dough into a ball. Using the heel of your hand, smear dough across surface a full arm’s length. Using a bench scrape, re-gather dough and repeat twice. Flatten into a disk, wrap tightly in plastic, and chill until firm ($\sim$2 hours).
    \ex Roll out dough on a lightly floured surface to about a $\sfrac{1}{4}\dprime$-thick round, dusting with flour as needed. Carefully transfer dough to tart pan and gently press into bottom and sides of pan, letting excess hang over edges. Run rolling pin over top of pan to shear off excess dough. Prick bottom of dough all over with a fork. Chill in freezer 1–12 hours.
    \ex Preheat oven to 325$\degree$. Bake tart shell until firm, dry, and starting to pull away from edges of pan, 14–18 minutes. Transfer to a wire rack and let cool.
    \ex Increase oven temperature to 350$\degree$. Whisk almond flour, all-purpose flour, and cocoa powder in a medium bowl. Beat butter and sugar in the clean bowl of a stand mixer fitted with a clean paddle attachment on medium-low speed until fluffy, about 5 minutes; scrape down bowl. With motor running, add eggs one at a time, incorporating completely between additions. Reduce speed; add dry ingredients and beat to combine. Add chocolate, vanilla extract, almond extract, and salt and beat until smooth.
    \ex Spread frangipane in tart shell. Top with pear slices, fanning to make an attractive pattern. Scatter almonds around tart edge and bake until pears are tender and frangipane barely jiggles, $\sim$40–45 minutes. Transfer to wire rack to cool.
\end{exe}
\end{minipage}}
\vskip 0.025in
\rule{\textwidth}{0.4pt}
\begin{enumerate}[]
    \item \textbf{Notes:}
    \begin{enumerate}[-]
        \item Just before serving, dust edges of tart with powdered sugar.
        \item Dough can be kept chilled for up to a week, and frozen for up to a month/
        \item Tart can be baked two days ahead. Store tightly wrapped at room temperature.
        \item Makes one 9$\dprime$ tart.
    \end{enumerate}
\end{enumerate}
\vskip 0.025in
\rule{\textwidth}{0.4pt}

\newpage

\section{Desserts}
\subsection{Panna Cotta}
\textit{Gourmet (August 1997)}

\rule{\textwidth}{0.4pt}
\adjustbox{valign=t}{\begin{minipage}{0.3\linewidth}
\begin{enumerate}[]
    \item $1$ envelope unflavored gelatin ($1$ tbsp)
    \item $2$ tbsp cold water
    \item $2$ cups heavy cream
    \item $1$ cup half and half
    \item $\sfrac{1}{3}$ cup sugar
    \item $1\sfrac{1}{2}$ tsp vanilla extract
\end{enumerate}
\end{minipage}}
\adjustbox{valign=t}{\begin{minipage}{0.03\linewidth}
\hfill
\end{minipage}}
\adjustbox{valign=t}{\begin{minipage}{0.66\linewidth}
\setcounter{exx}{0}
\begin{exe}
    \item In small saucepan, add gelatin to water to soften (1 minute)
    \item Heat gelatin on low until dissolved, then remove from  heat
    \item In large saucepan, bring dairy and sugar to boil over medium high heat
    \item Remove from heat and stir in vanilla and gelatin
    \item Divide mixture among eight $\sfrac{1}{2}$ cup ramekins, let cool to room temperature and then chill, covered, at least 4 hours
\end{exe}
\end{minipage}}
\vskip 0.025in
\rule{\textwidth}{0.4pt}
\begin{enumerate}[]
    \item \textbf{Notes:}
    \begin{enumerate}[-]
    \item Can be made one day in advance
    \item To serve, dip ramekins in hot water for 3 seconds; run thin knife around edges and invert
    \item Makes 8 servings
    \end{enumerate}
\end{enumerate}
\rule{\textwidth}{0.4pt}
\vfill

\subsection{Red Wine Syrup}
\textit{Julie Smolyansky, The Kefir Cookbook (2018)}

\rule{\textwidth}{0.4pt}
\adjustbox{valign=t}{\begin{minipage}{0.3\linewidth}
\begin{enumerate}[]
    \item $1$ bottle red wine
    \item $\sfrac{1}{2}$ cup honey
    \item $1$ tsp pepper, freshly cracked
\end{enumerate}
\end{minipage}}
\adjustbox{valign=t}{\begin{minipage}{0.03\linewidth}
\hfill
\end{minipage}}
\adjustbox{valign=t}{\begin{minipage}{0.66\linewidth}
\setcounter{exx}{0}
\begin{exe}
    \item Bring wine, honey, and pepper to boil
    \item Cook until alcohol is gone and reduced to a thick syrup
    \item Strain through fine sieve to remove pepper
\end{exe}
\end{minipage}}
\vskip 0.025in
\rule{\textwidth}{0.4pt}
\begin{enumerate}[]
    \item \textbf{Notes:}
    \begin{enumerate}[-]
    \item Wine should reduce by about $\sfrac{2}{3}$ in volume
    \item Product will be substantially thicker once chilled
    \item Makes approximately $\sfrac{2}{3}$ cup
    \end{enumerate}
\end{enumerate}
\rule{\textwidth}{0.4pt}
\vfill
\newpage

\section{Cocktails}
\subsection{Blackberry-Cucumber Mule}
\textit{Maggie Hoffman, The One Bottle Cocktail (2018)}

\rule{\textwidth}{0.4pt}
\adjustbox{valign=t}{\begin{minipage}{0.3\linewidth}
\begin{enumerate}[]
    \item $3$ blackberries
    \item $2$ slices cucumber ($\sfrac{1}{4}$ inch)
    \item $5$ mint leaves
    \item $2$ oz vodka
    \item $\sfrac{3}{4}$ oz lemon juice
    \item $\sfrac{1}{2}$ oz simple syrup
    \item $2\sfrac{1}{2}$ oz ginger beer
\end{enumerate}
\end{minipage}}
\adjustbox{valign=t}{\begin{minipage}{0.03\linewidth}
\hfill
\end{minipage}}
\adjustbox{valign=t}{\begin{minipage}{0.66\linewidth}
\setcounter{exx}{0}
\begin{exe}
    \item Muddle blackberries, cucumber, and mint in cocktail shaker
    \item Add vodka, lemon juice, and syrup; fill with ice, and shake
    \item Double strain into collins glass and top with ginger beer
    \item Garnish with mint leaf and cucumber wheel
\end{exe}
\rule{\textwidth}{0.4pt}
\begin{enumerate}[]
    \item \textbf{Notes:}
    \begin{enumerate}[-]
    \item Blackberry struggles to come through, maybe up?
    \item Shake with crushed ice
    \end{enumerate}
\end{enumerate}
\end{minipage}}
\vskip 0.025in
\rule{\textwidth}{0.4pt}
\vfill

\subsection{Midnight Sparkler}
\textit{Kat Boytsova, Epicurious (December 2016)}

\rule{\textwidth}{0.4pt}
\adjustbox{valign=t}{\begin{minipage}{0.3\linewidth}
\begin{enumerate}[]
    \item $4$ oz crème de violette
    \item $2$ oz orange juice
    \item $2$ oz lemon juice
    \item $2$ oz gin
    \item sparkling wine
    \item lemon wedges/twists
\end{enumerate}
\end{minipage}}
\adjustbox{valign=t}{\begin{minipage}{0.03\linewidth}
\hfill
\end{minipage}}
\adjustbox{valign=t}{\begin{minipage}{0.66\linewidth}
\setcounter{exx}{0}
\begin{exe}
    \item Combine crème de violette, orange juice, lemon juice, and gin in cocktail shaker. Shake with ice until cold
    \item Double strain into prepared glasses, top with sparkling wine, and garnish with lemon wedge/twist
\end{exe}
\rule{\textwidth}{0.4pt}
\begin{enumerate}[]
    \item \textbf{Notes:}
    \begin{enumerate}[-]
    \item Cocktail base (sans sparkling wine) can be made 1 day ahead
    \item Makes 4 servings
    \end{enumerate}
\end{enumerate}
\end{minipage}}
\vskip 0.025in
\rule{\textwidth}{0.4pt}
\vfill

\subsection{Saint-Florent}
\textit{Bon Appétit (October 2014)}

\rule{\textwidth}{0.4pt}
\adjustbox{valign=t}{\begin{minipage}{0.3\linewidth}
\begin{enumerate}[]
    \item \textbf{Honey syrup:}
    \begin{enumerate}[]
        \item $\sfrac{1}{4}$ cup honey
        \item $2$ tbsp hot water
    \end{enumerate}
    \item \textbf{Cocktail:}
    \begin{enumerate}[]
        \item $6$ tbsp gin
        \item $3$ tbsp lime juice
        \item $2$ tbsp Aperol
        \item $1$ tbsp honey syrup
        \item $2$ dashes Angostura bitters
        \item sparkling wine
        \item lime wheels
    \end{enumerate}
\end{enumerate}
\end{minipage}}
\adjustbox{valign=t}{\begin{minipage}{0.03\linewidth}
\hfill
\end{minipage}}
\adjustbox{valign=t}{\begin{minipage}{0.66\linewidth}
\setcounter{exx}{0}
\begin{exe}
    \item Combine honey and hot water in small, sealable jar. Shake to disolve honey and let cool completely.
    \item Combine gin, lime juice, aperol, honey syrup, and angostura in cocktail shaker. Shake over ice to chill.
    \item Double strain into two coupe glasses, top with sparkling wine, and garnish with lime wheel.
\end{exe}
\rule{\textwidth}{0.4pt}
\begin{enumerate}[]
    \item \textbf{Notes:}
    \begin{enumerate}[-]
    \item Cocktail base (sans sparkling wine) can be made 1 day ahead
    \item Capelletti may be substituted for Aperol
    \item Makes 2 servings
    \end{enumerate}
\end{enumerate}
\end{minipage}}
\vskip 0.025in
\rule{\textwidth}{0.4pt}
\vfill
\newpage

\end{document}
